\chapter{Unipept protein counts distribution}\label{ch:appendix-unipept-protein-counts-distribution}
\begin{table}[h!]
    \centering
    \begin{tabular}{|l|r|}
        \hline
        \textbf{\# Proteïnes} & \textbf{\# Peptides}\\
        \hline
        $\geq 1$     & 1\thinspace342\thinspace470\thinspace764 \\
        $\geq 2$     & 355\thinspace979\thinspace324            \\
        $\geq 10$    & 38\thinspace697\thinspace210             \\
        $\geq 10^2$  & 2\thinspace921\thinspace879              \\
        $\geq 10^3$  & 217\thinspace922                         \\
        $\geq 10^4$  & 13\thinspace008                          \\
        $\geq 10^5$  & 118                                      \\
        $\geq 10^6$  & 0                                        \\ \hline
    \end{tabular}
    \caption{Het aantal verschillende tryptische peptiden die matchen met op zijn minst $x$ proteïnes uit UniProtKB 2023\_03. Een klein voorbeeld: stel dat we de tryptiche peptide \texttt{ACACA} zoeken. Deze heeft 24\thinspace694 matches in UniProtKB. Dit wil zeggen dat dit één van de 13\thinspace008 peptiden is met meer dan $10^4$ matches.}
    \label{tab:number_peptide_matches}
\end{table}

\begin{table}[h!]
    \centering
    \begin{tabular}{|l|r|}
        \hline
        \textbf{\# NCBI taxonomy rank} & \textbf{\# Peptides} \\ \hline
        root & 12\thinspace369 \\ \hline
        superkingdom & 43 \\ \hline
        kingdom & 16 \\ \hline
        subkingdom & 0 \\ \hline
        superphylum & 0 \\ \hline
        phylum & 8 \\ \hline
        subphylum & 7 \\ \hline
        superclass & 1 \\ \hline
        class & 18 \\ \hline
        subclass & 1 \\ \hline
        superorder & 0 \\ \hline
        order & 0 \\ \hline
        infraorder & 1 \\ \hline
        superfamily & 0 \\ \hline
        family & 2 \\ \hline
        subfamily & 0 \\ \hline
        tribe & 1 \\ \hline
        subtribe & 0 \\ \hline
        genus & 55 \\ \hline
        subgenus & 0 \\ \hline
        species\_group & 0 \\ \hline
        species\_subgroup & 0 \\ \hline
        species & 200 \\ \hline
        subspecies & 0 \\ \hline
        strain & 1 \\ \hline
        varietas & 0 \\ \hline
        forma & 0 \\ \hline
    \end{tabular}
    \caption{Verdeling van de 13\thinspace000 verschillende peptiden die met meer $\geq 10^4$ proteïnes matchen. We zien dat voor de overgrote meerderheid de LCA resulteert op de root. Slechts voor 200 peptides is het resultaat op soortniveau.}
    \label{tab:peptides_distribution}
\end{table}

\begin{table}[h!]
    \centering
    \begin{tabular}{|r|l|}
        \hline
        \textbf{\# Peptides} & \textbf{LCA} \\ \hline
        119 & Alphainfluenzavirus influenzae \\ \hline
        32 & Human immunodeficiency virus \\ \hline
        14 & Hepatitis B virus \\ \hline
        9 & Betainfluenzavirus influenzae \\ \hline
        4 & Orthoflavivirus denguei \\ \hline
        3 & Simian immunodeficiency virus \\ \hline
        1 & Alcidodes juglans \\ \hline
        1 & Bacillus subtilis \\ \hline
        1 & Bacteroides thetaiotaomicron \\ \hline
        1 & Cannabis sativa \\ \hline
        1 & Capsicum baccatum \\ \hline
        1 & Echinocucumis hispida \\ \hline
        1 & Geissoloma marginatum \\ \hline
        1 & Homo sapiens \\ \hline
        1 & Human immunodeficiency virus \\ \hline
        1 & Kalanchoe fedtschenkoi \\ \hline
        1 & Leucosceptrum canum \\ \hline
        1 & Loxia curvirostra \\ \hline
        1 & Marinilactibacillus piezotolerans \\ \hline
        1 & Melanocenchris jacquemontii \\ \hline
        1 & Merops nubicus \\ \hline
        1 & Morbillivirus hominis \\ \hline
        1 & Phalaenopsis pulcherrima \\ \hline
        1 & Phormidesmis priestleyi \\ \hline
        1 & Rhodobacter maris \\ \hline
    \end{tabular}
    \caption{Overzicht van de geassocieerde soort voor de 200 peptides uit tabel~\ref{tab:peptides_distribution} die op soortniveau eindigen.
    Van de 200 peptides eindigt de meerderheid in een LCA die geassocieerd is met een virus (zoals HIV of influenza). Dit komt doordat er veel onderzoek gedaan wordt naar de verschillende bestaande rassen. Deze zitten allemaal in de UniProt Knowledgebase.}
    \label{tab:peptides_species}
\end{table}
