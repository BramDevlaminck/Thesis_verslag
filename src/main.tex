\documentclass[11pt,dutch,faculty=we,layout=titlefont,underline=false,titleUppercase=true,titleUnderline=true]{ugent2016-report}
\usepackage[dutch]{babel}
\usepackage{fontspec,unicode-math}
\usepackage{amsmath}
\usepackage{subcaption}
\usepackage{csquotes}
\usepackage{subfloat}
\usepackage{float}
\usepackage[outputdir=../out]{minted}

\usepackage{booktabs,makecell}
% Figures
%---------
\usepackage{graphicx, ugent2016-assets}
\graphicspath{{./figures/}}
\usepackage[inkscapelatex=false]{svg}
\usepackage{footmisc}

% Bibliography settings
%-----------------------
\usepackage{tikz}
\usepackage{biblatex}
\addbibresource{bibliography.bib}

\usepackage[hyperfootnotes=true]{hyperref}
\hypersetup{
    colorlinks,
    linkcolor={black},
    citecolor={blue!50!black},
    urlcolor={blue!80!black}
}


\setlength{\parindent}{0em}


\author{Bram Devlaminck}
\title{Unipept:\newline geavanceerde indexstructuren \newline voor identificatie en analyse\newline van arbitraire peptiden}
%\subtitle{Hello}
\academicyear{2023--2024}
\programme{Informatica}
\studentnumber{01902993}
\email{bram.devlaminck@ugent.be}

\titletext{%
    Promotoren: prof.\ dr.\ Peter Dawyndt, prof. dr.\ ir.\ Bart Mesuere \\%
    Begeleiders: dr. Pieter Verschaffelt, Tibo Vande Moortele
    \\ \\%
    {\small Masterproef ingediend tot het behalen van de academische graad van\\%
    Master of Science in de Informatica%
    }%
}


\begin{document}

    \maketitle


    \setmonofont[Scale=MatchLowercase,Contextuals={Alternate}]{Jetbrains Mono}

% ------------- CONSENT OF USE -----------

    \chapter*{Toelating tot bruikleen}

De auteur geeft de toelating deze masterproef voor consultatie beschikbaar te stellen en delen van de masterproef te kopiëren voor persoonlijk gebruik.
Elk ander gebruik valt onder de bepalingen van het auteursrecht, in het bijzonder met betrekking tot de verplichting de bron uitdrukkelijk te vermelden bij het aanhalen van resultaten uit deze masterproef.
\\ \\
Bram Devlaminck \\ \today
% TODO: voeg handtekening toe? maak wel dat het bestand niet op github staat, en ook niet de versie met dat handtekening

% ------------ TABLE OF CONTENTS ---------
    \tableofcontents
    \newpage


% =====================================================================
% Main matter
% =====================================================================


    \chapter{Introductie}\label{ch:introductie}


\section{Inleiding}\label{sec:inleiding}

\subsection{Genomica, transcriptomica \& proteomica}\label{subsec:genomica-transcriptomica-&-proteomica}
Eiwitten zijn alomtegenwoordig en spelen een belangrijke rol in ons dagelijkse leven.
Ze garanderen een correcte werking van belangrijke processen binnen elk organisme.
Hieronder vallen de levensprocessen van ons eigen lichaam, maar ook die van dieren, planten, bacteriën en zelfs virussen.
Om deze processen te analyseren zijn er meerdere benaderingen mogelijk.
Een eerste mogelijkheid is aan de hand van het onderzoeksgebied van de proteomica.
Dit is de studie van alle eiwitten die binnen een enkel organisme tot expressie kunnen komen.
Hierbij probeert men te begrijpen hoe eiwitten in elkaar zitten, hoe deze met elkaar en binnen een bepaalde omgeving met elkaar interageren en wat hun belangrijkste functie is.
Naast proteomica bestaan er nog twee andere gerelateerde disciplines.
\\ \\
De eerste alternatieve discipline is genomica, het onderzoek naar het genoom.
Het genoom van een organisme is de collectie van al het DNA binnen een organisme.
Dit stelt voor welke proteïnen mogelijks door het organisme geconstrueerd kunnen worden.
Een belangrijk verschil met proteomica is dat DNA instructies voorstelt voor de productie van alle mogelijke proteïnen die het organisme kan maken.
Het geeft dus geen informatie over de proteïnen die op dat moment in de tijd actief zijn.
Het is een voorstelling van wat het organisme kan, niet wat het op \textit{dit} moment aan het doen is.
Belangrijk is dat ongeveer 98\% van het menselijke genoom niet-coderend is, wat wil zeggen dat dit deel van het DNA niet omgezet kan worden naar een betekenisvolle proteïne.
\\ \\
De andere discipline is de transcriptomica.
Deze discipline onderzoekt het transcriptoom van een organisme, wat de verzameling is van alle RNA moleculen die in het organisme aanwezig zijn.
Het transcriptoom is een belangrijke indicatie van welke delen uit het DNA effectief proteïnen encoderen.
Dit omdat RNA, meer specifiek messenger RNA (mRNA) en transfer RNA (tRNA), een belangrijk onderdeel is van het proces om DNA om te zetten naar proteïnen.
\\ \\
Onze focus ligt vooral binnen het veld van de metaproteomica.
het \textit{meta} prefix zegt dat de te analyseren stalen niet van één organisme zijn, maar van meerdere organismen (typisch binnen hetzelfde ecosysteem).
Dit maakt de analyse moeilijker aangezien proteïnen van verschillende organismen gelijkaardige aminozuursequenties kunnen hebben (al dan niet door toeval).
Meer specifiek is het doel van metaproteomica om op zoek te gaan naar de taxa die hoort bij een verzameling van peptiden.
Een een veelvoorkomende categorie van peptiden zijn tryptische peptiden.

\subsection{Tryptische peptiden}\label{subsec:tryptische-peptiden}
Tryptische peptiden zijn peptiden die ontstaan na het knippen van proteïnen aan de hand van trypsine.
Dit is een protease (eiwitafbrekend enzym) dat proteïnen opsplitst in meerdere peptiden.
Er bestaan nog andere proteases, maar trypsine is veruit de populairste door zijn eenduidig gedrag en efficiëntie.
\\ \\
Trypsine zal eiwitten knippen na elk voorkomen van lysine (K) of arginine (R) indien het eerstvolgende aminozuur geen proline (P) is.
Deze vuistregel is echter niet perfect.
Soms zal trypsine, een locatie waar normaal geknipt moet worden, missen.
Dit noemen we een \textit{missed cleavage}.
Figuur~\ref{fig:trypsine} bevat een voorbeeld van de werking.

\begin{figure}[H]
    \centering
    \includesvg[width=0.9\textwidth]{trypsine_verwerking}
    \caption{Voorbeeld van de werking van trypsine op 2 proteïnen. De aminozuren in het rood zijn lysine (K) of arginine (R), waarna trypsine knipt (behalve als het eerstvolgende aminzoruur proline (P) is). De tweede proteïne bevat een voorbeeld waar niet geknipt wordt na lysine, doordat het opeenvolgende aminozuur proline is~\cite{phdPieterUnipept}.}
    \label{fig:trypsine}
\end{figure}

Om de peptiden uit een experiment te kunnen gebruiken bij computeranalyses moeten deze omgezet worden naar een stringrepresentatie.
Dit is is echter een moeilijk en ingewikkeld proces.
Eerst wordt de massa/ladingsverhouding (m/z) van peptiden aan de hand van een massa spectrometer gemeten.
Daarna worden deze resultaten aan de hand van diverse zoekprocessen (via zogenaamde zoekmachines) omgezet naar de stringvoorstelling van de peptide.
Deze sequenties vormen de input voor tools zoals Unipept.

\subsection{Unipept}\label{subsec:unipept-introductie}
Unipept biedt allerhande tools aan om stalen uit het onderzoeksveld van de metaproteomica te analyseren, maar er is ook een onderdeel (UMGAP~\cite{UMGAP_paper}) gericht op het analyseren van stalen uit de metagenomica.

\begin{itemize}
    \item \textbf{Unipept Web application} Dit is de originele Unipept tool en is publiek beschikbaar op \url{https://unipept.ugent.be}.
    Met de gebruiksvriendelijke \textit{user interface} wordt analyseren van metaproteomica data beschikbaar gesteld.
    De resultaten van deze analyse worden aan de hand van visualisaties en tabellen aan de gebruiker voorgesteld.
    Deze kunnen vervolgens makkelijk geëxporteerd worden (bijv. voor analyse in andere tools).
    \item \textbf{Unipept CLI} Dit is een \textit{power-user} tool om analyses uit te voeren op grotere stalen.
    \item \textbf{Unipept API} Dit is een collectie van \textit{endpoints} die andere applicaties toelaat om de functionaliteit van Unipept integreren.
    \item \textbf{Unipept Desktop} Dit is de recentste toevoeging aan het Unipept ecosysteem en laat toe dat onderzoekers niet noodzakelijk met de Unipept servers moeten communiceren om analyses uit te voeren.
    Deze applicatie combineert de voordelen van de web app, CLI en API en laat toe om lokaal stalen te analyseren, gebruik makende van een gebruiksvriendelijke UI\@.

\end{itemize}

Op dit moment is Unipept gericht op de analyse van tryptische peptiden.
De reden hiervoor is de manier waarop de achterliggende indexstructuur opgebouwd wordt.
Dit opbouwen gaat in grote lijnen als volgt:

\begin{enumerate}
    \item Haal alle proteïnen en bijbehorende taxonomische en functionele annotaties op uit de UniProtKB databank.
    \item Splits deze proteïnen volgens de vuistregel die trypsine volgt.
    \item Sla alle resulterende tryptische peptiden op in een indexstructuur.
\end{enumerate}

Deze aanpak heeft als voordeel dat we op een efficiënte manier tryptische peptiden kunnen opzoeken (samen met de bijbehorende annotaties).
Er is echter een belangrijke keerzijde aan deze manier van werken.
Het zoeken van niet-tryptische peptiden (hieronder vallen ook peptiden met \textit{missed cleavage}) is inefficiënt.
Dit komt doordat tijdens het opbouwen van de Unipept indexstructuur de vuistregel gevolgd wordt, en elke peptide in de indexstructuur strikt tryptisch is.
\\ \\
Op dit moment is er wel een \textit{workaround} die toe laat deze peptiden toch te zoeken.
Dit heeft echter wel een significante impact op de performantie.
Dit verklaart dan ook de nood aan een nieuwe indexstructuur die toe laat volledig willekeurig gesplitste peptiden te zoeken.
\newline
Voor een gedetailleerdere beschrijving over Unipept en het onderzoeksveld van metaproteomica is het aangeraden om de inleiding van het doctoraat van Dr. Pieter Verschaffelt te lezen~\cite{phdPieterUnipept}.
Dit vormde een duidelijke en goede basis voor deze inleiding.


\section{Probleemstelling}\label{sec:probleemstelling}
Het probleem waarvoor een oplossing gezocht wordt is het snel terugvinden van willekeurige peptiden in een eiwitdatabank.
Bij het vinden van een match moet het daarna mogelijk zijn de informatie op te halen die hoort bij alle proteïnen die matchen.
Binnen het onderzoeksgebied van de informatica kunnen we dit probleem als volgt herformuleren:
``In een grote verzameling van middellange strings (alle eiwitten in onze databank), moeten we voor een verzameling van korte strings (peptiden) terugvinden in welke van deze middellange strings ze voorkomen.''
\\ \\
Belangrijk hierbij is dat dit niet alleen snel gebeurt, maar we ook proberen het vereiste geheugen tot een minimum te beperken.
Dit omdat de datasets waarmee gewerkt wordt extreem groot zijn.
\\ \\
Tot slot willen we ook inexacte matching toevoegen tijdens het zoeken.
Door dit te doen kunnen we beter omgaan met kleine fouten die voorkomen tijdens het uitlezen van de experimentele stalen.
Om dit allemaal te bereiken is het doel van deze thesis om meerdere datastructuren uit te werken, te implementeren in Rust, en tot slot te testen.
Het gebruik van Rust laat ons toe om extreem hoge performantie te verkrijgen (vergelijkbaar met C en C++~\cite{rustPerformantie}) in combinatie met \textit{memory safety}\footnote{\textit{Memory safety} is een eigenschap die verzekert dat programma's enkel gebruik kunnen maken van geldige geheugenlocaties en geen \textit{undefined behaviour} zoals \textit{buffer overflows}, \textit{dangling pointers} en andere geheugen gerelateeerde fouten kunnen vertonen.}.
Bovendien zijn sommige delen van Unipept al geschreven in Rust (zie UMGAP~\cite{UMGAP_paper, UMGAP_source}).
Dit laat toe om waar mogelijk bestaande code te hergebruiken.


\section{Benchmark Datasets}\label{sec:datasets}

\subsection{Proteïnedatabanken}\label{subsec:proteine-databanken}
Om te testen hoe goed een implementatie is en hoe deze zich verhoudt ten opzichte van bestaande implementaties is het belangrijk om representatieve datasets te gebruiken.
Deze datasets zijn allemaal eiwitdatabanken die een subset vormen van UniProtKB (meer specifiek UniProtKB 2023\_04)~\cite{UniprotKB}.
UniProt bestaat uit twee onderdelen (gegeven statistieken zijn voor release 2023\_04).
\begin{enumerate}
    \item Swiss-Prot: Dit is een kleinere, manueel gecureerde, dataset met 570 157 eiwitsequenties.
    \item TrEMBL: Deze dataset bevat 251 600 768 sequenties en is dus veel groter dan Swiss-Prot.
    Het belangrijkste verschil is dat deze dataset \textbf{niet} manueel gecureerd is.
\end{enumerate}
Om het vergelijken van de performantie van verschillende implementaties praktisch te houden maak ik tijdens het testen gebruik van 2 subsets hiervan.

\paragraph{Swiss-Prot} Deze databank is één van de twee standaard onderdelen van UniProt.
Een kort overzicht van alle statistieken is terug te vinden in tabel~\ref{tab:swissprot_eigenschappen}.
Figuur~\ref{fig:swissprot_aminozuur} en figuur~\ref{fig:swissprot_length} geven meer inzicht in de distributie van de aminozuren en lengte van de proteïnen.

\begin{table}[h!]
    \centering
    \begin{tabular}{l l}
        Metriek                   & Waarde      \\
        \hline\hline
        Totaal aantal sequenties  & 569 619     \\
        Totale lengte             & 205 954 074 \\
        Minimale proteïnelengte   & 2           \\
        Maximale proteïnelengte   & 35 213      \\
        Gemiddelde proteïnelengte & 361.56      \\
        Mediaan proteïnelengte    & 295         \\
        \hline
    \end{tabular}
    \caption{Eigenschappen van de Swiss-Prot databank uit UniProt 2023\_04 die gebruikt wordt om de indexstructuur op te bouwen.}
    \label{tab:swissprot_eigenschappen}
\end{table}


\begin{figure}[H]
    \centering
    \includegraphics[width=0.7\linewidth]{swissprot_aminozuur_voorkomens}
    \caption{Aantal voorkomens per aminozuur voor alle proteïnen in de Swiss-Prot databank uit UniProt 2023\_04.}
    \label{fig:swissprot_aminozuur}
\end{figure}

\begin{figure}[H]
    \centering
    \subfloat[Overzicht van de lengtedistributie van alle proteïnen]{\includegraphics[width=0.485\linewidth]{swissprot_length_distribution_large}}
    \hfill
    \subfloat[Ingezoomd beeld van de lengtedistributie tot 1000 aminozuren lang]{\includegraphics[width=0.485\linewidth]{swissprot_length_distribution_small}}
    \caption{Lengtedistributie van de proteïnen in de Swiss-Prot databank.}\label{fig:swissprot_length}
\end{figure}

Doordat het gebruikte invoerbestand reeds verwerkt werd door een deel van de Unipept pipeline, is er een klein verschil tussen het totaal aantal sequenties in tabel~\ref{tab:swissprot_eigenschappen} en wat eerder aangegeven werd.
Hierbij worden onder andere sequenties met een onbekend taxon id verwijderd, wat het kleine verschil verklaart.

\paragraph{Human-Prot} Deze dataset is samengesteld aan de hand van drie referentiedatabanken afkomstig van UniProtKB\@.
Dit zijn de Human Genome~\cite{proteomes_homo_sapiens}, Influenza B~\cite{proteomes_infuenza_b} en Human Papillomavirus~\cite{proteomes_human_papillomavirus} databank.
Opnieuw komen deze allemaal uit UniProt 2023\_04.

Deze Human-Prot databank is kleiner dan Swiss-Prot waardoor het testen tijdens ontwikkeling sneller is.
Tabel~\ref{tab:humanprot_eigenschappen} somt enkele belangrijke metrieken op over deze dataset.
Figuur~\ref{fig:humanprot_aminozuur} en~\ref{fig:humanprot_length} gaan dieper in op een aantal details.

\begin{table}[h!]
    \centering
    \begin{tabular}{ l l }
        Metriek                   & Waarde     \\
        \hline\hline
        Totaal aantal sequenties  & 82 695     \\
        Totale lengte             & 30 293 046 \\
        Minimale proteïnelengte   & 2          \\
        Maximale proteïnelengte   & 35 991     \\
        Gemiddelde proteïnelengte & 366.32     \\
        Mediaan proteïnelengte    & 204        \\
        \hline
    \end{tabular}
    \caption{Eigenschappen van de Human-Prot databank die gebruikt om de indexstructuur op te bouwen.}
    \label{tab:humanprot_eigenschappen}
\end{table}

\begin{figure}[H]
    \centering
    \includegraphics[width=0.7\linewidth]{humanprot_aminozuur_voorkomens}
    \caption{Aantal voorkomens per aminozuur voor alle proteïnen in de Human-Prot databank.}
    \label{fig:humanprot_aminozuur}
\end{figure}

\begin{figure}[H]
    \centering
    \subfloat[Overzicht van de lengtedistributie van alle proteïnen]{\includegraphics[width=0.485\linewidth]{humanprot_length_distribution_large}}
    \hfill
    \subfloat[Ingezoomd beeld van de lengtedistributie tot 1000 aminozuren lang]{\includegraphics[width=0.485\linewidth]{humanprot_length_distribution_small}}
    \caption{Lengtedistributie van de proteïnen in de Human-Prot databank.}\label{fig:humanprot_length}
\end{figure}

We kunnen concluderen dat zo goed als alle letters gebruikt worden (ook al zijn er maar 20 aminozuren).
Dit komt doordat sommige letters eigenlijk een soort wildcard voorstellen.
Zo staat ``X'' voor elk mogelijk aminozuur, ``Z'' voor ``Q'' of ``E'',\ldots

Ook zien we dat de verdeling van de proteïnelengtes in de Swiss-Prot en Human-Prot datasets vergelijkbaar zijn.
Het zwaartepunt ligt bij Swiss-Prot wel iets later (rond 100 vs rond 200).

\subsection{Peptide zoekbestanden}\label{subsec:peptide-zoek-bestanden}
De zoekperformantie van onze indexstructuur is een erg belangrijk aspect.
Om dit te meten hebben we bij elke eiwitdatabank een lijst aan peptiden die we proberen te zoeken.
Voor beide databanken zijn enkele datasets opgesteld.

\subsubsection{Swiss-Prot}
Voor deze databank hebben we enkele zoekbestanden voorzien.
Twee bestanden die gesampled zijn en een reeks aan real-life stalen.
De twee artificiële bestanden zijn zo gekozen dat de ene enkel tryptische peptiden bevat, terwijl de andere ook peptiden bevat met \textit{missed cleavage}.
De eerste kan dus op dit moment al efficiënt door Unipept behandeld worden, terwijl dit voor de tweede niet mogelijk is.

\paragraph{Artificiële stalen}
Tabel~\ref{tab:artifiele_bestanden_statistieken} bevat in kolom twee en drie een kort overzicht met statistieken voor deze gesamplede bestanden.

\begin{table}[H]
    \hspace*{-0.35cm} % this table is just a bit too wide to, just manually move it a bit to the left to "center" it
    \begin{tabular}{l l l l}
        Metriek                    & SP zonder \textit{missed cleavages} & SP met \textit{missed cleavages} & Human-Prot    \\
        \hline\hline
        Totaal aantal sequenties   & 100 000                             & 100 000                          & 250 000       \\
        Totale lengte              & 1 605 909                           & 2 544 356                        & 2 458 834 046 \\
        Minimale proteïnelengte    & 5                                   & 5                                & 1             \\
        Maximale proteïnelengte    & 50                                  & 93                               & 12            \\
        Gemiddelde proteïnelengte  & 16.06                               & 25.44                            & 9.84          \\
        Mediaan proteïnelengte     & 13                                  & 23                               & 10            \\
        Aantal vindbare peptiden   & 67 375                              & 62 581                           & 250 000       \\
        Aantal tryptische peptiden & 100 000                             & 4107                             & 102 659       \\
        \hline
    \end{tabular}
    \caption{Eigenschappen van de verschillende zoekbestanden. De kolommen die beginnen met SP bevatten de statistieken van de Swiss-Prot zoekbestanden. De laatste kolom bevat de statistieken voor het zoekbestand dat hoort bij de Human-Prot eiwitdatabank.}
    \label{tab:artifiele_bestanden_statistieken}
\end{table}

\paragraph{Experimentele stalen}
Om de performantie beter te beoordelen, gebruiken we ook enkele stalen uit experimenten met een kleine micro-organisme gemeenschap, namelijk SIHUMIx\footnote{Simplified human intestinal microbiota}~\cite{SIHUMI_first_introduction, SIHUMI_frequently_used}.
Aangezien dit effectieve stalen zijn, bevatten deze \textit{missed cleavages} die natuurlijk ontstaan zijn.
Tabel~\ref{tab:sihumi_zoekbestanden} bevat de belangrijkste statistieken voor elk zoekbestand.
Elke kolom stelt een staal voor met als bestandsnaam \texttt{S<XX>.txt}.
Deze zoekbestanden worden in combinatie met de Swiss-Prot databank gebruikt tijdens het testen.

\begin{table}[H]
    \centering
    \begin{tabular}{ l l l l l l l }
        Metriek                    & S03     & S05     & S07     & S08     & S11     & S14     \\
        \hline\hline
        Totaal aantal sequenties   & 25 000  & 25 000  & 24 424  & 25 000  & 25 000  & 25 000  \\
        Totale lengte              & 420 544 & 420 423 & 373 633 & 316 114 & 366 922 & 430 674 \\
        Minimale proteïnelengte    & 6       & 6       & 6       & 6       & 6       & 6       \\
        Maximale proteïnelengte    & 50      & 50      & 47      & 43      & 50      & 50      \\
        Gemiddelde proteïnelengte  & 16.82   & 16.82   & 15.30   & 12.64   & 14.68   & 17.23   \\
        Mediaan proteïnelengte     & 15      & 16      & 14      & 12      & 14      & 16      \\
        Aantal vindbare peptiden   & 2570    & 2698    & 3652    & 4135    & 3792    & 2761    \\
        Aantal tryptische peptiden & 17 263  & 162     & 152     & 207     & 155     & 242     \\
        \hline
    \end{tabular}
    \caption{Eigenschappen van de SIHUMIx zoekbestanden. Elke kolom stelt een staal voor met als bestandsnaam \texttt{S<XX>.txt}.}
    \label{tab:sihumi_zoekbestanden}
\end{table}

\subsubsection{Human-Prot}
Voor deze databank hebben we één zoekbestand bestaande uit HLA-peptiden.
Dit zijn korte, niet-tryptische peptiden uit het immunopeptidomics onderzoeksveld.
Hierdoor kan Unipept op dit moment niet gebruikt worden om dit soort stalen te analyseren.
Elke peptide in dit zoekbestand is een sample van een proteïne uit de Human-Prot databank.
Hierdoor zijn alle peptiden die we zoeken effectief vindbaar in de dataset.
Een kort overzicht van enkele eigenschappen is terug te vinden in de laatste kolom van tabel~\ref{tab:artifiele_bestanden_statistieken}.
\\ \\
Appendix~\ref{ch:appendix-statistieken-zoekbestanden} bevat aanvullende grafieken over bovenstaande zoekbestanden.
Deze tonen voor elk zoekbestand de distributie van de aminozuren en de distributie van de peptidelengte.


\section{Benchmark hardware}\label{sec:benchmark-hardware}
Alle benchmarks zijn uitgevoerd op een virtuele machine van Unipept.
Tabel~\ref{tab:Matt_hardware} bevat een overzicht van de hardware van de fysieke machine en hoeveel toegekend is aan de VM\@.
Al deze informatie, en ook over de andere machines gebruik door Unipept, kan terug gevonden worden op de Unipept GitHub wiki~\cite{unipept_infrastructure}.

\begin{table}[h!]
    \centering
    \begin{tabular}{p{0.20\linewidth}p{0.45\linewidth}p{0.25\linewidth}}
        Onderdeel         & Fysieke server                                                            & Virtuele Machine    \\
        \hline\hline
        CPU               & 2x Intel Xeon 4410Y (12 cores / 24 threads, 2 - 3.9 GhZ, 30 MiB cache)    & 12 threads          \\
        RAM               & 768 GiB                                                                   & 128 GiB             \\
        Opslag            & 6x 16 TiB HDD (3.5 inch, 7.2K RPM SATA), 4x 3.84 TiB SSD (2.5 inch, SATA) & 1 TiB SSD, 4TiB SSD \\
        Besturingssysteem & Debian 12 (met Proxmox)                                                   & Ubuntu 22.04 LTS    \\
        \hline
    \end{tabular}
    \caption{Hardwarespecificaties van de fysieke server en virtuele machine die gebruikt worden tijdens het testen. Deze virtuele machine draait samen met nog enkele andere VMs op de server.}
    \label{tab:Matt_hardware}
\end{table}


    \chapter{Suffixbomen}\label{ch:suffix-bomen}
Een eerste datastructuur die het mogelijk maakt om snel kleine strings in een groot aantal andere strings op te zoeken zijn suffixbomen.
Meer precies eigenlijk een gegeneraliseerde suffixboom.
\\ \\
We behandelen deze datastructuur als eerste omdat hij vrij intuïtief is en de makkelijkste van alle mogelijke opties.
Bovendien kan een goede tijdscomplexiteit bereikt worden aangezien het zoeken in een suffixboom in $O(n)$ tijd kan (met $n$ de lengte van de zoekstring, in ons geval is dit een peptide).
Het opbouwen van de suffixboom kan ook in lineaire tijd gebeuren, al zei het dan lineair in de totale lengte van alle proteïnen in de databank (die we aanduiden met $m$). % TODO: vraag wat moest komen na: al zei het dan lineair in de totoale lengte van de ?

\section{Wat zijn suffixbomen?}\label{sec:wat-zijn-suffix-bomen?}
suffixbomen zijn een soort veralgemening van tries (prefix bomen).
Door er voor te zorgen dat het laatste teken uniek is zal elke suffix van de inputstring uniek zijn (elke suffix is dus nooit de prefix van een andere suffix).
Dit zorgt er voor dat elke suffix een eigen blad in de boom zal krijgen.
Dit is dan ook van waar de naam suffixboom komt.
Elk pad tot een blad in de boom zal exact 1 suffix voorstellen uit de inputstring waarvoor de boom gebouwd is.
Als voorbeeld stelt figuur~\ref{fig:suffix_tree_example} de suffixboom voor van de string \texttt{acacgt\$}.
Merk op dat we \texttt{`\$`} als uniek eindteken gebruiken.

\begin{figure}[H]
    \center
    \begin{tikzpicture}
    [
        level 1/.style = {sibling distance = 3.5cm, level distance = 2cm},
        level 2/.style = {sibling distance = 1.5cm, level distance = 2cm}
    ]

        \node[draw, circle] {}
        child {
            node[draw, rounded corners] {\texttt{t\$}}
            edge from parent node [above] {\texttt{t\$}}
        }
        child {
            node[draw, rounded corners] {\texttt{gt\$}}
            edge from parent node [below] {\texttt{gt\$}}
        }
        child {
            node[draw, circle] {}
            child {
                node[draw, rounded corners] {\texttt{cgt\$}}
                edge from parent node [left] {\texttt{gt\$}}
            }
            child {
                node[draw, rounded corners] {\texttt{cacgt\$}}
                edge from parent node [right] {\texttt{acgt\$}}
            }
            edge from parent node [right] {\texttt{c}}
        }
        child {
            node[draw, circle] {}
            child {
                node[draw, rounded corners] {\texttt{acacgt\$}}
                edge from parent node [left] {\texttt{gt\$}}
            }
            child {
                node[draw, rounded corners] {\texttt{acgt\$}}
                edge from parent node [right] {\texttt{acgt\$}}
            }
            edge from parent node [below] {\texttt{ac}}
        }
        child {
            node[draw, rounded corners] {\texttt{\$}}
            edge from parent node [above] {\texttt{\$}}
        }
        ;
    \end{tikzpicture}
    \caption{suffixboom voor the string \texttt{acacgt\$}}\label{fig:suffix_tree_example}

\end{figure}

Natuurlijk is dit niet efficiënt om effectief op deze manier op te slaan.
Als de tekst lengte $n$ heeft, heeft de suffixboom voor de tekst ten hoogste $2n - 1$ toppen en $2n - 2$ bogen.
Het aantal toppen en bogen is dus $\Theta(n)$.
Jammer genoeg vraagt alleen het opslaan van alle prefixen in de bladeren $\Theta(n^2)$ geheugen~\cite{AD3_ukkonen}.
In de plaats kunnen we pointers bijhouden naar het begin en einde van een substring in de originele string.
Dit zorgt er voor dat we geen kopie meer moeten opslaan van de originele string in elk blad!
We moeten dit zelfs niet in elk blad bijhouden!
We kunnen simpelweg bij elke boog tussen de toppen de labels bijhouden.
Het label van het blad kunnen we daarna reconstrueren door de labels van de bogen op weg naar dit blad te achter elkaar te plaatsen.
Door dit te doen is de nodige opslag per top een constante, en is het geheugengebruik lineair.
Figuur~\ref{fig:suffix_tree_example_indices} toont hoe dit er in de praktijk uit ziet.
Merk op dat de eind-index exclusief is.
Een boog met waarde \texttt{1,3}, stelt dus de substring \texttt{ca} voor uit het voorbeeld.

\begin{figure}[H]
    \center
    \begin{tikzpicture}
    [
        level 1/.style = {sibling distance = 2.5cm},
        level 2/.style = {sibling distance = 1cm}
    ]

        \node[draw, circle] {}
        child {
            [fill] circle (2pt)
            edge from parent node [above] {5,7}
        }
        child {
            [fill] circle (2pt)
            edge from parent node [below] {4,7}
        }
        child {
            node[draw, circle] {}
            child {
                [fill] circle (2pt)
                edge from parent node [left] {4,7}
            }
            child {
                [fill] circle (2pt)
                edge from parent node [right] {2,7}
            }
            edge from parent node [right] {1,2}
        }
        child {
            node[draw, circle] {}
            child {
                [fill] circle (2pt)
                edge from parent node [left] {4,7}
            }
            child {
                [fill] circle (2pt)
                edge from parent node [right] {2,7}
            }
            edge from parent node [below] {0,2}
        }
        child {
            [fill] circle (2pt)
            edge from parent node [above] {6,7}
        }
        ;
    \end{tikzpicture}
    \caption{suffixboom voor de string \texttt{acacgt\$} gebruik makende van indices}\label{fig:suffix_tree_example_indices}

\end{figure}


\section{Het Algoritme van Ukkonen}\label{sec:Ukkonen}
Het algoritme van Ukkonen om suffixbomen op te bouwen~\cite{Ukkonen1995} is niet eenvoudig.
De pseudocode en de theoretische beschrijving in de originele paper is beide vrij complex.
Het algoritme komt echter uitgebreid aan bod in een heleboel andere publicaties en boeken~\cite{Gusfield1997, AD3_ukkonen, CCB_course, Ukkonen_CCB}.
%\begin{enumerate}
%    \item Het boek \textit{Algorithms on Strings, Trees and Sequences}~\cite{Gusfield1997}.
%    \item De cursus \textit{Algoritmen \& Datastucturen 3} aan UGent gegeven door prof. Gunnar Brinkmann~\cite{AD3_ukkonen}.
%    \item De cursus Computational Challenges in Bioinformatics gegeven door prof. dr. Jan Fostier and prof. dr. Peter Dawyndt.
%    Naast een cursus met wat info over het algoritme van Ukkonen is er ook een implementatie van dit algoritme gemaakt door Jan Fostier in C++~\cite{Ukkonen_CCB}.
%\end{enumerate}

\subsection{Kotlin}\label{subsec:kotlin}
Ik ben begonnen met een implementatie van Ukkonen's algoritme in Kotlin zodat ik me niet op taal-specifieke problemen zou moeten focussen (vooral restricties rond \textit{borrowing} in Rust).
Hier was gelukkig de referentiecode van prof. Jan Fostier~\cite{Ukkonen_CCB} een grote hulp omdat dit het mogelijk maakte om tijdens het debuggen te zien wat de toestand van het programma is na $x$ stappen.
\\ \\
Één van de verschillen tussen mijn implementatie en de referentie-implementatie is de representatie van de kinderen.
In mijn implementatie is dit aan de hand van een HashMap in plaats van een array van pointers, simpelweg uit gemak zodat ik rechtstreeks een karakter als sleutel kon gebruiken, en die niet moest omzetten naar een index.
Om dit prototype te maken heb ik gekozen voor Kotlin boven Python aangezien Kotlin performanter is en ook een aangename ontwikkelingservaring biedt.
Hierdoor is het mogelijk om toch de test datasets op te bouwen in een redelijke tijd.
\\ \\
Uiteindelijk bleek de grootste struikelblok in het implementeren van Ukkonen enkele off-by-1 fouten.
Aangezien je tijdens het algoritme eigenlijk werkt met substrings, maar deze opgeslagen worden aan de hand van hun begin- en eind-index wordt het debuggen veel omslachtiger.
Tot slot had ik op het einde ook enkele bugs die niet voorkwamen in kleinere voorbeelden die met de hand uit te werken waren, wat daardoor relatief wat tijd vroeg om op te lossen.

\subsection{Rust}\label{subsec:rust}

\subsubsection{Eerste ervaring}
Aangezien dit mijn eerste ervaring was met Rust deel ik graag even mijn eerste bevindingen mee.
Zelf heb ik eerst \textit{the Rust book} gelezen~\cite{the_rust_book}.
Wat op zich veel goede informatie heeft, maar naar mijn ervaring soms te veel.
Een veelvoorkomend patroon in het boek is dat een deel van een concept geïntroduceerd wordt en dat je dan als lezer geïnformeerd wordt dat je er nu nog niet over moet denken, dat er meer informatie hierover komt in een later hoofdstuk.
Dit zorgt ervoor dat het soms moeilijk is de bomen door het bos te zien.
Zeker aangezien Rust vaak verschillende syntaxen heeft om hetzelfde te doen, waardoor je als lezer moeilijk je kunt focussen op de essentiële delen.
\\ \\
Om de meest essentiële basis componenten van Rust wat onder de knie te krijgen heb ik daarna de oefeningen van Rustlings~\cite{rustlings} gemaakt op aanraden van mede thesis student Stijn De Clercq.
Dit zijn een reeks aan erg kleine oefeningen die meestal in maximaal enkele minuten gemaakt zijn, maar er voor zorgen dat je toch al eens in contact komt met alle basisonderdelen van Rust.
Naar mijn mening is dit dus zeker een waardevolle toevoeging tijdens het leren van Rust.

\subsubsection{Boomstructuren}
\begin{quote}
    \textit{Rust is known to be notorious difficult when it comes to certain data structures like linked lists, trees, etc. \cite{rust_difficulty_quote}}
\end{quote}
Deze quote komt rechtstreeks uit een Medium artikel en toont direct aan dat het maken van een suffixboom in Rust niet-triviaal ging zijn.
De oorzaak hiervoor ligt bij het \textit{ownership} systeem van Rust.
Dit systeem zorgt er voor dat slechts één variabele eigenaar kan zijn van een stukje data.
In dit geval kan dus slechts één top een andere top opslaan, of er een \textit{mutable reference} naar hebben.
Meer praktisch wil dit dus zeggen dat slechts één top een \textit{pointer} kan hebben naar een andere top, met de toelating om die top aan te passen (wat nodig is tijdens het opbouwen van de boom. Er worden namelijk nog kindere toegevoegd en toppen gesplitst)
Dit is een groot probleem aangezien ouders pointers naar kinderen moeten hebben, de kinderen een pointer naar hun ouder, en er dan ook nog eens pointers zijn voor de suffix links.
\\ \\
Als oplossing hiervoor introduceert Rust het \texttt{Rc<T>} datatype.
Hierbij gaat Rust afstappen van zijn standaard \textit{ownership} systeem en gebruik maken van Reference Counting.
Pas wanneer alle referenties weg zijn zal het geheugen automatisch vrij gegeven worden.
De beperking hierbij is echter dat deze referenties \textit{immutable} zijn, dit volstaat niet tijdens het opbouwen van de boom.
\\ \\
Als oplossing hiervoor heeft Rust dan weer het \textit{interior mutability} patroon~\cite{interior_mutability} aan de hand van het datatype \texttt{Refcell}.
Dit laat toe om data toch aan te passen, ook al is een reference immutable.
Aangezien dit de standaard Rust regels doorbreekt, is dit \texttt{unsafe}\footnote{Dit is code waarvan de compiler niet kan nagaan als die aan alle voorwaarden voldoet die nodig zijn om \textit{memory safety} te kunnen garanderen. Dit sleutelwoord bestaat zodat de programmeur meer vrijheid zou kunnen krijgen om bepaalde patronen toch toe te kunnen passen. De verantwoordelijkheid om correct het geheugen te gebruiken wordt hier bij de programmeur gelegd. Een andere reden om \texttt{unsafe} te gebruiken is om bepaalde interacties met hardware uit te voeren. Deze zijn inherent onveilig en zouden anders onmogelijk zijn.} en kan Rust \textit{at compile-time} geen \textit{memory safety} meer garanderen.
\texttt{Refcell} zal gelukkig wel de nodige code invoegen zodat runtime memory safety wel gegarandeerd kan worden.
Mogelijke foutieve geheugen operaties zullen dus tijdens het uitvoeren van het programma gedetecteerd worden, \textbf{ten koste van performantie}.
\\ \\
Maar zelfs nu blijft er nog altijd een probleem.
Geheugen dat beheerd wordt aan de hand van \textit{reference counting} zal enkel vrijgegeven kunnen worden indien de \textit{reference counter} op 0 staat.
Er zijn echter scenario's waar dit nooit zal gebeuren.
Namelijk bij cyclische verwijzingen, een patroon dat jammer genoeg erg vaak voor komt (in ons geval bv een ouder die een pointer heeft naar een kind, en een kind een pointer naar de ouder).
Als oplossing hiervoor introduceert Rust dan weer het \texttt{Weak<T>} datatype.
\\ \\
Dit is duidelijk erg ingewikkeld, en introduceert ook nog eens performance overhead die niet nodig lijkt.
Een optie zou natuurlijk zijn door expliciet het \texttt{unsafe} keyword te gebruiken wat toe laat de ownership regels van Rust volledig uit te schakelen (zowel compile-time als run-time).
Het nadeel hiervan is natuurlijk dat we dan de garanties van memory safety kwijt zijn, wat net één van de hoofdredenen is om Rust te gebruiken.
Dit was dus geen mogelijke optie.
Gelukkig is er een alternatieve manier waar ik op gestoten ben, een arena-based implementatie~\cite{rust_arena_trees}.
Het idee hierbij is dat er één arena gemaakt wordt waarbij ownership erg simpel is.
In mijn implementatie is dit bijvoorbeeld een \texttt{Vector}.
Alle toppen worden hierbij in deze ene vector opgeslagen.
In plaats van pointers naar elkaar houden bij te houden zullen de toppen indexen bijhouden.
Deze index stelt de index in de arena van de top voor waarnaar anders een pointer wordt bijgehouden.
\\ \\
Na het maken van deze ontwerpaanpassingen bleef slechts één moeilijkheid over.
Uitzoeken hoe de cursor (die bij houdt waar we zijn in de boom tijdens het bouwen), de input string en de boom zelf zich van elkaar moeten verhouden in het ownership systeem.
Uiteindelijk viel dit vrij makkelijk uit te zoeken.
Het omzetten van de resterende Kotlin code naar Rust was erg simpel en bijna een één op één vertaling.
Het enige verschil is dat ik in de Rust implementatie gebruik heb gemaakt van een array om de kinderen bij te houden in plaats van een HashMap.

\subsubsection{Geheugen efficiëntie}
\begin{quote}
    \textit{And then I went and invented a null pointer.
    And if you use a null pointer you either have to check every reference or you risk disaster. \cite{null_mistake}}
\end{quote}
\textit{Null pointers} worden ook wel \textit{the billion-dollar mistake} genoemd vanwege het grote aantal bugs dat ze veroorzaken.
Daarom voorziet Rust een andere manier om de waarde \textit{null} voor te stellen.
Dit wordt gedaan aan de hand van de \texttt{Option<T>} enum.

\begin{minted}{Rust}
enum Option<T> {
    None,
    Some(T),
}
\end{minted}

Deze enum heeft 2 mogelijke waarden: \texttt{None} of \texttt{Some(T)}.
\texttt{None} is het equivalent van \textit{null}, terwijl \texttt{Some(T)} wil zeggen dat de waarde verschillend is van null, meer specifiek is de waarde \texttt{T}.
Aangezien het grootste deel van wat bijgehouden wordt per top eigenlijk pointers zijn maakte ik veelvoudig gebruik van deze Option-enum.
Alle pointers in een top kunnen namelijk null zijn.
De \textit{parent pointer} moet nullable zijn aangezien de root geen parent heeft, de \textit{child pointers} moeten allemaal nullable zijn omdat bladeren geen kinderen hebben (en in de interne toppen zijn niet alle kinderen altijd nodig) en de suffix-links moeten nullable zijn aangezien niet elke top een suffix link heeft naar een andere top.
\\ \\
Dit werkte perfect en kon mooi afgehandeld worden op de idiomatische manier die overeenkomt met goede Rust code.
Na de eerste benchmarks bleek het geheugengebruik echter problematisch.
Bijna exact 2x zo hoog als de equivalente C++ implementatie.
Om zo'n drastisch verschil in geheugenverbruik te kunnen verklaren moest er wel iets fundamenteel verschillen aan de manier dat toppen hun data bijhouden.
Al snel bleek dat het gebruik van \texttt{Option<usize>} als datatype in plaats van \texttt{usize} 8 bytes aan overhead per index had.
Dit is inderdaad exact het dubbele geheugenverbruik op een 64-bit machine aangezien een \texttt{usize} 8 bytes groot is.
Dit valt makkelijk te controleren aan de hand van de \texttt{std::mem::size\_of} functie, deel van de Rust standaard bibliotheek.
Onderstaand voorbeeld toont dat dit inderdaad het geval is.
\begin{minted}{Rust}
assert_eq!(mem::size_of::<Option<usize>>(), 16);
assert_eq!(mem::size_of::<usize>(), 8);
\end{minted}

Als oplossing heb ik uiteindelijk mijn eigen \textit{null} value gedefinieerd gebruik makende van een \textit{trait}\footnote{Een trait in Rust definieert een functionaliteit dat een bepaald type heeft, en kan delen met andere types}.
Deze oplossing verslaat volledig het doel van de \texttt{Option<T>} enum, maar is jammergenoeg nodig omdat het gewoonweg niet acceptabel is het geheugenverbruik te verdubbelen hiervoor.
Bovendien blijft memory safety gegarandeerd aangezien het foutief indexeren van de NULL-value (\texttt{usize::MAX} in dit geval) een index-out-of-bounds error creëert.
Wat tijdens runtime gedetecteerd wordt en dus geen verdere problemen geeft (afgezien van een mogelijke crash van het programma).

\begin{minted}{Rust}
/// Custom trait implemented by types that have a value that represents NULL
pub trait Nullable<T> {
    const NULL: T;

    fn is_null(&self) -> bool;
}

/// Type that represents the index of a node in the arena part of the tree
pub type NodeIndex = usize;

impl Nullable<NodeIndex> for NodeIndex {
    /// Use usize::MAX as NULL value since this will in practice never be reached.
    /// It is not possible to create 2^64-1 nodes (on a 64-bit machine).
    /// This would simply never fit in memory
    const NULL: NodeIndex = usize::MAX;

    fn is_null(&self) -> bool {
        *self == Self::NULL
    }
}
\end{minted}

\subsection{Performantie}\label{subsec:performantie}
Natuurlijk is het belangrijk dat de implementatie performant (en correct) is.
Aangezien we ook over een bestaande C++ implementatie van Ukkonen's algoritme beschikken, was dit een perfecte maatstaf.
Uiteindelijk heb ik één aanpassingen moeten maken in deze C++ code om een eerlijke vergelijking uit te voeren.
Oorspronkelijk werd er in elke top plaats gehouden voor 256 mogelijke kinderen.
Dit was veel te hoog voor onze usecase.
Er zijn nl. slechts 20 aminozuren en enkele \textit{wildcard characters}.
Dit verklaart onmiddellijk waarom het geheugengebruik ongeveer een factor 10 hoger was dan nodig.
Uiteindelijk ben ik gegaan voor een implementatie (zowel in Rust als C++) waarin plaats gehouden wordt voor 28 kinderen.
Dit zijn de 26 letters van het alfabet + \texttt{`\#`} + \texttt{`\$`}.
\texttt{`\#`} en \texttt{`\$`} worden gebruikt als resp. scheidingsteken en eindteken.
Dit is ook wat al gebeurde in de bestaande C++ implementatie.
\\ \\
Een andere aanpak zou kunnen zijn om HashMaps te gebruiken.
Het totale geheugenverbruik zal hierdoor afnemen naar ongeveer 60\% van het huidige verbruik, maar ten koste van performantie tijdens het zoeken (wat net erg belangrijk is).
Hoe dan ook blijft het geheugen verbruik extreem groot, welke implementatie ook gekozen wordt.
\\ \\
Het vergelijken van de implementaties heb ik opgesplitst in 2 stukken:
\begin{enumerate}
    \item het opbouwen van de index-structuur.
    \item Zoeken in de index-structuur
\end{enumerate}

\subsubsection{Opbouwen}
Om een representatief resultaat te krijgen is het opbouwen van de boom 10x uitgevoerd en zijn de gemiddelden van de resultaten genomen.
Om de uitvoeringstijd en het geheugenverbruik te meten heb ik gebruik gemaakt van het \texttt{time} commando.
De resultaten hiervan zijn terug te vinden in figuur~\ref{fig:tree_building}.
\begin{figure}[H]
    \centering
    \subfloat[Tijd nodig om de suffixboom op te bouwen]{\includegraphics[width=\linewidth]{building_tree_time}}\\[4ex] % [4ex] om wat extra vertical spacing in te voegen

    \subfloat[Maximaal gebruikt geheugen tijdens het opbouwen van de suffixboom]{\includegraphics[width=\linewidth]{building_tree_memory}}
    \caption{Vergelijking van C++ en Rust voor het opbouwen van de suffixboom. De tijd en het geheugengebruik is gemeten gebruik makende van het \texttt{time} commando met als invoerbestand de Swiss-Prot of Human-Prot eiwitdatabank.}\label{fig:tree_building}
\end{figure}

Uit deze grafieken vallen 2 duidelijke conclusies te trekken.
\begin{enumerate}
    \item De implementatie in Rust is $\pm$ 33\% sneller
    \item Het geheugenverbruik is erg vergelijkbaar.
    Dit valt te verwachten aangezien beide implementaties 8 bytes nodig hebben per \textit{pointer} en evenveel plaats voorzien voor de kinderen.
    Het kleine verschil valt te verklaren vanwege 1 veld dat ik niet bij houdt tijdens het opbouwen, dat wel gebruikt wordt in de C++ implementatie.
    Dit is de diepte van de top in de boom.
    Op de enkele plaatsen waar dit nodig is kan ik gebruik maken van andere variabelen om tot een equivalent resultaat te komen.
\end{enumerate}

\subsubsection{Zoeken}
Bij het zoeken zijn er 2 belangrijke manieren om te vergelijken.
\begin{enumerate}
    \item Zoek totdat we weten als er een match bestaat voor de peptide of niet, en stop dan.
    \item Zoek totdat er een match is, en doorzoek daarna de volledige subboom om alle informatie van de kinderen op te halen.

\end{enumerate}

\paragraph{Zoek een match}
De reden voor deze manier van zoeken is dat het mogelijk is om info te propageren van de bladeren tot bovenin de boom.
In ons geval is dit bijvoorbeeld de LCA van de taxon IDs op voorhand te berekenen.
Het zoeken van de LCA die overeenkomt met alle proteïnen waar de gevonden peptide mee matcht kan dus al stoppen vanaf er een match is.
\\ \\
Grafiek~\ref{fig:performance_match_tree} toont de nodige tijd om alle peptiden van de gebruikte zoekbestanden éénmalig te zoeken totdat er een (mis)match was voor de peptide.
De grafiek bevat de gemiddelde resultaten van 5000 uitvoeringen, maar zelfs dan bleven de resultaten wat schommelen.
Doordat de te meten tijd zo klein is, kan de kleinste invloed van omgevingsfactoren al voor een zichtbaar verschil zorgen.
Dit kan bv een achtergrondproces zijn, maar ook invloed van een andere VM die op de fysieke machine bezig is.
Dit was ook merkbaar tijdens het testen waar dat de verschillen tussen 2 opeenvolgende uitvoeringen vaak groter waren dan het verschil tussen de C++ en Rust implementatie.
Toch kunnen we besluiten dat de C++ implementatie een beetje performanter is aangezien dit in elk zoekbestand (nipt) sneller is (soms zelfs minder dan een milliseconde).

\begin{figure}[H]
    \centering
    \includegraphics[width=\linewidth]{search_match_performance_tree}
    \caption{Uitvoeringstijd in milliseconden voor het zoeken tot een match voor alle zoekbestanden. Deze resultaten zijn het gemiddelde van 5000 uitvoeringen. 1 iteratie wordt gezien als 1x elke peptide die deel is van het zoekbestand te zoeken in de suffixboom, en te stoppen wanneer er een (mis)match gevonden is. Het meten van de tijd is gebeurd in de code zelf.}
    \label{fig:performance_match_tree}
\end{figure}

Het verschil met de huidige implementatie van Unipept is gigantisch.
Daar duurt het op dit moment 2 minuten en 12 seconden om alle peptiden van het Swiss-Prot zoekbestand zonder \textit{missed cleavages} te zoeken,
en maar liefst 30 minuten 37 seconden voor het zoekbestand met \textit{missed cleavages}!
Dit is maar liefst $\frac{132 000}{152.98} = 857$ en $\frac{1 837 000}{140.64} = 13 000$ keer trager!
Als keerzijde van de medaille gebruikt Unipept op dit moment hiervoor slechts 6.7 GiB geheugen, en dit kan zelfs nog naar beneden.
Dit is ongeveer 13 keer lager!

\paragraph{Zoek match en haal informatie over kinderen op}
De reden dat dit belangrijk is, is dat alle bladeren in deze subboom de proteïnen voorstellen waar dat de gevonden peptide een deel van is.
De relevante informatie over de huidige peptide is daarom de informatie die verbonden is met deze proteïnen.

Figuur~\ref{fig:performance_all-occurrences_tree} bevat een overzicht van de nodige zoektijd voor beide implementaties op alle zoekbestanden.
We zien duidelijk dat er hier een gigantisch verschil is tussen de C++ en Rust implementatie.
Vermoedelijk komt dit door de andere \textit{memory layout} die ontstaat doordat de Rust implementatie 1 grote Vector gebruikt, terwijl de C++ implementatie losse toppen gebruikt die verspreid liggen in het geheugen.

\begin{figure}[H]
    \centering
    \includegraphics[width=\linewidth]{search_all-occurrences_performance_tree}
    \caption{Uitvoeringstijd inclusief het doorzoeken van de volledige subboom na match voor alle zoekbestanden. Deze resultaten zijn het gemiddelde van 10 uitvoeringen. 1 iteratie wordt gezien als 1x elke peptide die deel is van het zoekbestand te zoeken in de suffixboom, en bij een match de volledige resterende subboom te doorzoeken. Dit toont de tijd die nodig is om informatie uit de bladeren op te halen voor alle proteïnen waar een peptide substring van is. Het meten van de tijd is gebeurd in de code zelf.}
    \label{fig:performance_all-occurrences_tree}
\end{figure}


\section{Taxon ID aggregatie}\label{sec:taxon-id-aggregatie}
Één van de belangrijkste analyses die Unipept aanbiedt, is de taxonomische analyse waarin uitgezocht wordt met welke organismen de peptiden uit een staal overeenkomen.
Aangezien peptiden kunnen matchen met proteïnen die uit verschillende organismen komen moet er een manier gekozen worden om deze informatie te aggregeren, of te beslissen van welk organisme dit komt met de grootste kans.
Aangezien er geen manier is om met zekerheid te zeggen uit welke proteïne de peptide komt (als er meerdere opties zijn) gaat Unipept de info conservatief veralgemenen.
Anders gezegd: Unipept zal enkel info geven die geldt voor alle gematchte proteïnen.
Één van deze stukjes informatie is het Taxon ID\@.
In plaats van een lijst van alle mogelijke IDs te geven (wat een extreem grote lijst kan zijn), en wat zou vereisen de volledige subboom na het vinden van een match af te lopen, gaan we deze taxon IDs gaan aggregeren aan de hand van een strategie gebruik makende van de NCBI taxonomy database~\cite{NCBI_original_article, NCBI_update}.
Met andere woorden, we gaan dus op zoek naar de kleinste gemeenschappelijke voorouder van alle taxon IDs die in de bladeren van de subboom zitten van een bepaalde top.
Hiervoor bestaan verschillende strategieën die al uitgewerkt zijn in UMGAP, en die hier herbruikbaar waren.
\\ \\
Origineel was het plan om LCA* te gebruiken als aggregatie strategie.
Dit is een heuristiek om de LCA (lowest common ancestor) te berekenen.
Hierbij zoeken we de meest specifieke taxon in de boom die ofwel een ouder of kind is van elke taxon in de boom.
Anders gezegd is dit de LCA van een lijst taxa, nadat we alle taxa verwijdert hebben die ouder zijn van minstens één taxon in die lijst~\cite{UMGAP_paper}.
Het voordeel hiervan is dat we iets langer exactere info kunnen behouden.
Want bij LCA zelf zal het resultaat altijd één zijn vanaf één top in de subtree dit als LCA heeft (1 is namelijk ouder van alle andere taxons!).
\\ \\
Het idee was dat we niet elke keer naar de bladeren moeten gaan om het taxon ID te berekenen van 1 top, maar dat dit ging kunnen op basis van de taxon IDs van de directe kinderen van de top.
Op deze manier gingen we met één bottom-up sweep van de boom alle taxon IDs kunnen berekenen.
Dit bleek echter niet mogelijk omdat kijken naar de directe kinderen een ander resultaat geeft dan gebruik maken van de bladeren van de subboom.
Figuur~\ref{fig:lca*_diff} toont een minimaal voorbeeld uitgewerkt voor beide strategieën.
De licht grijze toppen zijn ingevuld zijn aan de hand van aggregatie, terwijl de zwarte toppen gegeven zijn.

\begin{figure}[H]
    \centering
    \subfloat[LCA* op basis van de bladeren]{
        \begin{tikzpicture}
            \node [gray] {1}
            child {node [gray] {1}
            child {node {9606}}
            child {node {10566}}}
            child {node {9606}
            };
        \end{tikzpicture}
    }\hspace{0.25\textwidth}%
    \subfloat[LCA* op basis van de kinderen]{
        \begin{tikzpicture}
            \node [gray] {9606}
            child {node [gray] {1}
            child {node {9606}}
            child {node {10566}}}
            child {node {9606}
            };
        \end{tikzpicture}
    }
    \caption{Minimaal voorbeeld van de 2 aggregatie manieren gebruik makende van LCA*. De grijze toppen zijn berekend aan de hand van een LCA*, terwijl de zwarte toppen gegeven zijn.}\label{fig:lca*_diff}
\end{figure}

Onderstaande uitleg behandelt de werkwijze voor bovenstaande figuren.
\begin{itemize}
    \item Het toepassen van LCA* voor het berekenen van de top op basis van de bladeren van de boom (\{9606, 10566, 9606\}) heeft als resultaat 1 voor de root van de boom.
    9606 en 10566 zijn geen ouder of kind van elkaar, dus zal LCA* hetzelfde doen als LCA\@.
    De kleinste gemeenschappelijke ouder van deze 2 taxons is 1.
    \item Het toepassen van LCA* op basis van de directe kinderen geeft als resultaat 9606.
    Dit val simpel te verklaren aangezien de LCA* van de linker subboom 1 is.
    Als we daarna dan de LCA* van \{1, 9606\} nemen wordt 1 verwijderd aangezien dit een ouder is van 9606.
    De LCA van 9606 is natuurlijk gewoon zichzelf!
\end{itemize}

Het berekenen van de LCA* op de eerste manier is echter niet schaalbaar voor de volledige suffixboom.
Om een idee van grootorde te geven: de suffixboom voor de Swiss-Prot dataset bevat 328 922 516 toppen in het totaal, waarvan 206 523 693 bladeren.
\\ \\
Daarom hebben we uiteindelijk toch voor de standaard LCA aggregatie manier gekozen.
Deze laat wel toe de toppen op deze efficiëntere manier te aggregeren.
UMGAP biedt 2 manieren aan om LCA te doen.
Gebruik makende van RMQ (Range Minimum Queries) en een boom-gebaseerde structuur.
Zelf maak ik gebruik van de RMQ implementatie aangezien deze significant sneller was (8 min 58 sec vs 20 min en 25 sec voor de Swiss-Prot databank).
Tot slot heb ik ook eens vergeleken hoe groot de behaalde tijdswinst is bij het gebruik kunnen maken van de directe aggregatie op de kinderen, vergeleken met moeten aggregeren op de bladeren.
Bij het aggregeren op basis van de bladeren met de RMQ implementatie was de uitvoeringstijd maar liefst 12 uur, 19 minuten en 16 seconden!
Dit is dus een extreem groot verschil.

\section{Conclusie suffixbomen}\label{sec:conclusie-suffix-bomen}
Het is duidelijk dat suffixbomen erg performant zijn voor dit scenario.
Het opbouwen gebeurt snel en het zoeken voor een match gaat vliegensvlug.
\\ \\
Door de eigen implementatie in Rust kunnen we ook wat tijd besparen ten opzichte van een equivalente C++ implementatie.
Een deel van de winst zit tijdens het opbouwen van de boom, maar vooral tijdens het zoeken wanneer informatie uit de bladeren gehaald moet worden.
Vermoedelijk ligt de andere geheugenstructuur hiervoor aan de basis.
\\ \\
Ondanks de veelbelovende resultaten op vlak van snelheid is er een keerzijde aan de medaille.
Het geheugengebruik is zo groot dat we op zoek moeten naar een andere datastructuur.
Voor de Swiss-Prot databank gaat het geheugenverbruik al boven 80 GB, terwijl ons einddoel is om dit te gebruiken op de TrEMBL dataset.
Dit wilt zeggen dat als we alles \textit{slechts} $\pm$ 500 maal kunnen opschalen het doel bereikt is.
Dit zou echter vereisen dat we een server hebben met ongeveer 50 TB aan RAM geheugen.
Dit is niet mogelijk, we moeten daarom op zoek naar een andere datastructuur die minder geheugen vereist.


    \chapter{Conclusie \& future work}\label{ch:conclusie}
In deze masterproef hebben we verschillende opties verkend om de huidige Unipept index voor UniProtKB te vervangen.
Hierbij lag de hoofdfocus op het vinden van een nieuwe indexstructuur die aan volgende opties voldeed.
\begin{enumerate}
    \item De index moet het mogelijk maken om arbitraire peptiden te kunnen zoeken.
    \item Het geheugenverbruik van de index moet beperkt zijn, zodat het mogelijk is niet enkel kleinere proteïnedatabanken te indexeren, maar ook de volledige UniProtKB database\@.
    \item De indexstructuur moet semi-exacte matching ondersteunen, zodat I en L aan elkaar gelijkgesteld kunnen worden.
\end{enumerate}

\section{Conclusie}
Een eerste indexstructuur die we bekekenen waren \textbf{suffixbomen}.
Hierbij hebben we een eigen Rust implementatie gemaakt van het algoritme van Ukkonen.
Deze suffixbomen bieden een extreem grote vrijheid, en het zoeken gaat bovendien extreem snel.
Ze zijn echter \textbf{onbruikbaar} voor grote proteïnedatabanken te indexeren vanwege het \textbf{geheugengebruik}.
\\ \\
Een volgende optie die we verkend hebben zijn \textbf{suffix arrays}.
Na testen bleek dat deze datastructuur ons een goeie \textbf{balans gaf tussen snelheid en geheugenverbruik}.
Aan de hand hiervan zijn we er in geslaagd een nieuwe indexstructuur voor UniProtKB te ontwikkelen die de huidige Unipept index volledig kan vervangen, en alle bovenstaande doelstellingen haalt.
Aan de hand van een aangepast zoeksysteem ondersteunen we zowel exacte als semi-exacte matching, waarbij efficiënt alle informatie van de gematchte peptides teruggegeven kan worden.
Het grootste nadeel van deze manier van werken is dat de functionele en taxonomische analyses van Unipept \textit{on the fly} moeten gebeuren, wat een beperkte, negatieve impact heeft op de performantie.
\\ \\
Als derde en vierde indexstructuur hebben we kort de \textbf{FM- en R-index} getest.
Na enkele testen bleek snel dat het \textbf{geheugengebruik} tijdens het bouwen hiervan \textbf{dubbel zo hoog lag} als bij suffix arrays.
Omwille hiervan hebben we deze opties niet verder uitgewerkt, ook al hebben beide indices interessante eigenschappen voor ons probleem.
Zo laten bidirectionele FM-indices toe om algemenere inexacte matching uit te voeren, en is de resulterende index bij R-indices kleiner door het gebruik van run-length encoding.

\section{Future work}
Ondanks dat we een index gevonden hebben die de huidige Unipept index kan vervangen, en willekeurige peptiden kan vinden in plaats van enkel tryptische peptiden, zijn er meerdere plaatsen waar nog extra onderzoek naar kan gebeuren.
\\ \\
Zo zou het extreem interessant zijn om een manier te vinden om de suffix array onmiddellijk \textbf{sparse te maken tijdens het opbouwen}.
Hierbij zal het bovendien belangrijk zijn dat deze implementatie een \textbf{lage constante factor heeft op vlak van geheugengebruik}.
Indien deze constante vrij groot is, zal de winst verloren gaan ten opzichte van de huidige implementatie, aangezien we altijd een kleine sparseness factor $k$ gebruiken.
Een andere aanpak zou kunnen zijn via een online algoritme waarbij we een bestaande SSA kunnen uitbreiden met nieuwe eiwitten.
Zo kunnen we tijdens het opbouwen één eiwit per keer toevoegen aan de suffix array, maar kunnen we ook nieuwere UniProtKB releases efficiënt afhandelen door te vertrekken van de index uit de vorige release.
\\ \\
Een tweede duidelijk punt van verbetering, is tijdens het \textbf{berekenen van de LCA*}.
Indien deze voorberekend wordt tijdens het opbouwen van de index zal dit niet enkel de performantie ten goede komen, het zal ook de nood voor de drempelwaarde B (die standaard op 10\thinspace000 staat) sterk verminderen.
De meest duidelijke piste hiervoor is aan de hand van Enhanced Suffix Arrays (ESAs).
In deze masterproef hebben we deze piste niet verder verkend vanwege de eerste indicatie dat het berekenen van de extra tabellen het geheugenverbruik zou verdubbelen, en dat het \textit{on the fly} berekenen van de LCA* een acceptabele overhead met zich mee brengt.
\\ \\
Een derde mogelijke route, is het \textbf{uitbreiden van de inexacte matching}.
Op dit moment is dit slechts in een extreem beperkte vorm aanwezig.
We hebben namelijk enkel de optie om I en L aan elkaar gelijk te stellen.
Tools zoals de eerder vermelde Expasy ScanProsite tool ondersteunen verschillende manieren om inexacte matching uit te voeren.
Zo zouden we ondersteuning voor karakterklassen, een reeks van herhalingen (al dan niet met een minimum en maximum bereik) en wildcards kunnen toevoegen.
Een andere manier van inexacte matching kan zijn via het toelaten van maximaal $x$ mismatches.
Op deze manier kan men ook omgaan met kleine mutaties die ontstaan in eiwitten, of fouten die ontstaan tijdens het inlezen via een massaspectrometer.
Over deze laatste vorm van inexacte matching is door het team van Unipept een vervolg masterproef uitgeschreven om dit volgend jaar te verkennen.
Hierbij zal verder ingegaan worden op FM- en R-indices.

% =====================================================================
% End matter
% =====================================================================

% ------------ REFERENCES ------------
    \printbibliography[heading=bibintoc,title={Referenties}] % check if bibliography is in table of contents


% ------------ APPENDIX ------------
    \appendix
    \chapter{Statistieken Zoekbestanden}\label{ch:appendix-statistieken-zoekbestanden}
Deze appendix bevat extra grafieken die niet relevant waren voor de tekst, maar toch extra inzicht kunnen geven in de consistentie van de zoekbestanden.
Deze grafieken visualiseren de verdeling van aminozuren en de lengte van de peptides.

\section{Human-Prot}\label{sec:human-prot-stats}
\begin{figure}[H]
    \centering
    \subfloat[Distributie van de aminozuren in het Human-Prot zoekbestand.]{\includegraphics[width=0.485\linewidth]{humanprot_search_amino_acids}}
    \hfill
    \subfloat[Lengtedistributie van de peptiden in het Human-Prot zoekbestand.]{\includegraphics[width=0.485\linewidth]{humanprot_search_lengths}}
    \caption{Extra statistieken over het Human-Prot zoekbestand}\label{fig:humanprot_search_other_stats}
\end{figure}

\section{Swiss-Prot}\label{sec:swiss-prot-stats}
\begin{figure}[H]
    \centering
    \subfloat[Distributie van de aminozuren in het Swiss-Prot zoekbestand zonder \textit{missed cleavage}.]{\includegraphics[width=0.485\linewidth]{swissprot_searchfile_no_missed_cleavage_amino_acids}}
    \hfill
    \subfloat[Lengtedistributie van de peptiden in het Swiss-Prot zoekbestand zonder \textit{missed cleavage}.]{\includegraphics[width=0.485\linewidth]{swissprot_searchfile_no_missed_cleavage_lengths}}
    \caption{Extra statistieken over het Swiss-Prot zoekbestand zonder \textit{missed cleavage}}\label{fig:swissprot_search_no_missed_cleavage_other_stats}
\end{figure}
\begin{figure}[H]
    \centering
    \subfloat[Distributie van de aminozuren in het Swiss-Prot zoekbestand met \textit{missed cleavage}.]{\includegraphics[width=0.485\linewidth]{swissprot_searchfile_missed_cleavage_amino_acids}}
    \hfill
    \subfloat[Lengtedistributie van de peptiden in het Swiss-Prot zoekbestand met \textit{missed cleavage}.]{\includegraphics[width=0.485\linewidth]{swissprot_searchfile_missed_cleavage_lengths}}
    \caption{Extra statistieken over het Swiss-Prot zoekbestand met \textit{missed cleavage}}\label{fig:swissprot_search_missed_cleavage_other_stats}
\end{figure}

\section{SIHUMI}\label{sec:sihumi-stats}
\begin{figure}[H]
    \centering
    \subfloat[Distributie van de aminozuren in het SIHUMI S03 zoekbestand.]{\includegraphics[width=0.485\linewidth]{sihumi_03_amino_acids}}
    \hfill
    \subfloat[Lengtedistributie van de peptiden in het in het SIHUMI S03 zoekbestand.]{\includegraphics[width=0.485\linewidth]{sihumi_03_length}}
    \caption{Extra statistieken over het SIHUMI S03 zoekbestand}\label{fig:sihumi_03_other_stats}
\end{figure}
\begin{figure}[H]
    \centering
    \subfloat[Distributie van de aminozuren in het SIHUMI S05 zoekbestand.]{\includegraphics[width=0.485\linewidth]{sihumi_05_amino_acids}}
    \hfill
    \subfloat[Lengtedistributie van de peptiden in het in het SIHUMI S05 zoekbestand.]{\includegraphics[width=0.485\linewidth]{sihumi_05_length}}
    \caption{Extra statistieken over het SIHUMI S05 zoekbestand}\label{fig:sihumi_05_other_stats}
\end{figure}
\begin{figure}[H]
    \centering
    \subfloat[Distributie van de aminozuren in het SIHUMI S07 zoekbestand.]{\includegraphics[width=0.485\linewidth]{sihumi_07_amino_acids}}
    \hfill
    \subfloat[Lengtedistributie van de peptiden in het in het SIHUMI S07 zoekbestand.]{\includegraphics[width=0.485\linewidth]{sihumi_07_length}}
    \caption{Extra statistieken over het SIHUMI S07 zoekbestand}\label{fig:sihumi_07_other_stats}
\end{figure}
\begin{figure}[H]
    \centering
    \subfloat[Distributie van de aminozuren in het SIHUMI S08 zoekbestand.]{\includegraphics[width=0.485\linewidth]{sihumi_08_amino_acids}}
    \hfill
    \subfloat[Lengtedistributie van de peptiden in het in het SIHUMI S08 zoekbestand.]{\includegraphics[width=0.485\linewidth]{sihumi_08_length}}
    \caption{Extra statistieken over het SIHUMI S08 zoekbestand}\label{fig:sihumi_08_other_stats}
\end{figure}
\begin{figure}[H]
    \centering
    \subfloat[Distributie van de aminozuren in het SIHUMI S11 zoekbestand.]{\includegraphics[width=0.485\linewidth]{sihumi_11_amino_acids}}
    \hfill
    \subfloat[Lengtedistributie van de peptiden in het in het SIHUMI S11 zoekbestand.]{\includegraphics[width=0.485\linewidth]{sihumi_11_length}}
    \caption{Extra statistieken over het SIHUMI S11 zoekbestand}\label{fig:sihumi_11_other_stats}
\end{figure}
\begin{figure}[H]
    \centering
    \subfloat[Distributie van de aminozuren in het SIHUMI S14 zoekbestand.]{\includegraphics[width=0.485\linewidth]{sihumi_14_amino_acids}}
    \hfill
    \subfloat[Lengtedistributie van de peptiden in het in het SIHUMI S14 zoekbestand.]{\includegraphics[width=0.485\linewidth]{sihumi_14_length}}
    \caption{Extra statistieken over het SIHUMI S14 zoekbestand}\label{fig:sihumi_14_other_stats}
\end{figure}

\end{document}