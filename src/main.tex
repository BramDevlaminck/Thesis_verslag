\documentclass[11pt,dutch,faculty=we,layout=titlefont,underline=false,titleUppercase=true,titleUnderline=true]{ugent2016-report}
\usepackage[dutch]{babel}
\usepackage{fontspec,unicode-math}
\usepackage{amsmath}
\usepackage{subcaption}
\usepackage{csquotes}
\usepackage{subfloat}
\usepackage{float}
\usepackage[outputdir=../out]{minted}
\usepackage{enumitem}

\usepackage{booktabs,makecell}
% Figures
%---------
\usepackage{graphicx, ugent2016-assets}
\graphicspath{{./figures/}}
\usepackage[inkscapelatex=false]{svg}
\usepackage{footmisc}
\usepackage{siunitx}
\usepackage{setspace}


% Bibliography settings
%-----------------------
\usepackage{tikz}
\usepackage{biblatex}
\addbibresource{bibliography.bib}

\usepackage[hyperfootnotes=true]{hyperref}
\hypersetup{
    colorlinks,
    linkcolor={black},
    citecolor={blue!50!black},
    urlcolor={blue!80!black}
}
\definecolor{lightgrey}{HTML}{AAAAAA}
\setlength{\parindent}{0em}
\renewcommand{\listingscaption}{Codefragment} % Listing->Codeblock


\author{Bram Devlaminck}
\title{Snelle semi-exacte matching\newline van peptiden met een geheugenefficiënte index voor UniProtKB}
%\subtitle{Hello}
\academicyear{2023--2024}
\programme{Informatica}
\studentnumber{01902993}
\email{bram.devlaminck@ugent.be}

\titletext{%
    Promotoren: prof.\ dr.\ Peter Dawyndt, prof. dr.\ ir.\ Bart Mesuere \\%
    Begeleiders: dr. Pieter Verschaffelt, Tibo Vande Moortele
    \\ \\%
    {\small Masterproef ingediend tot het behalen van de academische graad van\\%
    Master of Science in de Informatica%
    }%
}


\begin{document}

    \maketitle


    \setmonofont[Scale=MatchLowercase,Contextuals={Alternate}]{Jetbrains Mono}


% ------------- CONSENT OF USE -----------
%    \onehalfspacing
    \chapter*{Toelating tot bruikleen}

De auteur geeft de toelating deze masterproef voor consultatie beschikbaar te stellen en delen van de masterproef te kopiëren voor persoonlijk gebruik.
Elk ander gebruik valt onder de bepalingen van het auteursrecht, in het bijzonder met betrekking tot de verplichting de bron uitdrukkelijk te vermelden bij het aanhalen van resultaten uit deze masterproef.
\\ \\
Bram Devlaminck \\ \today
% TODO: voeg handtekening toe? maak wel dat het bestand niet op github staat, en ook niet de versie met dat handtekening

% ------------ TABLE OF CONTENTS ---------
    \tableofcontents
    \newpage


% =====================================================================
% Main matter
% =====================================================================


    \chapter{Introductie}\label{ch:introductie}


\section{Inleiding}\label{sec:inleiding}

\subsection{Genomica, transcriptomica \& proteomica}\label{subsec:genomica-transcriptomica-&-proteomica}
Eiwitten zijn alomtegenwoordig en spelen een belangrijke rol in ons dagelijkse leven.
Ze garanderen een correcte werking van belangrijke processen binnen elk organisme.
Hieronder vallen de levensprocessen van ons eigen lichaam, maar ook die van dieren, planten, bacteriën en zelfs virussen.
Om deze processen te analyseren zijn er meerdere benaderingen mogelijk.
Een eerste mogelijkheid is aan de hand van het onderzoeksgebied van de proteomica.
Dit is de studie van alle eiwitten die binnen een enkel organisme tot expressie kunnen komen.
Hierbij probeert men te begrijpen hoe eiwitten in elkaar zitten, hoe deze met elkaar en binnen een bepaalde omgeving met elkaar interageren en wat hun belangrijkste functie is.
Naast proteomica bestaan er nog twee andere gerelateerde disciplines.
\\ \\
De eerste alternatieve discipline is genomica, het onderzoek naar het genoom.
Het genoom van een organisme is de collectie van al het DNA binnen een organisme.
Dit stelt voor welke proteïnen mogelijks door het organisme geconstrueerd kunnen worden.
Een belangrijk verschil met proteomica is dat DNA instructies voorstelt voor de productie van alle mogelijke proteïnen die het organisme kan maken.
Het geeft dus geen informatie over de proteïnen die op dat moment in de tijd actief zijn.
Het is een voorstelling van wat het organisme kan, niet wat het op \textit{dit} moment aan het doen is.
Belangrijk is dat ongeveer 98\% van het menselijke genoom niet-coderend is, wat wil zeggen dat dit deel van het DNA niet omgezet kan worden naar een betekenisvolle proteïne.
\\ \\
De andere discipline is de transcriptomica.
Deze discipline onderzoekt het transcriptoom van een organisme, wat de verzameling is van alle RNA moleculen die in het organisme aanwezig zijn.
Het transcriptoom is een belangrijke indicatie van welke delen uit het DNA effectief proteïnen encoderen.
Dit omdat RNA, meer specifiek messenger RNA (mRNA) en transfer RNA (tRNA), een belangrijk onderdeel is van het proces om DNA om te zetten naar proteïnen.
\\ \\
Onze focus ligt vooral binnen het veld van de metaproteomica.
het \textit{meta} prefix zegt dat de te analyseren stalen niet van één organisme zijn, maar van meerdere organismen (typisch binnen hetzelfde ecosysteem).
Dit maakt de analyse moeilijker aangezien proteïnen van verschillende organismen gelijkaardige aminozuursequenties kunnen hebben (al dan niet door toeval).
Meer specifiek is het doel van metaproteomica om op zoek te gaan naar de taxa die hoort bij een verzameling van peptiden.
Een een veelvoorkomende categorie van peptiden zijn tryptische peptiden.

\subsection{Tryptische peptiden}\label{subsec:tryptische-peptiden}
Tryptische peptiden zijn peptiden die ontstaan na het knippen van proteïnen aan de hand van trypsine.
Dit is een protease (eiwitafbrekend enzym) dat proteïnen opsplitst in meerdere peptiden.
Er bestaan nog andere proteases, maar trypsine is veruit de populairste door zijn eenduidig gedrag en efficiëntie.
\\ \\
Trypsine zal eiwitten knippen na elk voorkomen van lysine (K) of arginine (R) indien het eerstvolgende aminozuur geen proline (P) is.
Deze vuistregel is echter niet perfect.
Soms zal trypsine, een locatie waar normaal geknipt moet worden, missen.
Dit noemen we een \textit{missed cleavage}.
Figuur~\ref{fig:trypsine} bevat een voorbeeld van de werking.

\begin{figure}[H]
    \centering
    \includesvg[width=0.9\textwidth]{trypsine_verwerking}
    \caption{Voorbeeld van de werking van trypsine op 2 proteïnen. De aminozuren in het rood zijn lysine (K) of arginine (R), waarna trypsine knipt (behalve als het eerstvolgende aminzoruur proline (P) is). De tweede proteïne bevat een voorbeeld waar niet geknipt wordt na lysine, doordat het opeenvolgende aminozuur proline is~\cite{phdPieterUnipept}.}
    \label{fig:trypsine}
\end{figure}

Om de peptiden uit een experiment te kunnen gebruiken bij computeranalyses moeten deze omgezet worden naar een stringrepresentatie.
Dit is is echter een moeilijk en ingewikkeld proces.
Eerst wordt de massa/ladingsverhouding (m/z) van peptiden aan de hand van een massa spectrometer gemeten.
Daarna worden deze resultaten aan de hand van diverse zoekprocessen (via zogenaamde zoekmachines) omgezet naar de stringvoorstelling van de peptide.
Deze sequenties vormen de input voor tools zoals Unipept.

\subsection{Unipept}\label{subsec:unipept-introductie}
Unipept biedt allerhande tools aan om stalen uit het onderzoeksveld van de metaproteomica te analyseren, maar er is ook een onderdeel (UMGAP~\cite{UMGAP_paper}) gericht op het analyseren van stalen uit de metagenomica.

\begin{itemize}
    \item \textbf{Unipept Web application} Dit is de originele Unipept tool en is publiek beschikbaar op \url{https://unipept.ugent.be}.
    Met de gebruiksvriendelijke \textit{user interface} wordt analyseren van metaproteomica data beschikbaar gesteld.
    De resultaten van deze analyse worden aan de hand van visualisaties en tabellen aan de gebruiker voorgesteld.
    Deze kunnen vervolgens makkelijk geëxporteerd worden (bijv. voor analyse in andere tools).
    \item \textbf{Unipept CLI} Dit is een \textit{power-user} tool om analyses uit te voeren op grotere stalen.
    \item \textbf{Unipept API} Dit is een collectie van \textit{endpoints} die andere applicaties toelaat om de functionaliteit van Unipept integreren.
    \item \textbf{Unipept Desktop} Dit is de recentste toevoeging aan het Unipept ecosysteem en laat toe dat onderzoekers niet noodzakelijk met de Unipept servers moeten communiceren om analyses uit te voeren.
    Deze applicatie combineert de voordelen van de web app, CLI en API en laat toe om lokaal stalen te analyseren, gebruik makende van een gebruiksvriendelijke UI\@.

\end{itemize}

Op dit moment is Unipept gericht op de analyse van tryptische peptiden.
De reden hiervoor is de manier waarop de achterliggende indexstructuur opgebouwd wordt.
Dit opbouwen gaat in grote lijnen als volgt:

\begin{enumerate}
    \item Haal alle proteïnen en bijbehorende taxonomische en functionele annotaties op uit de UniProtKB databank.
    \item Splits deze proteïnen volgens de vuistregel die trypsine volgt.
    \item Sla alle resulterende tryptische peptiden op in een indexstructuur.
\end{enumerate}

Deze aanpak heeft als voordeel dat we op een efficiënte manier tryptische peptiden kunnen opzoeken (samen met de bijbehorende annotaties).
Er is echter een belangrijke keerzijde aan deze manier van werken.
Het zoeken van niet-tryptische peptiden (hieronder vallen ook peptiden met \textit{missed cleavage}) is inefficiënt.
Dit komt doordat tijdens het opbouwen van de Unipept indexstructuur de vuistregel gevolgd wordt, en elke peptide in de indexstructuur strikt tryptisch is.
\\ \\
Op dit moment is er wel een \textit{workaround} die toe laat deze peptiden toch te zoeken.
Dit heeft echter wel een significante impact op de performantie.
Dit verklaart dan ook de nood aan een nieuwe indexstructuur die toe laat volledig willekeurig gesplitste peptiden te zoeken.
\newline
Voor een gedetailleerdere beschrijving over Unipept en het onderzoeksveld van metaproteomica is het aangeraden om de inleiding van het doctoraat van Dr. Pieter Verschaffelt te lezen~\cite{phdPieterUnipept}.
Dit vormde een duidelijke en goede basis voor deze inleiding.


\section{Probleemstelling}\label{sec:probleemstelling}
Het probleem waarvoor een oplossing gezocht wordt is het snel terugvinden van willekeurige peptiden in een eiwitdatabank.
Bij het vinden van een match moet het daarna mogelijk zijn de informatie op te halen die hoort bij alle proteïnen die matchen.
Binnen het onderzoeksgebied van de informatica kunnen we dit probleem als volgt herformuleren:
``In een grote verzameling van middellange strings (alle eiwitten in onze databank), moeten we voor een verzameling van korte strings (peptiden) terugvinden in welke van deze middellange strings ze voorkomen.''
\\ \\
Belangrijk hierbij is dat dit niet alleen snel gebeurt, maar we ook proberen het vereiste geheugen tot een minimum te beperken.
Dit omdat de datasets waarmee gewerkt wordt extreem groot zijn.
\\ \\
Tot slot willen we ook inexacte matching toevoegen tijdens het zoeken.
Door dit te doen kunnen we beter omgaan met kleine fouten die voorkomen tijdens het uitlezen van de experimentele stalen.
Om dit allemaal te bereiken is het doel van deze thesis om meerdere datastructuren uit te werken, te implementeren in Rust, en tot slot te testen.
Het gebruik van Rust laat ons toe om extreem hoge performantie te verkrijgen (vergelijkbaar met C en C++~\cite{rustPerformantie}) in combinatie met \textit{memory safety}\footnote{\textit{Memory safety} is een eigenschap die verzekert dat programma's enkel gebruik kunnen maken van geldige geheugenlocaties en geen \textit{undefined behaviour} zoals \textit{buffer overflows}, \textit{dangling pointers} en andere geheugen gerelateeerde fouten kunnen vertonen.}.
Bovendien zijn sommige delen van Unipept al geschreven in Rust (zie UMGAP~\cite{UMGAP_paper, UMGAP_source}).
Dit laat toe om waar mogelijk bestaande code te hergebruiken.


\section{Benchmark Datasets}\label{sec:datasets}

\subsection{Proteïnedatabanken}\label{subsec:proteine-databanken}
Om te testen hoe goed een implementatie is en hoe deze zich verhoudt ten opzichte van bestaande implementaties is het belangrijk om representatieve datasets te gebruiken.
Deze datasets zijn allemaal eiwitdatabanken die een subset vormen van UniProtKB (meer specifiek UniProtKB 2023\_04)~\cite{UniprotKB}.
UniProt bestaat uit twee onderdelen (gegeven statistieken zijn voor release 2023\_04).
\begin{enumerate}
    \item Swiss-Prot: Dit is een kleinere, manueel gecureerde, dataset met 570 157 eiwitsequenties.
    \item TrEMBL: Deze dataset bevat 251 600 768 sequenties en is dus veel groter dan Swiss-Prot.
    Het belangrijkste verschil is dat deze dataset \textbf{niet} manueel gecureerd is.
\end{enumerate}
Om het vergelijken van de performantie van verschillende implementaties praktisch te houden maak ik tijdens het testen gebruik van 2 subsets hiervan.

\paragraph{Swiss-Prot} Deze databank is één van de twee standaard onderdelen van UniProt.
Een kort overzicht van alle statistieken is terug te vinden in tabel~\ref{tab:swissprot_eigenschappen}.
Figuur~\ref{fig:swissprot_aminozuur} en figuur~\ref{fig:swissprot_length} geven meer inzicht in de distributie van de aminozuren en lengte van de proteïnen.

\begin{table}[h!]
    \centering
    \begin{tabular}{l l}
        Metriek                   & Waarde      \\
        \hline\hline
        Totaal aantal sequenties  & 569 619     \\
        Totale lengte             & 205 954 074 \\
        Minimale proteïnelengte   & 2           \\
        Maximale proteïnelengte   & 35 213      \\
        Gemiddelde proteïnelengte & 361.56      \\
        Mediaan proteïnelengte    & 295         \\
        \hline
    \end{tabular}
    \caption{Eigenschappen van de Swiss-Prot databank uit UniProt 2023\_04 die gebruikt wordt om de indexstructuur op te bouwen.}
    \label{tab:swissprot_eigenschappen}
\end{table}


\begin{figure}[H]
    \centering
    \includegraphics[width=0.7\linewidth]{swissprot_aminozuur_voorkomens}
    \caption{Aantal voorkomens per aminozuur voor alle proteïnen in de Swiss-Prot databank uit UniProt 2023\_04.}
    \label{fig:swissprot_aminozuur}
\end{figure}

\begin{figure}[H]
    \centering
    \subfloat[Overzicht van de lengtedistributie van alle proteïnen]{\includegraphics[width=0.485\linewidth]{swissprot_length_distribution_large}}
    \hfill
    \subfloat[Ingezoomd beeld van de lengtedistributie tot 1000 aminozuren lang]{\includegraphics[width=0.485\linewidth]{swissprot_length_distribution_small}}
    \caption{Lengtedistributie van de proteïnen in de Swiss-Prot databank.}\label{fig:swissprot_length}
\end{figure}

Doordat het gebruikte invoerbestand reeds verwerkt werd door een deel van de Unipept pipeline, is er een klein verschil tussen het totaal aantal sequenties in tabel~\ref{tab:swissprot_eigenschappen} en wat eerder aangegeven werd.
Hierbij worden onder andere sequenties met een onbekend taxon id verwijderd, wat het kleine verschil verklaart.

\paragraph{Human-Prot} Deze dataset is samengesteld aan de hand van drie referentiedatabanken afkomstig van UniProtKB\@.
Dit zijn de Human Genome~\cite{proteomes_homo_sapiens}, Influenza B~\cite{proteomes_infuenza_b} en Human Papillomavirus~\cite{proteomes_human_papillomavirus} databank.
Opnieuw komen deze allemaal uit UniProt 2023\_04.

Deze Human-Prot databank is kleiner dan Swiss-Prot waardoor het testen tijdens ontwikkeling sneller is.
Tabel~\ref{tab:humanprot_eigenschappen} somt enkele belangrijke metrieken op over deze dataset.
Figuur~\ref{fig:humanprot_aminozuur} en~\ref{fig:humanprot_length} gaan dieper in op een aantal details.

\begin{table}[h!]
    \centering
    \begin{tabular}{ l l }
        Metriek                   & Waarde     \\
        \hline\hline
        Totaal aantal sequenties  & 82 695     \\
        Totale lengte             & 30 293 046 \\
        Minimale proteïnelengte   & 2          \\
        Maximale proteïnelengte   & 35 991     \\
        Gemiddelde proteïnelengte & 366.32     \\
        Mediaan proteïnelengte    & 204        \\
        \hline
    \end{tabular}
    \caption{Eigenschappen van de Human-Prot databank die gebruikt om de indexstructuur op te bouwen.}
    \label{tab:humanprot_eigenschappen}
\end{table}

\begin{figure}[H]
    \centering
    \includegraphics[width=0.7\linewidth]{humanprot_aminozuur_voorkomens}
    \caption{Aantal voorkomens per aminozuur voor alle proteïnen in de Human-Prot databank.}
    \label{fig:humanprot_aminozuur}
\end{figure}

\begin{figure}[H]
    \centering
    \subfloat[Overzicht van de lengtedistributie van alle proteïnen]{\includegraphics[width=0.485\linewidth]{humanprot_length_distribution_large}}
    \hfill
    \subfloat[Ingezoomd beeld van de lengtedistributie tot 1000 aminozuren lang]{\includegraphics[width=0.485\linewidth]{humanprot_length_distribution_small}}
    \caption{Lengtedistributie van de proteïnen in de Human-Prot databank.}\label{fig:humanprot_length}
\end{figure}

We kunnen concluderen dat zo goed als alle letters gebruikt worden (ook al zijn er maar 20 aminozuren).
Dit komt doordat sommige letters eigenlijk een soort wildcard voorstellen.
Zo staat ``X'' voor elk mogelijk aminozuur, ``Z'' voor ``Q'' of ``E'',\ldots

Ook zien we dat de verdeling van de proteïnelengtes in de Swiss-Prot en Human-Prot datasets vergelijkbaar zijn.
Het zwaartepunt ligt bij Swiss-Prot wel iets later (rond 100 vs rond 200).

\subsection{Peptide zoekbestanden}\label{subsec:peptide-zoek-bestanden}
De zoekperformantie van onze indexstructuur is een erg belangrijk aspect.
Om dit te meten hebben we bij elke eiwitdatabank een lijst aan peptiden die we proberen te zoeken.
Voor beide databanken zijn enkele datasets opgesteld.

\subsubsection{Swiss-Prot}
Voor deze databank hebben we enkele zoekbestanden voorzien.
Twee bestanden die gesampled zijn en een reeks aan real-life stalen.
De twee artificiële bestanden zijn zo gekozen dat de ene enkel tryptische peptiden bevat, terwijl de andere ook peptiden bevat met \textit{missed cleavage}.
De eerste kan dus op dit moment al efficiënt door Unipept behandeld worden, terwijl dit voor de tweede niet mogelijk is.

\paragraph{Artificiële stalen}
Tabel~\ref{tab:artifiele_bestanden_statistieken} bevat in kolom twee en drie een kort overzicht met statistieken voor deze gesamplede bestanden.

\begin{table}[H]
    \hspace*{-0.35cm} % this table is just a bit too wide to, just manually move it a bit to the left to "center" it
    \begin{tabular}{l l l l}
        Metriek                    & SP zonder \textit{missed cleavages} & SP met \textit{missed cleavages} & Human-Prot    \\
        \hline\hline
        Totaal aantal sequenties   & 100 000                             & 100 000                          & 250 000       \\
        Totale lengte              & 1 605 909                           & 2 544 356                        & 2 458 834 046 \\
        Minimale proteïnelengte    & 5                                   & 5                                & 1             \\
        Maximale proteïnelengte    & 50                                  & 93                               & 12            \\
        Gemiddelde proteïnelengte  & 16.06                               & 25.44                            & 9.84          \\
        Mediaan proteïnelengte     & 13                                  & 23                               & 10            \\
        Aantal vindbare peptiden   & 67 375                              & 62 581                           & 250 000       \\
        Aantal tryptische peptiden & 100 000                             & 4107                             & 102 659       \\
        \hline
    \end{tabular}
    \caption{Eigenschappen van de verschillende zoekbestanden. De kolommen die beginnen met SP bevatten de statistieken van de Swiss-Prot zoekbestanden. De laatste kolom bevat de statistieken voor het zoekbestand dat hoort bij de Human-Prot eiwitdatabank.}
    \label{tab:artifiele_bestanden_statistieken}
\end{table}

\paragraph{Experimentele stalen}
Om de performantie beter te beoordelen, gebruiken we ook enkele stalen uit experimenten met een kleine micro-organisme gemeenschap, namelijk SIHUMIx\footnote{Simplified human intestinal microbiota}~\cite{SIHUMI_first_introduction, SIHUMI_frequently_used}.
Aangezien dit effectieve stalen zijn, bevatten deze \textit{missed cleavages} die natuurlijk ontstaan zijn.
Tabel~\ref{tab:sihumi_zoekbestanden} bevat de belangrijkste statistieken voor elk zoekbestand.
Elke kolom stelt een staal voor met als bestandsnaam \texttt{S<XX>.txt}.
Deze zoekbestanden worden in combinatie met de Swiss-Prot databank gebruikt tijdens het testen.

\begin{table}[H]
    \centering
    \begin{tabular}{ l l l l l l l }
        Metriek                    & S03     & S05     & S07     & S08     & S11     & S14     \\
        \hline\hline
        Totaal aantal sequenties   & 25 000  & 25 000  & 24 424  & 25 000  & 25 000  & 25 000  \\
        Totale lengte              & 420 544 & 420 423 & 373 633 & 316 114 & 366 922 & 430 674 \\
        Minimale proteïnelengte    & 6       & 6       & 6       & 6       & 6       & 6       \\
        Maximale proteïnelengte    & 50      & 50      & 47      & 43      & 50      & 50      \\
        Gemiddelde proteïnelengte  & 16.82   & 16.82   & 15.30   & 12.64   & 14.68   & 17.23   \\
        Mediaan proteïnelengte     & 15      & 16      & 14      & 12      & 14      & 16      \\
        Aantal vindbare peptiden   & 2570    & 2698    & 3652    & 4135    & 3792    & 2761    \\
        Aantal tryptische peptiden & 17 263  & 162     & 152     & 207     & 155     & 242     \\
        \hline
    \end{tabular}
    \caption{Eigenschappen van de SIHUMIx zoekbestanden. Elke kolom stelt een staal voor met als bestandsnaam \texttt{S<XX>.txt}.}
    \label{tab:sihumi_zoekbestanden}
\end{table}

\subsubsection{Human-Prot}
Voor deze databank hebben we één zoekbestand bestaande uit HLA-peptiden.
Dit zijn korte, niet-tryptische peptiden uit het immunopeptidomics onderzoeksveld.
Hierdoor kan Unipept op dit moment niet gebruikt worden om dit soort stalen te analyseren.
Elke peptide in dit zoekbestand is een sample van een proteïne uit de Human-Prot databank.
Hierdoor zijn alle peptiden die we zoeken effectief vindbaar in de dataset.
Een kort overzicht van enkele eigenschappen is terug te vinden in de laatste kolom van tabel~\ref{tab:artifiele_bestanden_statistieken}.
\\ \\
Appendix~\ref{ch:appendix-statistieken-zoekbestanden} bevat aanvullende grafieken over bovenstaande zoekbestanden.
Deze tonen voor elk zoekbestand de distributie van de aminozuren en de distributie van de peptidelengte.


\section{Benchmark hardware}\label{sec:benchmark-hardware}
Alle benchmarks zijn uitgevoerd op een virtuele machine van Unipept.
Tabel~\ref{tab:Matt_hardware} bevat een overzicht van de hardware van de fysieke machine en hoeveel toegekend is aan de VM\@.
Al deze informatie, en ook over de andere machines gebruik door Unipept, kan terug gevonden worden op de Unipept GitHub wiki~\cite{unipept_infrastructure}.

\begin{table}[h!]
    \centering
    \begin{tabular}{p{0.20\linewidth}p{0.45\linewidth}p{0.25\linewidth}}
        Onderdeel         & Fysieke server                                                            & Virtuele Machine    \\
        \hline\hline
        CPU               & 2x Intel Xeon 4410Y (12 cores / 24 threads, 2 - 3.9 GhZ, 30 MiB cache)    & 12 threads          \\
        RAM               & 768 GiB                                                                   & 128 GiB             \\
        Opslag            & 6x 16 TiB HDD (3.5 inch, 7.2K RPM SATA), 4x 3.84 TiB SSD (2.5 inch, SATA) & 1 TiB SSD, 4TiB SSD \\
        Besturingssysteem & Debian 12 (met Proxmox)                                                   & Ubuntu 22.04 LTS    \\
        \hline
    \end{tabular}
    \caption{Hardwarespecificaties van de fysieke server en virtuele machine die gebruikt worden tijdens het testen. Deze virtuele machine draait samen met nog enkele andere VMs op de server.}
    \label{tab:Matt_hardware}
\end{table}


    \chapter{Suffixbomen}\label{ch:suffix-bomen}
Een eerste datastructuur die het mogelijk maakt om snel kleine strings in een groot aantal andere strings op te zoeken zijn suffixbomen.
Meer precies eigenlijk een gegeneraliseerde suffixboom.
\\ \\
We behandelen deze datastructuur als eerste omdat hij vrij intuïtief is en de makkelijkste van alle mogelijke opties.
Bovendien kan een goede tijdscomplexiteit bereikt worden aangezien het zoeken in een suffixboom in $O(n)$ tijd kan (met $n$ de lengte van de zoekstring, in ons geval is dit een peptide).
Het opbouwen van de suffixboom kan ook in lineaire tijd gebeuren, al zei het dan lineair in de totale lengte van alle proteïnen in de databank (die we aanduiden met $m$). % TODO: vraag wat moest komen na: al zei het dan lineair in de totoale lengte van de ?

\section{Wat zijn suffixbomen?}\label{sec:wat-zijn-suffix-bomen?}
suffixbomen zijn een soort veralgemening van tries (prefix bomen).
Door er voor te zorgen dat het laatste teken uniek is zal elke suffix van de inputstring uniek zijn (elke suffix is dus nooit de prefix van een andere suffix).
Dit zorgt er voor dat elke suffix een eigen blad in de boom zal krijgen.
Dit is dan ook van waar de naam suffixboom komt.
Elk pad tot een blad in de boom zal exact 1 suffix voorstellen uit de inputstring waarvoor de boom gebouwd is.
Als voorbeeld stelt figuur~\ref{fig:suffix_tree_example} de suffixboom voor van de string \texttt{acacgt\$}.
Merk op dat we \texttt{`\$`} als uniek eindteken gebruiken.

\begin{figure}[H]
    \center
    \begin{tikzpicture}
    [
        level 1/.style = {sibling distance = 3.5cm, level distance = 2cm},
        level 2/.style = {sibling distance = 1.5cm, level distance = 2cm}
    ]

        \node[draw, circle] {}
        child {
            node[draw, rounded corners] {\texttt{t\$}}
            edge from parent node [above] {\texttt{t\$}}
        }
        child {
            node[draw, rounded corners] {\texttt{gt\$}}
            edge from parent node [below] {\texttt{gt\$}}
        }
        child {
            node[draw, circle] {}
            child {
                node[draw, rounded corners] {\texttt{cgt\$}}
                edge from parent node [left] {\texttt{gt\$}}
            }
            child {
                node[draw, rounded corners] {\texttt{cacgt\$}}
                edge from parent node [right] {\texttt{acgt\$}}
            }
            edge from parent node [right] {\texttt{c}}
        }
        child {
            node[draw, circle] {}
            child {
                node[draw, rounded corners] {\texttt{acacgt\$}}
                edge from parent node [left] {\texttt{gt\$}}
            }
            child {
                node[draw, rounded corners] {\texttt{acgt\$}}
                edge from parent node [right] {\texttt{acgt\$}}
            }
            edge from parent node [below] {\texttt{ac}}
        }
        child {
            node[draw, rounded corners] {\texttt{\$}}
            edge from parent node [above] {\texttt{\$}}
        }
        ;
    \end{tikzpicture}
    \caption{suffixboom voor the string \texttt{acacgt\$}}\label{fig:suffix_tree_example}

\end{figure}

Natuurlijk is dit niet efficiënt om effectief op deze manier op te slaan.
Als de tekst lengte $n$ heeft, heeft de suffixboom voor de tekst ten hoogste $2n - 1$ toppen en $2n - 2$ bogen.
Het aantal toppen en bogen is dus $\Theta(n)$.
Jammer genoeg vraagt alleen het opslaan van alle prefixen in de bladeren $\Theta(n^2)$ geheugen~\cite{AD3_ukkonen}.
In de plaats kunnen we pointers bijhouden naar het begin en einde van een substring in de originele string.
Dit zorgt er voor dat we geen kopie meer moeten opslaan van de originele string in elk blad!
We moeten dit zelfs niet in elk blad bijhouden!
We kunnen simpelweg bij elke boog tussen de toppen de labels bijhouden.
Het label van het blad kunnen we daarna reconstrueren door de labels van de bogen op weg naar dit blad te achter elkaar te plaatsen.
Door dit te doen is de nodige opslag per top een constante, en is het geheugengebruik lineair.
Figuur~\ref{fig:suffix_tree_example_indices} toont hoe dit er in de praktijk uit ziet.
Merk op dat de eind-index exclusief is.
Een boog met waarde \texttt{1,3}, stelt dus de substring \texttt{ca} voor uit het voorbeeld.

\begin{figure}[H]
    \center
    \begin{tikzpicture}
    [
        level 1/.style = {sibling distance = 2.5cm},
        level 2/.style = {sibling distance = 1cm}
    ]

        \node[draw, circle] {}
        child {
            [fill] circle (2pt)
            edge from parent node [above] {5,7}
        }
        child {
            [fill] circle (2pt)
            edge from parent node [below] {4,7}
        }
        child {
            node[draw, circle] {}
            child {
                [fill] circle (2pt)
                edge from parent node [left] {4,7}
            }
            child {
                [fill] circle (2pt)
                edge from parent node [right] {2,7}
            }
            edge from parent node [right] {1,2}
        }
        child {
            node[draw, circle] {}
            child {
                [fill] circle (2pt)
                edge from parent node [left] {4,7}
            }
            child {
                [fill] circle (2pt)
                edge from parent node [right] {2,7}
            }
            edge from parent node [below] {0,2}
        }
        child {
            [fill] circle (2pt)
            edge from parent node [above] {6,7}
        }
        ;
    \end{tikzpicture}
    \caption{suffixboom voor de string \texttt{acacgt\$} gebruik makende van indices}\label{fig:suffix_tree_example_indices}

\end{figure}


\section{Het Algoritme van Ukkonen}\label{sec:Ukkonen}
Het algoritme van Ukkonen om suffixbomen op te bouwen~\cite{Ukkonen1995} is niet eenvoudig.
De pseudocode en de theoretische beschrijving in de originele paper is beide vrij complex.
Het algoritme komt echter uitgebreid aan bod in een heleboel andere publicaties en boeken~\cite{Gusfield1997, AD3_ukkonen, CCB_course, Ukkonen_CCB}.
%\begin{enumerate}
%    \item Het boek \textit{Algorithms on Strings, Trees and Sequences}~\cite{Gusfield1997}.
%    \item De cursus \textit{Algoritmen \& Datastucturen 3} aan UGent gegeven door prof. Gunnar Brinkmann~\cite{AD3_ukkonen}.
%    \item De cursus Computational Challenges in Bioinformatics gegeven door prof. dr. Jan Fostier and prof. dr. Peter Dawyndt.
%    Naast een cursus met wat info over het algoritme van Ukkonen is er ook een implementatie van dit algoritme gemaakt door Jan Fostier in C++~\cite{Ukkonen_CCB}.
%\end{enumerate}

\subsection{Kotlin}\label{subsec:kotlin}
Ik ben begonnen met een implementatie van Ukkonen's algoritme in Kotlin zodat ik me niet op taal-specifieke problemen zou moeten focussen (vooral restricties rond \textit{borrowing} in Rust).
Hier was gelukkig de referentiecode van prof. Jan Fostier~\cite{Ukkonen_CCB} een grote hulp omdat dit het mogelijk maakte om tijdens het debuggen te zien wat de toestand van het programma is na $x$ stappen.
\\ \\
Één van de verschillen tussen mijn implementatie en de referentie-implementatie is de representatie van de kinderen.
In mijn implementatie is dit aan de hand van een HashMap in plaats van een array van pointers, simpelweg uit gemak zodat ik rechtstreeks een karakter als sleutel kon gebruiken, en die niet moest omzetten naar een index.
Om dit prototype te maken heb ik gekozen voor Kotlin boven Python aangezien Kotlin performanter is en ook een aangename ontwikkelingservaring biedt.
Hierdoor is het mogelijk om toch de test datasets op te bouwen in een redelijke tijd.
\\ \\
Uiteindelijk bleek de grootste struikelblok in het implementeren van Ukkonen enkele off-by-1 fouten.
Aangezien je tijdens het algoritme eigenlijk werkt met substrings, maar deze opgeslagen worden aan de hand van hun begin- en eind-index wordt het debuggen veel omslachtiger.
Tot slot had ik op het einde ook enkele bugs die niet voorkwamen in kleinere voorbeelden die met de hand uit te werken waren, wat daardoor relatief wat tijd vroeg om op te lossen.

\subsection{Rust}\label{subsec:rust}

\subsubsection{Eerste ervaring}
Aangezien dit mijn eerste ervaring was met Rust deel ik graag even mijn eerste bevindingen mee.
Zelf heb ik eerst \textit{the Rust book} gelezen~\cite{the_rust_book}.
Wat op zich veel goede informatie heeft, maar naar mijn ervaring soms te veel.
Een veelvoorkomend patroon in het boek is dat een deel van een concept geïntroduceerd wordt en dat je dan als lezer geïnformeerd wordt dat je er nu nog niet over moet denken, dat er meer informatie hierover komt in een later hoofdstuk.
Dit zorgt ervoor dat het soms moeilijk is de bomen door het bos te zien.
Zeker aangezien Rust vaak verschillende syntaxen heeft om hetzelfde te doen, waardoor je als lezer moeilijk je kunt focussen op de essentiële delen.
\\ \\
Om de meest essentiële basis componenten van Rust wat onder de knie te krijgen heb ik daarna de oefeningen van Rustlings~\cite{rustlings} gemaakt op aanraden van mede thesis student Stijn De Clercq.
Dit zijn een reeks aan erg kleine oefeningen die meestal in maximaal enkele minuten gemaakt zijn, maar er voor zorgen dat je toch al eens in contact komt met alle basisonderdelen van Rust.
Naar mijn mening is dit dus zeker een waardevolle toevoeging tijdens het leren van Rust.

\subsubsection{Boomstructuren}
\begin{quote}
    \textit{Rust is known to be notorious difficult when it comes to certain data structures like linked lists, trees, etc. \cite{rust_difficulty_quote}}
\end{quote}
Deze quote komt rechtstreeks uit een Medium artikel en toont direct aan dat het maken van een suffixboom in Rust niet-triviaal ging zijn.
De oorzaak hiervoor ligt bij het \textit{ownership} systeem van Rust.
Dit systeem zorgt er voor dat slechts één variabele eigenaar kan zijn van een stukje data.
In dit geval kan dus slechts één top een andere top opslaan, of er een \textit{mutable reference} naar hebben.
Meer praktisch wil dit dus zeggen dat slechts één top een \textit{pointer} kan hebben naar een andere top, met de toelating om die top aan te passen (wat nodig is tijdens het opbouwen van de boom. Er worden namelijk nog kindere toegevoegd en toppen gesplitst)
Dit is een groot probleem aangezien ouders pointers naar kinderen moeten hebben, de kinderen een pointer naar hun ouder, en er dan ook nog eens pointers zijn voor de suffix links.
\\ \\
Als oplossing hiervoor introduceert Rust het \texttt{Rc<T>} datatype.
Hierbij gaat Rust afstappen van zijn standaard \textit{ownership} systeem en gebruik maken van Reference Counting.
Pas wanneer alle referenties weg zijn zal het geheugen automatisch vrij gegeven worden.
De beperking hierbij is echter dat deze referenties \textit{immutable} zijn, dit volstaat niet tijdens het opbouwen van de boom.
\\ \\
Als oplossing hiervoor heeft Rust dan weer het \textit{interior mutability} patroon~\cite{interior_mutability} aan de hand van het datatype \texttt{Refcell}.
Dit laat toe om data toch aan te passen, ook al is een reference immutable.
Aangezien dit de standaard Rust regels doorbreekt, is dit \texttt{unsafe}\footnote{Dit is code waarvan de compiler niet kan nagaan als die aan alle voorwaarden voldoet die nodig zijn om \textit{memory safety} te kunnen garanderen. Dit sleutelwoord bestaat zodat de programmeur meer vrijheid zou kunnen krijgen om bepaalde patronen toch toe te kunnen passen. De verantwoordelijkheid om correct het geheugen te gebruiken wordt hier bij de programmeur gelegd. Een andere reden om \texttt{unsafe} te gebruiken is om bepaalde interacties met hardware uit te voeren. Deze zijn inherent onveilig en zouden anders onmogelijk zijn.} en kan Rust \textit{at compile-time} geen \textit{memory safety} meer garanderen.
\texttt{Refcell} zal gelukkig wel de nodige code invoegen zodat runtime memory safety wel gegarandeerd kan worden.
Mogelijke foutieve geheugen operaties zullen dus tijdens het uitvoeren van het programma gedetecteerd worden, \textbf{ten koste van performantie}.
\\ \\
Maar zelfs nu blijft er nog altijd een probleem.
Geheugen dat beheerd wordt aan de hand van \textit{reference counting} zal enkel vrijgegeven kunnen worden indien de \textit{reference counter} op 0 staat.
Er zijn echter scenario's waar dit nooit zal gebeuren.
Namelijk bij cyclische verwijzingen, een patroon dat jammer genoeg erg vaak voor komt (in ons geval bv een ouder die een pointer heeft naar een kind, en een kind een pointer naar de ouder).
Als oplossing hiervoor introduceert Rust dan weer het \texttt{Weak<T>} datatype.
\\ \\
Dit is duidelijk erg ingewikkeld, en introduceert ook nog eens performance overhead die niet nodig lijkt.
Een optie zou natuurlijk zijn door expliciet het \texttt{unsafe} keyword te gebruiken wat toe laat de ownership regels van Rust volledig uit te schakelen (zowel compile-time als run-time).
Het nadeel hiervan is natuurlijk dat we dan de garanties van memory safety kwijt zijn, wat net één van de hoofdredenen is om Rust te gebruiken.
Dit was dus geen mogelijke optie.
Gelukkig is er een alternatieve manier waar ik op gestoten ben, een arena-based implementatie~\cite{rust_arena_trees}.
Het idee hierbij is dat er één arena gemaakt wordt waarbij ownership erg simpel is.
In mijn implementatie is dit bijvoorbeeld een \texttt{Vector}.
Alle toppen worden hierbij in deze ene vector opgeslagen.
In plaats van pointers naar elkaar houden bij te houden zullen de toppen indexen bijhouden.
Deze index stelt de index in de arena van de top voor waarnaar anders een pointer wordt bijgehouden.
\\ \\
Na het maken van deze ontwerpaanpassingen bleef slechts één moeilijkheid over.
Uitzoeken hoe de cursor (die bij houdt waar we zijn in de boom tijdens het bouwen), de input string en de boom zelf zich van elkaar moeten verhouden in het ownership systeem.
Uiteindelijk viel dit vrij makkelijk uit te zoeken.
Het omzetten van de resterende Kotlin code naar Rust was erg simpel en bijna een één op één vertaling.
Het enige verschil is dat ik in de Rust implementatie gebruik heb gemaakt van een array om de kinderen bij te houden in plaats van een HashMap.

\subsubsection{Geheugen efficiëntie}
\begin{quote}
    \textit{And then I went and invented a null pointer.
    And if you use a null pointer you either have to check every reference or you risk disaster. \cite{null_mistake}}
\end{quote}
\textit{Null pointers} worden ook wel \textit{the billion-dollar mistake} genoemd vanwege het grote aantal bugs dat ze veroorzaken.
Daarom voorziet Rust een andere manier om de waarde \textit{null} voor te stellen.
Dit wordt gedaan aan de hand van de \texttt{Option<T>} enum.

\begin{minted}{Rust}
enum Option<T> {
    None,
    Some(T),
}
\end{minted}

Deze enum heeft 2 mogelijke waarden: \texttt{None} of \texttt{Some(T)}.
\texttt{None} is het equivalent van \textit{null}, terwijl \texttt{Some(T)} wil zeggen dat de waarde verschillend is van null, meer specifiek is de waarde \texttt{T}.
Aangezien het grootste deel van wat bijgehouden wordt per top eigenlijk pointers zijn maakte ik veelvoudig gebruik van deze Option-enum.
Alle pointers in een top kunnen namelijk null zijn.
De \textit{parent pointer} moet nullable zijn aangezien de root geen parent heeft, de \textit{child pointers} moeten allemaal nullable zijn omdat bladeren geen kinderen hebben (en in de interne toppen zijn niet alle kinderen altijd nodig) en de suffix-links moeten nullable zijn aangezien niet elke top een suffix link heeft naar een andere top.
\\ \\
Dit werkte perfect en kon mooi afgehandeld worden op de idiomatische manier die overeenkomt met goede Rust code.
Na de eerste benchmarks bleek het geheugengebruik echter problematisch.
Bijna exact 2x zo hoog als de equivalente C++ implementatie.
Om zo'n drastisch verschil in geheugenverbruik te kunnen verklaren moest er wel iets fundamenteel verschillen aan de manier dat toppen hun data bijhouden.
Al snel bleek dat het gebruik van \texttt{Option<usize>} als datatype in plaats van \texttt{usize} 8 bytes aan overhead per index had.
Dit is inderdaad exact het dubbele geheugenverbruik op een 64-bit machine aangezien een \texttt{usize} 8 bytes groot is.
Dit valt makkelijk te controleren aan de hand van de \texttt{std::mem::size\_of} functie, deel van de Rust standaard bibliotheek.
Onderstaand voorbeeld toont dat dit inderdaad het geval is.
\begin{minted}{Rust}
assert_eq!(mem::size_of::<Option<usize>>(), 16);
assert_eq!(mem::size_of::<usize>(), 8);
\end{minted}

Als oplossing heb ik uiteindelijk mijn eigen \textit{null} value gedefinieerd gebruik makende van een \textit{trait}\footnote{Een trait in Rust definieert een functionaliteit dat een bepaald type heeft, en kan delen met andere types}.
Deze oplossing verslaat volledig het doel van de \texttt{Option<T>} enum, maar is jammergenoeg nodig omdat het gewoonweg niet acceptabel is het geheugenverbruik te verdubbelen hiervoor.
Bovendien blijft memory safety gegarandeerd aangezien het foutief indexeren van de NULL-value (\texttt{usize::MAX} in dit geval) een index-out-of-bounds error creëert.
Wat tijdens runtime gedetecteerd wordt en dus geen verdere problemen geeft (afgezien van een mogelijke crash van het programma).

\begin{minted}{Rust}
/// Custom trait implemented by types that have a value that represents NULL
pub trait Nullable<T> {
    const NULL: T;

    fn is_null(&self) -> bool;
}

/// Type that represents the index of a node in the arena part of the tree
pub type NodeIndex = usize;

impl Nullable<NodeIndex> for NodeIndex {
    /// Use usize::MAX as NULL value since this will in practice never be reached.
    /// It is not possible to create 2^64-1 nodes (on a 64-bit machine).
    /// This would simply never fit in memory
    const NULL: NodeIndex = usize::MAX;

    fn is_null(&self) -> bool {
        *self == Self::NULL
    }
}
\end{minted}

\subsection{Performantie}\label{subsec:performantie}
Natuurlijk is het belangrijk dat de implementatie performant (en correct) is.
Aangezien we ook over een bestaande C++ implementatie van Ukkonen's algoritme beschikken, was dit een perfecte maatstaf.
Uiteindelijk heb ik één aanpassingen moeten maken in deze C++ code om een eerlijke vergelijking uit te voeren.
Oorspronkelijk werd er in elke top plaats gehouden voor 256 mogelijke kinderen.
Dit was veel te hoog voor onze usecase.
Er zijn nl. slechts 20 aminozuren en enkele \textit{wildcard characters}.
Dit verklaart onmiddellijk waarom het geheugengebruik ongeveer een factor 10 hoger was dan nodig.
Uiteindelijk ben ik gegaan voor een implementatie (zowel in Rust als C++) waarin plaats gehouden wordt voor 28 kinderen.
Dit zijn de 26 letters van het alfabet + \texttt{`\#`} + \texttt{`\$`}.
\texttt{`\#`} en \texttt{`\$`} worden gebruikt als resp. scheidingsteken en eindteken.
Dit is ook wat al gebeurde in de bestaande C++ implementatie.
\\ \\
Een andere aanpak zou kunnen zijn om HashMaps te gebruiken.
Het totale geheugenverbruik zal hierdoor afnemen naar ongeveer 60\% van het huidige verbruik, maar ten koste van performantie tijdens het zoeken (wat net erg belangrijk is).
Hoe dan ook blijft het geheugen verbruik extreem groot, welke implementatie ook gekozen wordt.
\\ \\
Het vergelijken van de implementaties heb ik opgesplitst in 2 stukken:
\begin{enumerate}
    \item het opbouwen van de index-structuur.
    \item Zoeken in de index-structuur
\end{enumerate}

\subsubsection{Opbouwen}
Om een representatief resultaat te krijgen is het opbouwen van de boom 10x uitgevoerd en zijn de gemiddelden van de resultaten genomen.
Om de uitvoeringstijd en het geheugenverbruik te meten heb ik gebruik gemaakt van het \texttt{time} commando.
De resultaten hiervan zijn terug te vinden in figuur~\ref{fig:tree_building}.
\begin{figure}[H]
    \centering
    \subfloat[Tijd nodig om de suffixboom op te bouwen]{\includegraphics[width=\linewidth]{building_tree_time}}\\[4ex] % [4ex] om wat extra vertical spacing in te voegen

    \subfloat[Maximaal gebruikt geheugen tijdens het opbouwen van de suffixboom]{\includegraphics[width=\linewidth]{building_tree_memory}}
    \caption{Vergelijking van C++ en Rust voor het opbouwen van de suffixboom. De tijd en het geheugengebruik is gemeten gebruik makende van het \texttt{time} commando met als invoerbestand de Swiss-Prot of Human-Prot eiwitdatabank.}\label{fig:tree_building}
\end{figure}

Uit deze grafieken vallen 2 duidelijke conclusies te trekken.
\begin{enumerate}
    \item De implementatie in Rust is $\pm$ 33\% sneller
    \item Het geheugenverbruik is erg vergelijkbaar.
    Dit valt te verwachten aangezien beide implementaties 8 bytes nodig hebben per \textit{pointer} en evenveel plaats voorzien voor de kinderen.
    Het kleine verschil valt te verklaren vanwege 1 veld dat ik niet bij houdt tijdens het opbouwen, dat wel gebruikt wordt in de C++ implementatie.
    Dit is de diepte van de top in de boom.
    Op de enkele plaatsen waar dit nodig is kan ik gebruik maken van andere variabelen om tot een equivalent resultaat te komen.
\end{enumerate}

\subsubsection{Zoeken}
Bij het zoeken zijn er 2 belangrijke manieren om te vergelijken.
\begin{enumerate}
    \item Zoek totdat we weten als er een match bestaat voor de peptide of niet, en stop dan.
    \item Zoek totdat er een match is, en doorzoek daarna de volledige subboom om alle informatie van de kinderen op te halen.

\end{enumerate}

\paragraph{Zoek een match}
De reden voor deze manier van zoeken is dat het mogelijk is om info te propageren van de bladeren tot bovenin de boom.
In ons geval is dit bijvoorbeeld de LCA van de taxon IDs op voorhand te berekenen.
Het zoeken van de LCA die overeenkomt met alle proteïnen waar de gevonden peptide mee matcht kan dus al stoppen vanaf er een match is.
\\ \\
Grafiek~\ref{fig:performance_match_tree} toont de nodige tijd om alle peptiden van de gebruikte zoekbestanden éénmalig te zoeken totdat er een (mis)match was voor de peptide.
De grafiek bevat de gemiddelde resultaten van 5000 uitvoeringen, maar zelfs dan bleven de resultaten wat schommelen.
Doordat de te meten tijd zo klein is, kan de kleinste invloed van omgevingsfactoren al voor een zichtbaar verschil zorgen.
Dit kan bv een achtergrondproces zijn, maar ook invloed van een andere VM die op de fysieke machine bezig is.
Dit was ook merkbaar tijdens het testen waar dat de verschillen tussen 2 opeenvolgende uitvoeringen vaak groter waren dan het verschil tussen de C++ en Rust implementatie.
Toch kunnen we besluiten dat de C++ implementatie een beetje performanter is aangezien dit in elk zoekbestand (nipt) sneller is (soms zelfs minder dan een milliseconde).

\begin{figure}[H]
    \centering
    \includegraphics[width=\linewidth]{search_match_performance_tree}
    \caption{Uitvoeringstijd in milliseconden voor het zoeken tot een match voor alle zoekbestanden. Deze resultaten zijn het gemiddelde van 5000 uitvoeringen. 1 iteratie wordt gezien als 1x elke peptide die deel is van het zoekbestand te zoeken in de suffixboom, en te stoppen wanneer er een (mis)match gevonden is. Het meten van de tijd is gebeurd in de code zelf.}
    \label{fig:performance_match_tree}
\end{figure}

Het verschil met de huidige implementatie van Unipept is gigantisch.
Daar duurt het op dit moment 2 minuten en 12 seconden om alle peptiden van het Swiss-Prot zoekbestand zonder \textit{missed cleavages} te zoeken,
en maar liefst 30 minuten 37 seconden voor het zoekbestand met \textit{missed cleavages}!
Dit is maar liefst $\frac{132 000}{152.98} = 857$ en $\frac{1 837 000}{140.64} = 13 000$ keer trager!
Als keerzijde van de medaille gebruikt Unipept op dit moment hiervoor slechts 6.7 GiB geheugen, en dit kan zelfs nog naar beneden.
Dit is ongeveer 13 keer lager!

\paragraph{Zoek match en haal informatie over kinderen op}
De reden dat dit belangrijk is, is dat alle bladeren in deze subboom de proteïnen voorstellen waar dat de gevonden peptide een deel van is.
De relevante informatie over de huidige peptide is daarom de informatie die verbonden is met deze proteïnen.

Figuur~\ref{fig:performance_all-occurrences_tree} bevat een overzicht van de nodige zoektijd voor beide implementaties op alle zoekbestanden.
We zien duidelijk dat er hier een gigantisch verschil is tussen de C++ en Rust implementatie.
Vermoedelijk komt dit door de andere \textit{memory layout} die ontstaat doordat de Rust implementatie 1 grote Vector gebruikt, terwijl de C++ implementatie losse toppen gebruikt die verspreid liggen in het geheugen.

\begin{figure}[H]
    \centering
    \includegraphics[width=\linewidth]{search_all-occurrences_performance_tree}
    \caption{Uitvoeringstijd inclusief het doorzoeken van de volledige subboom na match voor alle zoekbestanden. Deze resultaten zijn het gemiddelde van 10 uitvoeringen. 1 iteratie wordt gezien als 1x elke peptide die deel is van het zoekbestand te zoeken in de suffixboom, en bij een match de volledige resterende subboom te doorzoeken. Dit toont de tijd die nodig is om informatie uit de bladeren op te halen voor alle proteïnen waar een peptide substring van is. Het meten van de tijd is gebeurd in de code zelf.}
    \label{fig:performance_all-occurrences_tree}
\end{figure}


\section{Taxon ID aggregatie}\label{sec:taxon-id-aggregatie}
Één van de belangrijkste analyses die Unipept aanbiedt, is de taxonomische analyse waarin uitgezocht wordt met welke organismen de peptiden uit een staal overeenkomen.
Aangezien peptiden kunnen matchen met proteïnen die uit verschillende organismen komen moet er een manier gekozen worden om deze informatie te aggregeren, of te beslissen van welk organisme dit komt met de grootste kans.
Aangezien er geen manier is om met zekerheid te zeggen uit welke proteïne de peptide komt (als er meerdere opties zijn) gaat Unipept de info conservatief veralgemenen.
Anders gezegd: Unipept zal enkel info geven die geldt voor alle gematchte proteïnen.
Één van deze stukjes informatie is het Taxon ID\@.
In plaats van een lijst van alle mogelijke IDs te geven (wat een extreem grote lijst kan zijn), en wat zou vereisen de volledige subboom na het vinden van een match af te lopen, gaan we deze taxon IDs gaan aggregeren aan de hand van een strategie gebruik makende van de NCBI taxonomy database~\cite{NCBI_original_article, NCBI_update}.
Met andere woorden, we gaan dus op zoek naar de kleinste gemeenschappelijke voorouder van alle taxon IDs die in de bladeren van de subboom zitten van een bepaalde top.
Hiervoor bestaan verschillende strategieën die al uitgewerkt zijn in UMGAP, en die hier herbruikbaar waren.
\\ \\
Origineel was het plan om LCA* te gebruiken als aggregatie strategie.
Dit is een heuristiek om de LCA (lowest common ancestor) te berekenen.
Hierbij zoeken we de meest specifieke taxon in de boom die ofwel een ouder of kind is van elke taxon in de boom.
Anders gezegd is dit de LCA van een lijst taxa, nadat we alle taxa verwijdert hebben die ouder zijn van minstens één taxon in die lijst~\cite{UMGAP_paper}.
Het voordeel hiervan is dat we iets langer exactere info kunnen behouden.
Want bij LCA zelf zal het resultaat altijd één zijn vanaf één top in de subtree dit als LCA heeft (1 is namelijk ouder van alle andere taxons!).
\\ \\
Het idee was dat we niet elke keer naar de bladeren moeten gaan om het taxon ID te berekenen van 1 top, maar dat dit ging kunnen op basis van de taxon IDs van de directe kinderen van de top.
Op deze manier gingen we met één bottom-up sweep van de boom alle taxon IDs kunnen berekenen.
Dit bleek echter niet mogelijk omdat kijken naar de directe kinderen een ander resultaat geeft dan gebruik maken van de bladeren van de subboom.
Figuur~\ref{fig:lca*_diff} toont een minimaal voorbeeld uitgewerkt voor beide strategieën.
De licht grijze toppen zijn ingevuld zijn aan de hand van aggregatie, terwijl de zwarte toppen gegeven zijn.

\begin{figure}[H]
    \centering
    \subfloat[LCA* op basis van de bladeren]{
        \begin{tikzpicture}
            \node [gray] {1}
            child {node [gray] {1}
            child {node {9606}}
            child {node {10566}}}
            child {node {9606}
            };
        \end{tikzpicture}
    }\hspace{0.25\textwidth}%
    \subfloat[LCA* op basis van de kinderen]{
        \begin{tikzpicture}
            \node [gray] {9606}
            child {node [gray] {1}
            child {node {9606}}
            child {node {10566}}}
            child {node {9606}
            };
        \end{tikzpicture}
    }
    \caption{Minimaal voorbeeld van de 2 aggregatie manieren gebruik makende van LCA*. De grijze toppen zijn berekend aan de hand van een LCA*, terwijl de zwarte toppen gegeven zijn.}\label{fig:lca*_diff}
\end{figure}

Onderstaande uitleg behandelt de werkwijze voor bovenstaande figuren.
\begin{itemize}
    \item Het toepassen van LCA* voor het berekenen van de top op basis van de bladeren van de boom (\{9606, 10566, 9606\}) heeft als resultaat 1 voor de root van de boom.
    9606 en 10566 zijn geen ouder of kind van elkaar, dus zal LCA* hetzelfde doen als LCA\@.
    De kleinste gemeenschappelijke ouder van deze 2 taxons is 1.
    \item Het toepassen van LCA* op basis van de directe kinderen geeft als resultaat 9606.
    Dit val simpel te verklaren aangezien de LCA* van de linker subboom 1 is.
    Als we daarna dan de LCA* van \{1, 9606\} nemen wordt 1 verwijderd aangezien dit een ouder is van 9606.
    De LCA van 9606 is natuurlijk gewoon zichzelf!
\end{itemize}

Het berekenen van de LCA* op de eerste manier is echter niet schaalbaar voor de volledige suffixboom.
Om een idee van grootorde te geven: de suffixboom voor de Swiss-Prot dataset bevat 328 922 516 toppen in het totaal, waarvan 206 523 693 bladeren.
\\ \\
Daarom hebben we uiteindelijk toch voor de standaard LCA aggregatie manier gekozen.
Deze laat wel toe de toppen op deze efficiëntere manier te aggregeren.
UMGAP biedt 2 manieren aan om LCA te doen.
Gebruik makende van RMQ (Range Minimum Queries) en een boom-gebaseerde structuur.
Zelf maak ik gebruik van de RMQ implementatie aangezien deze significant sneller was (8 min 58 sec vs 20 min en 25 sec voor de Swiss-Prot databank).
Tot slot heb ik ook eens vergeleken hoe groot de behaalde tijdswinst is bij het gebruik kunnen maken van de directe aggregatie op de kinderen, vergeleken met moeten aggregeren op de bladeren.
Bij het aggregeren op basis van de bladeren met de RMQ implementatie was de uitvoeringstijd maar liefst 12 uur, 19 minuten en 16 seconden!
Dit is dus een extreem groot verschil.

\section{Conclusie suffixbomen}\label{sec:conclusie-suffix-bomen}
Het is duidelijk dat suffixbomen erg performant zijn voor dit scenario.
Het opbouwen gebeurt snel en het zoeken voor een match gaat vliegensvlug.
\\ \\
Door de eigen implementatie in Rust kunnen we ook wat tijd besparen ten opzichte van een equivalente C++ implementatie.
Een deel van de winst zit tijdens het opbouwen van de boom, maar vooral tijdens het zoeken wanneer informatie uit de bladeren gehaald moet worden.
Vermoedelijk ligt de andere geheugenstructuur hiervoor aan de basis.
\\ \\
Ondanks de veelbelovende resultaten op vlak van snelheid is er een keerzijde aan de medaille.
Het geheugengebruik is zo groot dat we op zoek moeten naar een andere datastructuur.
Voor de Swiss-Prot databank gaat het geheugenverbruik al boven 80 GB, terwijl ons einddoel is om dit te gebruiken op de TrEMBL dataset.
Dit wilt zeggen dat als we alles \textit{slechts} $\pm$ 500 maal kunnen opschalen het doel bereikt is.
Dit zou echter vereisen dat we een server hebben met ongeveer 50 TB aan RAM geheugen.
Dit is niet mogelijk, we moeten daarom op zoek naar een andere datastructuur die minder geheugen vereist.


    \chapter{Suffix arrays}\label{ch:suffix-arrays}
Een tweede datastructuur die we in meer detail bekijken, is de suffix array.
Het voordeel van deze datastructuur is het lagere geheugengebruik in vergelijking met suffixbomen.


\section{Wat zijn suffix arrays?}\label{sec:wat-zijn-suffix-arrays?}
Suffix arrays (SAs) zijn een geheugenefficiëntere voorstelling voor de bladeren van een suffixboom.
In plaats van een boomstructuur met een veel pointers maken ze gebruik van \textbf{een array die de volgnummers van elke suffix in de originele string bevat}.
Deze volgnummers worden lexicografisch gesorteerd op basis van de overeenkomstige suffix.
Figuur~\ref{fig:suffixtree_vs_suffixarray} geeft een voorbeeld van een suffixboom en suffix array opgebouwd over de tekst \texttt{acacgt\$}.

\begin{center}
    \texttt{tekst: a|c|a|c|g|t|\$\\index: 0|1|2|3|4|5|6}
\end{center}
\begin{figure}[H]

    \begin{subfigure}[b]{0.6\linewidth}
        \resizebox{\linewidth}{!}{
            \begin{tikzpicture}
            [
                level 1/.style = {sibling distance = 2.5cm},
                level 2/.style = {sibling distance = 1cm}
            ]

                \node[draw, circle] (End2) {}
                child {
                    [fill] circle (2pt)
                    edge from parent node [above] {6,7}
                }
                child {
                    node[draw, circle] (Start1) {}
                    child {
                        [fill] circle (2pt)
                        edge from parent node [left] {2,7}
                    }
                    child {
                        [fill] circle (2pt)
                        edge from parent node [right] {4,7}
                    }
                    edge from parent node [below] {0,2}
                }
                child {
                    node[draw, circle] (End1) {}
                    child {
                        [fill] circle (2pt)
                        edge from parent node [left] {2,7}
                    }
                    child {
                        [fill] circle (2pt)
                        edge from parent node [right] {4,7}
                    }
                    edge from parent node [left] {1,2}
                }
                child {
                    [fill] circle (2pt)
                    edge from parent node [below] {4,7}
                }
                child {
                    [fill] circle (2pt)
                    edge from parent node [above] {5,7}
                }
                ;
                \draw[dashed, ->] (Start1) to[out=-20,in=200] (End1);
                \draw[dashed, ->] (End1) to[out=60,in=-60] (End2);
            \end{tikzpicture}
        }
        \caption{Suffixboom}
    \end{subfigure}
    \begin{subfigure}[b]{0.4\linewidth}
        \centering
        \begin{tikzpicture}[thick,scale=.6]
            \draw (0,0) grid (7,1);
            \path (.5,.5) node{$6$} foreach \i in {0,2,1,3,4,5} {++(1,0) node{$\i$}};
        \end{tikzpicture}
        \vspace{3em} % vertically center the array a bit
        \caption{Suffix array}
    \end{subfigure}

    \caption{Suffixboom en suffix array voor de string \texttt{acacgt\$}.}\label{fig:suffixtree_vs_suffixarray}
\end{figure}

Wanneer we de bladeren van de suffixboom van links naar rechts doorlopen, dan zien we dat dit overeen komt met de suffix array.
Dit is de link tussen deze twee datastructuren.
Merk op dat \textbf{de suffix array minder data bevat ten opzichte van de suffixboom}.
De interne knopen en suffix links uit de suffixboom ontbreken.
Indien deze informatie ook nodig is, kan gebruik gemaakt worden van zogenaamde Enhanced Suffix Arrays (ESAs).
Hierbij worden naast de suffix array nog drie extra tabellen bijgehouden: de Longest Common Prefix (LCP), child en suffix link tabellen.


\section{Complexiteit}\label{sec:complexiteit}
Aangezien de suffix array bestaat uit de volgnummers van de lexicografisch gesorteerde suffixen, kunnen we deze suffixen aan de hand van traditionele sorteeralgoritmen zoals merge sort~\cite{mergeSort} in $O(n^2 \log n)$ tijd en $O(n^2)$ geheugen sorteren, wat onmiddellijk de bijhorende suffix array als resultaat oplevert.
Hierbij is $n$ de lengte van de tekst.
De totale tijdscomplexiteit hiervan is $O(n^2 \log n)$ en niet $O(n \log n)$ aangezien elke vergelijking van twee suffixen in het slechtste geval $n$ karaktervergelijkingen vereist.
Ondertussen bestaan er echter verschillende algoritmen die een tijdscomplexiteit van $O(n)$ bereiken~\cite{sais, ko_alura, radixSA, dark_archon, libdivsufsort} voor het opbouwen van de suffix array van een tekst.
Bovendien vereisen deze veel minder geheugen dan een equivalente suffixboom.
Sommige implementaties vereisen slechts $5n + O(1)$ geheugen~\cite{dark_archon, libdivsufsort, libsais}.
\\ \\
Zoeken in een suffix array kan in $O(m \log n)$.
Hierbij is $n$ opnieuw de lengte van de tekst, en $m$ de lengte van de zoekstring.
Het resultaat van deze zoekopdracht is een interval in de suffix array waarbinnen de matches liggen, of een leeg interval indien er geen matches zijn.
Indien we effectief alle matches willen ophalen moeten we deze allemaal overlopen en komt de totale complexiteit uit op $O(m \log n + |Occ|)$ met $|Occ|$ het aantal gevonden matches in de tekst.


\section{Bestaande implementaties}\label{sec:bestaande-implementaties}
Aangezien er meerdere sterk geoptimaliseerde implementaties bestaan voor het opbouwen van een suffix array, vergelijken we eerst de performantie van deze implementaties.
Hieruit kunnen we nadien een snelle en geheugenefficiënte implementatie selecteren.
Tabel~\ref{tab:sa_building} bevat een overzicht van verschillende algoritmen.
Voor sommige werden verschillende implementaties getest.

\begin{table}[H]
    \begin{minipage}{\linewidth}
        \centering
        \begin{tabular}{l l S[table-format=-2.2] S[table-format=-2.2] S[table-format=-1.2] S[table-format=-1.2]}
            Algoritme & Programmeertaal & \multicolumn{2}{c}{Tijd (in s)} & \multicolumn{2}{c}{Geheugen (in GB)} \\
            \hline\hline
            &                      & {32-bit} & {64-bit} & {32-bit} & {64-bit} \\
            \cline{3-6}
            libdivsufsort\footnote{\url{https://github.com/y-256/libdivsufsort}}                                       & C                    & 15.01    & 15.97    & 1.03     & 1.86     \\
            libdivsufsort\footnote{\url{https://github.com/baku4/libdivsufsort-rs}}                                    & Rust + bindings to C & 16.00    & 15.52    & 1.03     & 1.86     \\
            libdivsufsort\footnote{\url{https://github.com/fasterthanlime/stringsearch/tree/master/crates/divsufsort}}  & Rust                 & 20.23    & {-}      & 1.03     & {-}      \\
            dark archon a4\footnote{\url{https://github.com/kvark/dark-archon}}                                        & C                    & 39.34    & {-}      & 1.09     & {-}      \\
            libsais\footnote{\url{https://github.com/IlyaGrebnov/libsais}}                                             & C                    & 6.38     & 6.46     & 1.03     & 1.86     \\
            SA-IS\footnote{\url{https://github.com/Tascate/Suffix-Arrays-in-CPP}}                                      & C++                  & 24.39    & {-}      & 3.80     & {-}      \\
            SA-IS\footnote{\url{https://github.com/sile/sais}}                                                         & C++                  & 18.73    & {-}      & 1.46     & {-}      \\
            radixSA\footnote{\url{https://github.com/mariusmni/radixSA64}}                                             & C++                  & 9.74     & 11.26    & 2.11     & 3.52     \\
            \hline
        \end{tabular}
        \caption{Uitvoeringstijd en maximaal geheugengebruik voor het opbouwen van een suffix array aan de hand van verschillende algoritmen voor de Swiss-Prot eiwitdatabank.
        Indien er een 32-bit en 64-bit integer implementatie beschikbaar was, werden deze allebei getest. Een - staat voor niet getest. Deze testen werden lokaal uitgevoerd op een M1 Pro MacBook Pro. De specificaties hiervan zijn terug te vinden in tabel~\ref{tab:macbook_hardware}.}
        \label{tab:sa_building}
    \end{minipage}
\end{table}

We kunnen concluderen dat \textbf{libsais duidelijk de snelste implementatie is om Swiss-Prot te indexeren}.
\textbf{Samen met libdivsufsort gebruikt deze de laagste hoeveelheid geheugen}, wat libdivsufsort ook interessant maakt.
Een ander voordeel dat libsais en libdivsufsort gemeen hebben, naast hun minimale geheugengebruik, is dat ze allebei een 64-bit integer implementatie hebben.
Dit is belangrijk voor het indexeren van UniprotKB omdat de totale tekst langer is dan het grootste 32-bit integer.
Dit zorgt ervoor dat alle 32-bit integer implementaties onbruikbaar zijn voor dit einddoel.
Tot slot valt ook te zien dat het verschil tussen de C en Rust versie die bindings heeft naar de C code klein is.
De overhead van het oproepen van de C code uit Rust is dus minimaal.

\subsection{Enhanced suffix arrays}\label{subsec:enhanced-suffix-arrays}
Om een indicatie te hebben van de extra hoeveelheid geheugen die nodig is om gebruik te maken van Enhanced Suffix Arrays kunnen we via de libsais bibliotheek LCP array berekenen.
Deze LCP array is slechts één van de drie extra tabellen die deel zijn van ESAs, maar het is voor ons wel de meest interessante.
De baseline waarmee we willen vergelijken is de hoeveelheid geheugen die we nodig hebben om enkel de SA op te bouwen.
Voor de Swiss-Prot eiwitdatabank was dit 2.3 GB\@.
Als we ook de LCP array berekenen, loopt het maximale geheugengebruik op naar 5.72 GB\@.
\textbf{Het berekenen van deze extra array vraagt dus ongeveer dubbel zoveel geheugen, waardoor we deze optie niet verder verkennen}.
Bovendien berekenen we deze LCP array om het voorberekenen van de geaggregeerde taxon ID voor elke interne top van de overeenkomstige suffixboom mogelijk te maken.
Daarna moet er op basis van deze SA en LCP array nog een compacte representatie gevonden worden van de boomstructuur.
Dit is niet vanzelfsprekend.


\section{Toepassen van suffix arrays op een eiwitdatabank}\label{sec:toepassen-van-suffix-arrays-op-een-eiwitdatabank}
Het moeilijkste deel van onze probleemstelling is het opbouwen van de suffix array.
Dit kunnen we oplossen aan de hand van de algoritmen uit sectie~\ref{sec:bestaande-implementaties}.
Eens we die suffix array opgebouwd hebben, blijft er echter nog een stuk van ons probleem over.
Eerst moeten we nog een mapping maken van de gevonden suffixen naar het bijbehorende eiwit.
Op basis van dit eiwit wordt daarna de LCA gezocht.

\subsection{Bouwen van de suffix array}\label{subsec:bouwen-van-de-suffix-array}
Zoals in de inleiding van sectie~\ref{sec:probleemstelling} beschreven wordt, willen we Rust gebruiken vanwege de combinatie tussen \textit{memory safety} en hoge performantie.
We willen echter gebruikmaken van de al bestaande geoptimaliseerde algoritmen om een suffix array op te bouwen.
Om beide doelen te bereiken, maken we gebruik van de interoperabiliteit tussen Rust en C/C++.
Zo bestaan er al bindings\footnote{\url{https://crates.io/crates/libdivsufsort-rs}} van Rust naar de originele implementatie van libdivsufsort\cite{libdivsufsort} (in C).
Ook al blijkt uit het testen dat dit algoritme voor het opbouwen van de indexstructuur over een kleinere eiwitdatabank niet het snelste is, is het geheugengebruik wel minimaal.
Dit laat toe om te experimenteren met het opbouwen van een SA zonder al te veel extra werk, en al onmiddellijk te zien hoe het geheugengebruik evolueert.
\\ \\
Later hebben we zelf nog een simple Rust wrapper geschreven rond de libsais C-code.
Hiervoor hebben we de \texttt{bindgen}\footnote{\url{https://crates.io/crates/bindgen}} crate gebruikt.
Op deze manier was het ook mogelijk om gebruik te maken van de snellere libsais algoritme eens we wisten dat SAs een efficiënte en schaalbare oplossing waren voor de probleemstelling.
\\ \\
Het nadeel van het gebruiken van deze bindings naar C-code is dat het oproepen van de effectieve C-code gebeurt in een \texttt{unsafe} blok.
Hierbij is het dus mogelijk dat er geheugenfouten in het programma sluipen.
Dit risico is echter miniem aangezien dit bestaande, geteste bibliotheken zijn.
Bovendien zijn we ook zeker dat eventuele geheugenfouten enkel hierdoor kunnen ontstaan en is dit de verantwoordelijkheid van de ontwikkelaar van de bibliotheek.
Dit is dus een afweging tussen optimale performantie (waarbij we het wiel niet hoeven heruit te vinden), en garantie van \textit{memory safety}.

\subsection{Mapping van suffix naar proteïne}\label{subsec:mapping-van-suffix-naar-proteine}
Bepalen welke proteïne hoort bij een bepaalde suffix kan op twee manieren.
\textbf{Een eerste optie, laten we deze een \textit{dense mapping} noemen, is om expliciet voor elke suffix bij te houden bij welke proteïne die hoort}.
Dit kan aan de hand van een array die even lang is als het aantal suffixen.
Het voordeel van deze aanpak is dat het vinden van de bijbehorende proteïne in $O(1)$ tijd kan, hiervoor is echter wel $O(m)$ geheugen nodig, met $m$ de lengte van de totale tekst.
\\ \\
\textbf{De tweede optie, aan de hand van een \textit{sparse mapping}, is om enkel de eerste of laatste suffix per proteïne bij te houden}.
Het voordeel van deze aanpak is dat er minder geheugen nodig is, meer precies $O(p)$ geheugen met $p$ het aantal proteïnen.
Het nadeel is dan weer dat het vinden van de bijbehorende proteïne trager is.
Dit neemt $O(\log p)$ tijd in beslag aan de hand van binair zoeken.
\\ \\
In Figuur~\ref{fig:dense_vs_sparse} wordt de uitvoeringstijd voor beide implementaties vergeleken.
\textbf{Standaard (en in alle komende testen) zullen wij gebruikmaken van de \textit{sparse mapping} aangezien de performantie impact in de praktijk beperkt is, en er een significante hoeveelheid geheugengebruik uitgespaard kan worden}.
Zo loopt het verschil in geheugengebruik op tot 0.8 GB voor Swiss-Prot.
Dit is maar liefst 25\% van de indexgrootte bij de \textit{dense mapping}.
Het grote verschil in uitvoeringstijd bij het Human-Prot peptidebestand valt te verklaren vanwege het extreem hoog aantal matches dat daar te vinden is voor de korte peptiden.
In de praktijk zullen we dit maximaal aantal matches echter limiteren (zie sectie~\ref{subsec:zoeken}).
Dit zal op zijn beurt ook de overhead van de sparse mapping beperkt houden.
\begin{figure}[H]
    \centering
    \subfloat[Tijd nodig om alle matches te zoeken.]{\includegraphics[width=\linewidth]{dense_vs_sparse_time}}\\[4ex] % [4ex] om wat extra vertical spacing in te voegen
    \centering
    \subfloat[Maximaal gebruikt geheugen tijdens het zoeken naar alle matches.]{\includegraphics[width=\linewidth]{dense_vs_sparse_memory}}
    \caption{Zoektijd en geheugengebruik bij het gebruik bij een \textit{dense} of \textit{sparse mapping} van de suffixen naar de proteïnes.}\label{fig:dense_vs_sparse}
\end{figure}

\subsection{Berekenen van de LCA}\label{subsec:berekenen-van-de-lca}
Zoals eerder vermeld, bevat een suffix array geen informatie over de interne toppen die voorkomen bij een suffixboom.
Dit zorgt ervoor dat het niet mogelijk is om op basis hiervan de LCA van de organismen voor te berekenen voor al deze interne toppen.
Dit moet nu \textit{on the fly} gebeuren tijdens het zoekproces zelf.
Hierdoor is er ook \textbf{geen reden meer om LCA te gebruiken in de plaats van LCA*}.
LCA* was namelijk onze eerste keuze aangezien het resultaat hierbij minder onderhevig is aan eiwitten die een te algemene annotatie hebben.
Bij suffixbomen zijn we daar echter van af moeten stappen om het voorberekenen efficiënter te maken.


\section{Sparse en compressed suffix arrays}\label{sec:sparse-en-compressed-suffix-arrays}
Om het geheugengebruik van suffix arrays verder te verkleinen, kan er gebruikgemaakt worden van sparse of compressed suffix arrays.
In principe doen ze allebei hetzelfde.
Er wordt namelijk slechts een stuk van de originele suffix array bijgehouden.
Het verschil zit in welk stuk bijgehouden wordt.
Zoals het gezegde \textit{There is not such thing as free lunch} ons vertelt, heeft alles zijn voor- en nadelen.
Het verkleinen van de SA heeft een negatieve impact op de zoekperformantie.
We zullen namelijk slechts een deel van de peptide kunnen zoeken in de verkleinde SA\@.
Dit deel van de peptide is natuurlijk korter dan de volledige peptide, en zal in het algemeen meer matches opleveren.
Daarna moet voor elk van deze matches het niet-gematchte deel van de peptide nog gecontroleerd worden.
\\ \\
\textbf{Sparse suffix arrays (SSAs)} bouwen een suffix array op basis van elke k-de suffix van de input tekst.
Indien we slechts 50\% van de suffixen bijhouden, komt dit erop neer dat enkel suffixen die beginnen op een \textbf{even index in de inputtekst} bijgehouden worden.
Bij \textbf{compressed suffix arrays (CSAs)} wordt daarentegen slechts elke k-de waarde van de SA bijgehouden.
Indien we in dit geval 50\% van de suffixen bijhouden komt dit erop neer dat enkel suffixen die op een \textbf{even index staan in de opgebouwde suffix array} over blijven.
\\ \\
Voor beide opties is de populairste manier om ze op te bouwen aan de hand van \textbf{\textit{sampling} op de volledige SA\@}.
Hierdoor blijft het maximale geheugengebruik tijdens het opbouwen identiek aan het gebruik van de volledige SA\@.
De resulterende index zal wel kleiner zijn, waardoor de server die deze index zal hosten lagere geheugenvereisten heeft.
\\ \\
Het samplen van een opgebouwde suffix array is de meest gebruikte methode tot op vandaag, vanwege de bestaande sterk geoptimaliseerde implementaties van de klassieke SA constructiealgoritmen.
Bij sparse suffix arrays is het beste algoritme qua tijdscomplexiteit tot nu toe een Monte Carlo algoritme dat $O(n)$ tijd en $O(b)$ geheugen nodig heeft en een Las Vegas algoritme dat $O(n \sqrt{\log b})$ tijd en $O(b)$ geheugen verbruikt.
Hierbij is $n$ de lengte van de tekst, en $b$ het aantal effectief gebruikte suffixen in de sparse SA~\cite{building_sparse_sa}.
Van het Monte-Carlo algoritme is er een bestaande implementatie\footnote{\url{https://github.com/lorrainea/SSA/tree/main/MA}}.
Wanneer we aan de hand hiervan een SSA met sparseness factor 3 voor Swiss-Prot opbouwen, blijkt dit niet alleen trager te zijn ($\pm$ 10 minuten).
Ook het geheugengebruik ligt merkelijk hoger ($\pm$ 10 GB).
Terwijl we slechts een tiental seconden nodig hebben om de standaard SA op te bouwen in combinatie met 1.8 GB RAM\@.
Voor compressed suffix arrays bestaat er een algoritme dat een tijdscomplexiteit van $O(n)$ heeft in combinatie met $O(n \log \sigma)$ bits aan geheugen~\cite{building_compressed_sa}.
Hierbij is $n$ de tekstlengte en $\sigma$ de alfabetgrootte.
\\ \\
Het grootste nadeel aan deze algoritmen in de context van deze thesis is dat er nog geen sterk geoptimaliseerde implementaties bestaan.
Bovendien zal de \textbf{factor van ingevoegde sparseness in ons geval altijd vrij klein} zijn om de zoektijden beperkt te houden (aangezien we werken met een erg grote dataset en vrij korte strings).
Indien we dus een implementatie hebben om rechtstreeks een CSA of SSA te bouwen, maar met een grotere constante qua geheugengebruik, zal vanwege de kleine sparseness factor de winst snel verloren gaan.
\textbf{In ons geval is een SSA interessanter dan een CSA} omwille van de verschillende restricties tijdens het zoeken die een SSA en CSA hebben.
\\ \\
Bij het gebruik van sampling factor $k$ kunnen we bij een SSA alle sequenties van lengte $k$ en groter zoeken.
Bij een CSA is het mogelijk dat ook sommige sequenties die groter zijn dan $k$ niet te vinden zijn, en sommige kortere wel.
Dit omdat het mogelijk is dat er in het slechtste geval een gat van $n - \frac{n}{k}$ suffixen zit tussen twee opeenvolgende bijgehouden suffixen in de CSA\@.
Hierbij is $n$ de lengte van de invoertekst.
Het zoeken van extreem korte sequenties is echter niet interessant omdat deze erg weinig informatie bevatten, terwijl het verliezen van enkele langere sequenties (zonder te kunnen voorspellen welke) net extra informatieverlies met zich meebrengt.
\\ \\
Bovenstaand probleem kan opgelost worden door gebruik te maken van extra datastructuren.
Zo kan een FM-index~\cite{fm_index} gebruikmaken van een CSA, en kunnen we hierbij wel alle sequenties zoeken.
Om dit te doen moeten we echter tijdens het opbouwen van de indexstructuur nog bijkomende onderdelen berekenen (zoals de BWT\footnote{Burrows–Wheeler transformatie~\cite{bwt}} van de tekst).
Dit verhoogt opnieuw het geheugengebruik en de uitvoeringstijd tijdens het opbouwen.

\subsection{Zoeken met sparse suffix arrays}\label{subsec:zoeken-in-sparse-suffix-arrays}
Het zoeken met een SSA is erg gelijkaardig aan het zoeken met een volledige SA\@.
Een belangrijke \textbf{restrictie} is echter dat met de SSA \textbf{geen strings gezocht kunnen worden die kleiner zijn dan de sparseness factor $k$}.
Bovendien heeft deze sparseness factor ook impact op de zoekperformantie.
Bij het zoeken met sparseness factor $k$ moeten we $k$ verschillende zoekopdrachten uitvoeren.
In zoekopdracht $i$ worden de eerste $i - 1$ tekens van de zoekstring overgeslagen.
In plaats van de zoekstring, zoeken we dus een suffix van de zoekstring in de sparse indexstructuur.
\\ \\
Voor elke matchende suffix (stel dat dit suffix $s$ is) moet daarna gecontroleerd worden of de overgeslagen prefix van $i$ tekens matcht met de $i$ tekens die op posities $[s - i, i[$ in de geïndexeerde tekst staan.
Naast de SSA moet dus ook de volledige tekst in het werkgeheugen bijgehouden om te zoeken in de (sparse) SA\@.
Als dit zo is, dan matcht suffix $s - i$ met de gezochte string.
Het eindresultaat is de unie van de resultaten van elke iteratie.
Figuur~\ref{fig:sparse_sa} geeft een voorbeeld van het opzoeken van de peptide \texttt{acg} in de tekst \texttt{acacgt\$}.
Hierbij wordt gebruikgemaakt van een SA met sparseness factor $k$ = 3.

\begin{center}
    \texttt{tekst: a|c|a|c|g|t|\$\\index: 0|\textcolor{lightgrey}{1}|\textcolor{lightgrey}{2}|3|\textcolor{lightgrey}{4}|\textcolor{lightgrey}{5}|6}
\end{center}
\begin{figure}[H]
    \hfill
    \begin{subfigure}[t]{0.45\linewidth}
        \centering
        \begin{tikzpicture}[thick,scale=.6]
            \draw (0,0) grid (7,1);
            \path (.5,.5) node{$6$} foreach \i in {0,2,1,3,4,5} {++(1,0) node{$\i$}};
        \end{tikzpicture}
        \caption{Suffix array}
    \end{subfigure}
    \hfill
    \begin{subfigure}[t]{0.45\linewidth}
        \centering
        \begin{tikzpicture}[thick,scale=.6]
            \draw (0,0) grid (3,1);
            \path (.5,.5) node{$6$} foreach \i in {0,3} {++(1,0) node{$\i$}};
        \end{tikzpicture}
        \caption{Sparse suffix array met sparseness factor 3.}
    \end{subfigure}
    \hfill
    \caption{SA en SSA voor de tekst \texttt{acacgt\$}.}
    \label{fig:sparse_sa}
\end{figure}

\begin{enumerate}
    \item Sla 0 tekens over.
    Zoek \texttt{acg} met de SSA\@.
    We vinden geen matches.
    \item Sla 1 teken over.
    Dit wil zeggen dat we \texttt{cg} zoeken met de SSA\@.
    Dit levert één match op met suffix 3 (op index 1 in de SSA).
    \begin{enumerate}
        \item Controleer of de suffix die deels gematcht is ook volledig matcht met onze zoekstring.
        Dit wil zeggen dat we de overgeslagen prefix van $i = 1$ tekens (in dit geval de letter \texttt{a}) moeten kunnen matchen met de eerste $i$ tekens van suffix $3 - 1 = 2$.
        \item We zien dat het eerste teken van suffix 2 inderdaad een \texttt{a} is.
        Dit kan rechtstreeks via de tekst gecontroleerd worden aan de hand van $i$ karakters die vergeleken moeten worden.
        Suffix 2 is dus een match voor de zoekstring \texttt{acg}.
    \end{enumerate}
    \item Sla 2 tekens over.
    Zoek \texttt{g} in de SSA\@.
    We vinden geen matches.
    \item We concluderen dat enkel suffix 2 matcht met de gezochte string \texttt{acg}.
\end{enumerate}

\subsubsection{Performantie en indexgrootte}
Aangezien de sparseness factor $k$ impact heeft op de zoekperformantie is het belangrijk te zien hoe groot deze impact is, zodat we een goede keuze kunnen maken.
Op het eerste gezicht lijkt het vergroten van deze factor enkel meer iteraties toe te voegen, maar dit heeft zware gevolgen.
Deze extra iteraties zullen ervoor zorgen dat er \textbf{kortere strings} in de SSA gezocht worden, die in het algemeen \textbf{meer matches} opleveren.
Zeker wanneer de zoekstrings zelf al vrij kort zijn, wat voor unipept typisch het geval is.
Voor tryptische peptiden zijn er bijvoorbeeld veel peptiden met lengte 5.
Voor al deze matches moeten we controleren of de tekens die ervoor komen matchen met de overgeslagen prefix.
Figuur~\ref{fig:search_sparseness}~(a) visualiseert de impact op de zoektijd van het Swiss-Prot peptidebestand met en zonder \textit{missed cleavages}.
Deze bestanden worden hiervoor gebruikt omdat de kortste sequentie die ze bevatten 5 aminozuren lang is.
Hierdoor blijft voor het gebruikte interval nog steeds elke peptide zoekbaar.
Bij het Human-Prot zoekbestand zijn er ook kortere peptiden die we moeten overslaan, waardoor dit een slechte representatie is voor de evolutie van de zoektijd.
\\
\begin{figure}[H]
    \centering
    \subfloat[Zoektijd voor het Swiss-Prot peptidebestand met en zonder \textit{missed cleavages}. De zoekoperaties zijn uitgevoerd op een sparse suffix array gebouwd op basis van de Swiss-Prot eiwitdatabank, met $k = 1, 2, 3, 4, 5$.]{\includegraphics[width=0.9\linewidth]{swissprot_searchtime_sparseness}}\\[4ex] % [4ex] om wat extra vertical spacing in te voegen

    \subfloat[Grootte van de volledige index en SSA voor de Swiss-Prot databank met sparseness factor $k = 1, 2, 3, 4, 5$.]{\includegraphics[width=0.8\linewidth]{index_size_SSA}}
    \caption{Zoektijd en indexgrootte van een SSA voor Swiss-Prot met $k = 1, 2, 3, 4, 5$.}\label{fig:search_sparseness}
\end{figure}

De zoektijd explodeert wanneer we de sparseness factor $k$ verhogen, omdat een deel van de gezochte peptiden bestaat uit 5 aminozuren.
Wanneer we deze proberen zoeken in een SSA met sparseness factor 5, zoeken we eigenlijk enkel één letter in de SSA (de laatste van de peptide).
Dit zal erg veel matches opleveren, en op zijn beurt erg veel werk vereisen om alle prefixen te controleren.
\\ \\
Anderzijds zal het verhogen van de sparseness factor de indexgrootte verkleinen.
Hierbij is het belangrijk dat enkel de SA mee verkleint.
Er blijft \textbf{altijd een vaste hoeveelheid geheugen nodig om de tekst zelf op te slaan}.
In Figuur~\ref{fig:search_sparseness}~(b) is duidelijk te zien dat, voor dit geval, de overgang van sparseness factor 4 naar 5 erg weinig winst heeft qua nodige opslagruimte, maar een grote impact heeft op de performantie.
\\ \\
Er zijn dus twee erg belangrijke bevindingen over sparse suffix arrays, en de manier waarop wij ze opbouwen.
\begin{enumerate}
    \item Probeer de sparseness factor zo laag mogelijk te houden, zo blijft ook de zoektijd voor kortere peptiden beperkt.
    \item Het maximale geheugengebruik voor het opbouwen van de SSA blijft constant onafhankelijk van de sparseness factor.
    De hoeveelheid geheugen om ze nadien te gebruiken kunnen we wel verkleinen.
    Hierdoor kan de index op minder krachtige machines gebruikt worden na het opbouwen.
\end{enumerate}

Dit impliceert dat we de sparseness factor $k$ enkel moeten verhogen om het geheugengebruik te beperken bij een al opgebouwde indexstructuur.
Dit is vooral van toepassing voor UniProtKB waar het nuttig is om kort een krachtige machine te gebruiken die de SA bouwt, waarna een minder krachtige machine de SSA host.
In het perfecte geval wordt \textbf{de sparseness factor zo gekozen zodat de SSA samen met de tekst net in het RAM geheugen past} van deze minder krachtige machine.


\section{Parallellisatie}\label{sec:parallellisatie}
Om het zoeken nog verder te versnellen, kan gebruikgemaakt worden van parallellisatie.
We hebben namelijk een \textbf{groot aantal peptiden} waarvoor we telkens dezelfde \textbf{statische indexstructuur} moeten doorzoeken.
Rust maakt dit proces vrij simpel omdat het \textit{ownership} systeem dataraces voorkomt (behalve wanneer gebruikgemaakt wordt van \textit{unsafe} code of het \textit{interior mutability} patroon)\cite{rust_data_races}.
Om een datatype te gebruiken in combinatie met multithreading moet deze de \texttt{Sync} en \texttt{Send} trait implementeren.
Deze traits worden door het typesysteem automatisch afgeleid.
Namelijk, wanneer alle componenten van een type aan de \texttt{Sync} en \texttt{Send} trait voldoen, dan voldoet je nieuwe type ook automatisch.
\\ \\
Uiteindelijk hebben we twee geparallelliseerde implementaties gemaakt.
In de eerste wordt alles volledig zelf beheerd.
Hierbij verdelen we zelf welke data naar welke thread gaat, worden de threads manueel opgestart en sluiten we ze ook zelf af.
In de tweede implementatie wordt gebruikgemaakt van de Rayon crate~\cite{rayon}.
Deze laat toe om op een simpele manier een sequentiële lus over een variabele te parallelliseren.
In ons geval was het omzetten van een sequentiële implementatie (nadat alle types voldeden aan de \texttt{Sync} en \texttt{Send} trait) zo simpel als het vervangen van \texttt{.iter()} door \texttt{.par\_iter()}.
Ook in deze implementatie is het mogelijk om manueel een specifiek aantal threads te kiezen.
Standaard gebruikt Rayon het aantal beschikbare logische cores op de machine, maar het instellen van een ander aantal kan aan de hand van één lijntje code.

\begin{minted}{Rust}
// Sequentieel
let results = peptiden
    .iter()
    .map(|peptide| search_peptide(peptide))
    .collect();

// Parallel
let results = peptiden
    .par_iter()
    .map(|peptide| search_peptide(peptide))
    .collect();
\end{minted}

\subsection{Manueel threaden vs Rayon}\label{subsec:manueel-threaden-vs-rayon}
Aangezien we twee verschillende implementaties hebben, is het interessant om na te gaan hoe deze ten opzichte van elkaar presteren.
Figuur~\ref{fig:threading_default_vs_rayon} toont de evolutie van de zoektijden voor een verschillend aantal threads.
We zien duidelijk dat de versie die gebruikmaakt van \textbf{Rayon net iets sneller} is en dat beide implementaties ongeveer \textbf{lineair schalen}.
De schaling is niet perfect één-op-één ten opzichte van het aantal threads omdat het inlezen en uitschrijven van de output sequentieel blijft.
Het verschil in uitvoeringstijd valt mogelijks te verklaren aan de manier waarop de data verdeeld wordt over de threads, in combinatie met een efficiëntere manier om de resultaten uit de threads te verwerken.
In de eigen implementatie kreeg elke thread simpelweg $\frac{1}{x}$ van alle peptiden toegekend, met $x$ het aantal threads.
De resultaten moeten daarna aan de hand van enkele stappen uit de thread-scope gehaald worden zodat deze terug beschikbaar zijn voor de rest van het programma.
Rayon maakt gebruik van een ingewikkelder systeem aan de hand van \textit{work stealing}~\cite{rayon_stealing} om de data over de threads te verdelen.



\begin{figure}[H]
    \centering
    \subfloat[Absolute uitvoeringstijd voor 1--12 threads.]{\includegraphics[width=0.7\linewidth]{threading_default_vs_rayon}}
    \label{fig:threading_default_vs_rayon_1}
\end{figure}
\begin{figure}[H]\ContinuedFloat
    \centering
    \subfloat[Relatieve versnelling ten opzichte van uitvoering op 1 thread.]{\includegraphics[width=0.7\linewidth]{threading_default_vs_rayon_relative}}
    \caption{Tijdsmeting om het strikt tryptische Swiss-Prot peptidebestand te zoeken in een index met sparseness factor 3 voor 5\% van UniProtKB.}\label{fig:threading_default_vs_rayon}
\end{figure}

Naast het verschil in performantie zijn er nog andere voordelen verbonden aan het gebruik van Rayon.
De \textbf{code is namelijk veel simpeler en daarom ook beter te onderhouden}.
Dit motiveert onze keuze om finaal gebruik te maken van Rayon.


\section{Suffix arrays vs Suffixbomen}\label{sec:performantie}
Nu we verschillende manier verkend hebben om suffix arrays op te bouwen, is het interessant om suffix arrays te vergelijken met suffixbomen.
Op basis hiervan kunnen we vaststellen welke verbeteringen we gemaakt hebben, of deze indexstructuur goed genoeg schaalt om toepasbaar te zijn op de volledige UniProtKB databank en in welke tijd de peptiden gezocht kunnen worden.

\subsection{Opbouwen}\label{subsec:opbouwen}
Figuur~\ref{fig:array_building} visualiseert de tijd nodig om de indexstructuur op te bouwen in combinatie met het geheugengebruik.
Er is duidelijk een \textbf{mooie tijdswinst} verkregen, wat een bonus is voor het lokaal opbouwen van indices.
Voor het opbouwen van de index op UniProtKB is dit echter minder van belang aangezien dit proces slechts om de 8 weken moet gebeuren.
Zolang de nodige CPU-tijd hiervoor niet langer dan één à twee dagen is, is dit acceptabel.
Een veel belangrijkere vaststelling is het \textbf{maximale geheugengebruik tijdens opbouwen}.
Ook dit is \textbf{drastisch gedaald}.
Dit is exact de reden dat we deze indexstuctuur gekozen hebben.
Wanneer we het resultaat voor Swiss-Prot extrapoleren naar UniProtKB, gebruikmakende van de assumptie dat UniProtKB ongeveer 500 keer groter is, dan is de verwachting dat er ongeveer 1.2 TB RAM nodig zal zijn.
Hierbij heeft de sparseness factor $k$ geen invloed op het maximale geheugenverbruik tijdens het opbouwen van de suffix array.
We moeten namelijk eerst de volledige suffix array bouwen om daarna te samplen hieruit.
Deze 1.2 TB blijft een grote hoeveelheid aan geheugen, maar is wel al beschikbaar op de huidige generatie aan servers.
Zo bieden zowel Amazon via AWS, google via GCP en Microsoft via Azure instanties aan met enkele TB aan RAM\@.

\begin{figure}[H]
    \centering
    \subfloat[Tijd nodig om de indexstructuur op te bouwen.]{\includegraphics[width=\linewidth]{building_array_libsais_time}}\\[4ex] % [4ex] om wat extra vertical spacing in te voegen

    \subfloat[Maximaal geheugengebruik om de indexstructuur op te bouwen.]{\includegraphics[width=\linewidth]{building_array_libsais_memory}}
    \caption{Vergelijking tussen de nodige tijd en hoeveelheid geheugen om een suffix array met libsais of een suffixboom in onze eigen Rust implementatie op te bouwen. De tijd en het geheugengebruik zijn gemeten met het Unix \texttt{time} commando. Als invoerbestand gebruiken we hier de Swiss-Prot of Human-Prot eiwitdatabank.}\label{fig:array_building}
\end{figure}

\subsection{Zoeken}\label{subsec:zoeken}
Aangezien er met de SA-indexstructuur geen voorberekening gebeurt van LCA's, is het enkel nuttig om de zoektijd inclusief het berekenen van de LCA te bekijken.
De tijd tot een match levert ons in dat geval nog niet alle nodige informatie op, zoals dit wel het geval was bij de suffixboom.
\\ \\
Figuur~\ref{fig:cutoff_humanprot}~(a) toont de zoekperformantie met suffix arrays.
Hier is onmiddellijk zichtbaar de de performantie voor het Human-Prot zoekbestand significant slechter is.
Na wat onderzoek bleek dat het berekenen van de LCA* erg traag werd indien er een groot aantal matches was.
Daarom is er besloten om een \textbf{drempelwaarde} (laten we deze B, van bovengrens, noemen) in te stellen.
Indien een peptide meer dan dan B matches heeft, wordt verondersteld dat de wortel de laagste gemeenschappelijke voorouder is van alle matches.
Uit eerder onderzoek uitgevoerd door het Unipept-team~\cite{unipept_cutoff} op de Unipept index die gebouwd was op basis van UniProt 2023\_03 bleek dat dit in de praktijk in de overgrote meerderheid van de gevallen ook effectief het geval is.
De resultaten hiervan kunnen teruggevonden worden in de tabellen in appendix~\ref{ch:appendix-unipept-protein-counts-distribution}.
Zo zijn er slechts $\pm$ 13\thinspace000 van de 1.3 miljard tryptische peptiden die meer dan 10\thinspace000 eiwitten matchen.
Hierbij was voor 95\% van de peptiden de LCA gelijk aan de wortel.
Slechts voor 200 peptiden was er een resultaat op soortniveau.
\\
\begin{figure}[H]
    \centering
    \subfloat[Bereken de LCA* voor alle matches. Voor de suffix arrays is gebruik gemaakt van sparseness factor $k = 1$.]{\includegraphics[width=0.7\linewidth]{no_cutoff_humanprot_search}}\\[4ex] % [4ex] om wat extra vertical spacing in te voegen

    \subfloat[Bereken de LCA* enkel als er minder dan 10\thinspace000 matches zijn.]{\includegraphics[width=0.7\linewidth]{cutoff_humanprot_search}}
    \caption{Berekenen van de LCA* (inclusief zoeken) voor alle peptiden (a) zonder en (b) met het gebruik van de drempelwaarde (B=10\thinspace000) op het Human-Prot en Swiss-Prot proteïnenbestand. De suffix array maakt gebruik van sparseness factor $k = 1$.}\label{fig:cutoff_humanprot}
\end{figure}

In Figuur~\ref{fig:cutoff_humanprot} is duidelijk te zien dat de \textbf{uitvoeringstijd drastisch daalt} wanneer de drempelwaarde op 10\thinspace000 matches geplaatst wordt.
Als we dit specifiek bekijken voor de Human-Prot peptidebestanden en eiwitdatabank is de uitvoeringstijd maar liefst 300 keer sneller.
Bovendien is ook hier het \textbf{informatieverlies minimaal}.
Van de 250\thinspace000 peptiden zijn er 12 peptiden die een ander resultaat verkrijgen in de output.
Deze 12 peptiden zijn echter slechts twee unieke peptiden (die gewoon meerdere keren voorkomen in het peptidebestand).
Dit zijn de peptiden \texttt{EKP} en \texttt{SKE}.
Indien we de LCA* effectief berekenen is het resultaat in beide gevallen 9606, terwijl we gebruikmakende van een drempelwaarde de root (1) terug geven.
\\ \\
Nu we het zoeken in de suffix array afgetopt hebben aan de hand van een drempelwaarde, is het interessant om te kijken hoe de zoekperformantie zich verhoudt ten opzichte van het zoeken in een suffixboom.
In de praktijk is het zoeken in een suffix array 5 tot 10 maal trager voor onze testbestanden.
Een deel van deze extra zoektijd kan opgevangen worden door gebruik te maken van parallel zoeken.
Indien we hier gebruik van maken, duurt het zoeken van 100\thinspace000 peptiden op de Human-Prot en Swiss-Prot databank opnieuw enkele tientallen tot honderden milliseconden.
Natuurlijk zouden we het zoeken in een suffixboom ook kunnen parallelliseren, maar dit hebben we niet geprobeerd omwille van de eerder vermelde geheugenproblemen bij suffixbomen.


    
    \chapter{Andere indexstructuren}\label{ch:andere-indices}
Naast suffixbomen en suffix arrays bestaan er ook nog enkele andere interessante indexstructuren.
We behandelen in dit hoofdstuk de FM-index en de R-index.
Deze maken achterliggend gebruik van suffix arrays, maar zorgen ervoor dat de resulterende index kleiner is.


\section{FM-index}\label{sec:fm-index}
De FM-index~\cite{fm_index} werd als eerste beschreven door Ferragina en Manzini.
Officieel staat de naam voor \textit{Full-text index in Minute space}.
Een FM-index bestaat uit \textbf{3 essentiële componenten}.
De \textbf{Borrows-wheeler transformatie}~\cite{bwt}, \textbf{een (sparse/compressed) suffix array (SA)} en \textbf{bitvectoren} die de rank-operatie ondersteunen.
Het opbouwen van de index voor tekst $T$ kan gebeuren in $O(n)$ tijd, met $n$ de lengte van $T$.
Het zoeken of er een match is, kan in $O(m)$ tijd, met $m$ de lengte van patroon $P$.
Het zoeken inclusief het vinden van waar alle matches in de originele string zitten, kan in $O(m + \text{|occ}(P, T)\text{|} \log n)$.
Hierbij is $\text{occ}(P, T)$ de set van alle indices waar patroon $P$ matcht in de tekst $T$.
In plaats van de originele tekst volledig bij te houden, wordt de BWT van de tekst bijgehouden.
Door de extra informatie die in de BWT verwerkt zit (ten opzichte van de tekst zelf), is het mogelijk om een \textbf{grotere sparseness factor $k$} toe te passen op de SA, zonder dat dit impact heeft op de minimale lengte van zoekbare peptiden.
Zo is het mogelijk om sparseness factor $k = 32$ te gebruiken, waarbij het nog steeds mogelijk is om peptiden die bestaan uit één aminozuur te zoeken, maar dit heeft natuurlijk een negatieve impact op de performantie.
De FM-index laat dus toe om een \textbf{kleinere index} te bekomen, door het gebruik van een hogere sparseness factor $k$ op de SA, zonder verlies van functionaliteit.
Tot slot kan men via een variant van de FM-index, de bidirectionele FM-index, en door gebruik te maken van zogenaamde \textbf{zoekschema's}, ook \textbf{inexacte matching} ondersteunen waarbij maximaal $x$ mismatches toegelaten zijn.

\subsection{Verschillende implementaties}\label{subsec:verschillende-implementaties}
Om een goed beeld te krijgen over het geheugengebruik tijdens het opbouwen, hebben we verschillende FM-index implementaties getest.
Opnieuw focussen we ons vooral op het geheugengebruik aangezien dit de primaire restrictie is tijdens het opbouwen.
Tabel~\ref{tab:fm_index_building} geef een overzicht van de geteste implementaties.
Omdat ons einddoel bestaat uit het indexeren van UniProtKB, kunnen we ons opnieuw enkel focussen op de 64-bit implementaties.
UniProtKB bevat namelijk meer tekens dan door de maximale 32-bit integer voorgesteld kan worden.
Als referentie maken we gebruik van de resultaten uit Tabel~\ref{tab:sa_building}.
Hieruit konden we concluderen dat er 1.86 GB RAM nodig is om een suffix array op te bouwen (met 64-bit integers).
Wanneer we dit vergelijken met de resultaten voor de geteste FM-index implementaties, zien we dat deze \textbf{bijna twee maal zo veel geheugen nodig hebben}.
Zelfs wanneer een sparseness factor geïntroduceerd wordt, zal dit steeds in een hoger geheugengebruik resulteren dan bij een suffix array.
De volledige suffix array is namelijk nodig om efficiënt de BWT af te leiden.
Bovendien moeten we op basis van deze BWT nog een bitvector bijhouden voor elk teken in ons alfabet, behalve voor het unieke eindteken.
Voor UniProtKB zou dit willen zeggen dat er al ongeveer $\frac{88 \cdot 27}{8} \approx 300$ GB RAM nodig hebben enkel hiervoor.
Bovendien moeten deze bitvectoren ook allemaal de rank operatie ondersteunen.
Dit kan efficiënt met ongeveer nog eens 25\% extra geheugen via het rank9 algoritme~\cite{CCB_course, rank9}.
Hierdoor loopt het extra geheugengebruik op tot ongeveer 375 GB\@.
Om die reden hebben we in het kader van deze masterproef deze optie niet verder verkend.

\begin{table}[H]
    \begin{minipage}{\linewidth}
        \centering
        \resizebox{\textwidth}{!}{
            \begin{tabular}{l l S[table-format=-2.2] S[table-format=-2.2] S[table-format=-1.2] S[table-format=-1.2]}
                Implementatie & Programmeertaal & \multicolumn{2}{c}{Tijd (in s)} & \multicolumn{2}{c}{Geheugen (in GB)} \\
                \hline\hline
                &                         & {32-bit} & {64-bit} & {32-bit} & {64-bit} \\
                \cline{3-6}
                \url{https://crates.io/crates/fm-index}     & Rust                    & {-}      & 33.92    & {-}      & 3.22     \\
                \url{https://github.com/simongog/sdsl-lite} & C++                     & {-}      & 27.74    & {-}      & 3.70     \\
                \url{https://github.com/ocfnash/FM-Index}   & Cython met C++ bindings & 57.94    & {-}      & 1.96     & {-}      \\
                \hline
            \end{tabular}
        }
        \caption{Uitvoeringstijd en maximaal geheugengebruik voor het opbouwen van een FM-index voor de Swiss-Prot proteïnedatabank aan de hand van verschillende implementaties.
        Afhankelijk van de gebruikte implementatie (32- of 64-bit) is een andere kolom ingevuld.
        Een - staat voor niet getest. Deze testen werden lokaal uitgevoerd op een M1 Pro MacBook Pro.
        De specificaties hiervan zijn terug te vinden in tabel~\ref{tab:macbook_hardware}.}
        \label{tab:fm_index_building}
    \end{minipage}
\end{table}


\section{R-index}\label{sec:r-index}
R-indices~\cite{r_index2, r_index1} zijn een verdere evolutie van de FM-index waarbij \textbf{run-length encoding (RLE)} toegepast wordt op de \textbf{BWT van de tekst}.
De index heeft grootte $O(r)$ met $r$ het aantal BWT runs van de tekst van lengte $n$.
Vanwege het gebruik van RLE op de BWT, zal de \textbf{resulterende index kleiner} zijn naarmate er \textbf{meer herhaling} voorkomt in de geïndexeerde tekst.
Om na te gaan wat het effect hiervan is op onze proteïnedatabanken, hebben we de R-index getest op de eerste 0.5\%, 1\%,\ldots\space van de volledige UniProtKB databank.
Figuur~\ref{fig:sa_vs_r_index} visualiseert het geheugengebruik en de resulterende indexgrootte in vergelijking met suffix arrays.
Merk op dat het geheugengebruik tijdens het opbouwen en de grootte van de resulterende R-index meer dan verdubbelt bij het overgaan van 2\% (1\thinspace651\thinspace521\thinspace046 tekens) naar 4\% (3\thinspace319\thinspace904\thinspace170 tekens) van UniProtKB\@.
De oorzaak hiervan is het overschakelen van 32-bit naar 64-bit integers binnen de R-index.
Bij de indices tot en met 2\% van UniProtKB maakt de R-index implementatie gebruik van 32-bit integers omdat er minder tekens in de geïndexeerde tekst staan dan de maximale 32-bit integer waarde.
Bij grotere databanken wordt deze 32-bit integer limiet overschreden, waardoor de R-index implementatie moet overschakelen naar het gebruik van 64-bit integers.
\\ \\
Enerzijds valt duidelijk te zien dat het opbouwen van een \textbf{suffix array minder geheugen vraagt}.
Anderzijds is de \textbf{resulterende R-index kleiner}, en wordt deze bovendien procentueel kleiner voor grotere databanken.
Dit is de winst die verkregen wordt door het gebruik van de run-length encoding op de BWT\@ van de tekst.
Natuurlijk zouden we bij de suffix array ook vrij simpel de resulterende index kunnen verkleinen door de SA sparse te maken.
Hierbij zal op een gegeven punt de winst in indexgrootte erg klein zijn omdat de volledige tekst altijd opgeslagen moet worden.
Bij het gebruik van sparseness factor $k = 3$ voor de grootste geteste index zou de totale index voor de suffix array teruggedrongen worden tot 15.31 GB\@.
Van deze 15.31 GB is 4.17 GB de tekst zelf.
Merk op dat de grootte van de resulterende index voor suffix arrays dan nog steeds lineair zal stijgen als de databank groeit, maar met een kleinere constante.
Aangezien de \textbf{resulterende indexgrootte voor R-indices sublineair stijgt}, zullen we de sparseness factor $k$ van suffix arrays groter en groter moeten maken om een even kleine resulterende index te hebben.
Dit maakt een R-index op dit vlak een interessante indexstructuur voor grote proteïnedatabanken.
\\ \\
Opnieuw is het geheugengebruik tijdens het opbouwen hier een sterk beperkende factor.
Dit geheugengebruik is meer dan twee keer zo groot als bij het bouwen van een suffix array, waardoor we deze optie niet verder zullen verkennen.
Bovendien kunnen we door het gebruik van een hogere sparseness factor de resulterende suffix array toch snel twee tot drie maal kleiner maken, zonder al te veel performantieverlies.

\begin{figure}[H]
    \centering
    \subfloat[Maximaal geheugengebruik tijdens het opbouwen van een suffix array en R-index.]{\includegraphics[width=0.8\linewidth]{sa_vs_r_index_build_memory}}\\[4ex] % [4ex] om wat extra vertical spacing in te voegen

    \subfloat[Resulterende indexgrootte voor een suffix array (sparseness factor $k = 1$) en R-index.]{\includegraphics[width=0.8\linewidth]{sa_vs_r_index_index_memory}}
    \caption{Vergelijking van de suffix array en R-index op vlak van geheugengebruik en de resulterende indexgrootte voor verschillende deelverzamelingen van UniProtKB. Bij elke test gaat het om de eerst $x$ procent van de database. Bij de suffix array wordt sparseness factor $k = 1$ gebruikt. Het gebruik van een andere sparseness factor heeft enkel invloed op de grootte van de resulterende index.}\label{fig:sa_vs_r_index}
\end{figure}



    
    \chapter{Een nieuwe UniProtKB-index voor Unipept}\label{ch:een-nieuwe-uniprotkb-index-voor-unipept}
In de vorige hoofdstukken hebben we suffixbomen, suffix arrays, FM-indices en R-indices verkend als indexstructuren die toegepast kunnen worden om een eiwitdatabank te indexeren.
Uit die hoofdstukken bleek namelijk dat een suffix array de laagste geheugenvereisten heeft.
In dit hoofdstuk gaan we dieper in op het opbouwen van een index voor UniProtKB, en het in productie brengen ervan.
Aangezien we de huidige Unipept index willen vervangen door deze nieuwe indexstructuur behandelen we bovendien ook nog enkele extra gewenste features.

\section{Opbouwen van de SA}\label{sec:opbouwen-van-de-sa}
Zoals eerder vermeld levert een ruwe extrapolatie op dat we $\pm$ 1.2 TB aan geheugen nodig zou hebben om een index voor UniProtKB op te bouwen.
Hierbij werd er echter van uitgegaan dat UniProtKB 500\times meer eiwitten dan Swiss-Prot bevat.
Dit is een overschatting van de realiteit waar UniProt op dit moment \textit{slechts} $\pm$ 440 keer groter is.
Dit zorgt ervoor dat het opbouwen mogelijks al \textbf{haalbaar is op HPC van UGent} waar nodes beschikbaar zijn die ongeveer 940 GB aan beschikbaar geheugen hebben.
Na dit te proberen bleek \textit{slechts} 740 GB RAM nodig.
Figuur~\ref{fig:build_uniprot} toont de nodige tijd en hoeveelheid geheugen om dit te realiseren.
\textbf{Opvallend hierbij is dat de libsais implementatie hier trager is dan libdivsufsort}, terwijl dit voor alle kleinere datasets net omgekeerd was.
Afhankelijk van de dataset is het ene algoritme dus sneller dan het andere.
Het geheugengebruik van beide algoritmen blijft echter erg gelijkaardig, wat het belangrijkste is voor ons.
\\
\begin{figure}[H]
    \centering
    \subfloat[Tijd nodig om een SA-index voor UniProtKB te bouwen.]{\includegraphics[width=\linewidth]{build_uniprot_time}}\\[4ex] % [4ex] om wat extra vertical spacing in te voegen

    \subfloat[Maximale hoeveelheid geheugen gebruikt tijdens het opbouwen van een SA-index voor UniProtKB.]{\includegraphics[width=\linewidth]{build_uniprot_memory}}
    \caption{Statistieken voor het opbouwen van een SA voor UniProtKB}\label{fig:build_uniprot}
\end{figure}

\section{Een sparseness factor kiezen}
Een volgende stap na het opbouwen van de volledige SA was het beslissen van de sparseness factor.
Zoals eerder aangegeven in de conclusie van sectie~\ref{subsec:zoeken-in-sparse-suffix-arrays} willen we deze sparseness factor zo laag mogelijk kiezen, met als restrictie dat de SSA nog steeds in het geheugen moet kunnen gehouden worden.
De beschikbare machines die UniPept hosten hebben elk $pm$ 0.5 TB RAM ter beschikking.
Dit wil zeggen dat we de resulterende index hierin moeten krijgen, en ook nog genoeg overhead moeten laten zodat de machine zeker niet out-of-memory gaat tijdens het hosten van de index en het verwerken van meerdere requests.
Praktisch gezien komt dit er op neer dat we gebruikmaken van \textbf{sparseness factor 3}.
Dit resulteert in een SA van 231.81 GB\@.
In combinatie met de tekst (86.93 GB) resulteert dit in een \textbf{totale indexgrootte van 318.74 GB}, wat comfortabel in de 0.5 TB past.
Hierbij moet natuurlijk nog wat overhead gerekend worden om de mapping van de suffix naar proteïne bij te houden.
Bovendien bevatten deze servers ook nog andere databanken om verdere aggregaties te berekenen.
Zo biedt Unipept naast een taxonomische analyse ook nog een functionele analyse aan.
Deze functionele analyse wordt uitgevoerd op basis van de gematchte eiwitten die de indexstructuur terug geeft.
Indien we sparseness factor 2 zouden gebruiken komt de totale indexgrootte uit op $\pm$ 450 GB\@.
Dit op zich zou nog net in de machine passen, maar laat niet genoeg ruimte voor de andere processen.

\section{Isoleucine en Leucine gelijkstellen}\label{sec:isoleucine-en-leucine-equivalentie}
Naast het vinden van exacte matches is het vinden van inexacte matches ook interessant.
Vooral het vinden van matches waarbij we een I (Isoleucine) en L (Leucine) aan elkaar gelijkstellen is een belangrijke optie in de huidige Unipept index.
\textbf{Deze twee aminozuren kunnen niet uit elkaar gehaald worden door een massaspectrometer vanwege hun identieke massa}.
Door deze restrictie van de huidige hardware is het voor onderzoekers erg nuttig om alle matches te vinden waar een I ook een L kan zijn of omgekeerd.
Om deze vorm van inexacte matching toe te voegen aan de index gebruikmakende van een suffix array zijn meerdere opties.
\begin{enumerate}
    \item \textbf{Twee indices:} Bouw een extra index waarbij in de tekst elke L ook door een I vervangen wordt.
    Afhankelijk van als de gebruiker I gelijk wil stellen aan L, wordt daarna de request afgehandeld door de correcte index.
    Hierbij moeten we enkel dezelfde vervangoperatie uitvoeren als bij de index indien I gelijk gesteld wordt aan L\@.
    Deze optie wordt niet verder uitgewerkt omdat we niet alleen twee indices zouden moeten opbouwen, ook de nodige hoeveelheid geheugen om de index te hosten zou verdubbelen.
    Dit is iets wat we willen vermijden.
    \item \textbf{Index waarbij I $\neq$ L:} Genereer de varianten van de gezochte peptide \textit{on the fly} tijdens het zoekproces.
    Hierbij kan gebruikgemaakt worden van het feit dat 2 peptiden die identiek zijn, behalve dat hun I's en L'en omgewisseld kunnen worden, hetzelfde zoekpad afleggen in de SA, behalve wanneer het teken dat op dat moment verwerkt wordt een I, J, K of L is.
    Dit zorgt ervoor dat voor een willekeurig patroon een groot stuk van de zoekboom gemeenschappelijk zal zijn.
    Bovendien zullen we tijdens het zoekproces een tak erg snel kunnen snoeien.
    Dit is mogelijk wanneer op een bepaalde positie enkel een I of L beschikbaar is.
    \item \textbf{Index waarbij I = L:} Bouw de index voor een variant van de UniProtKB database.
    Hierbij vervangen we elke I door een L, of omgekeerd, en bouwen we dus een index op waarbij I gelijkgesteld is aan L\@.
    Tijdens het zoeken van een peptide doen we dezelfde vervanging, waarna we alle matches hebben waarbij I gelijkgesteld is aan L\@.
    Indien we I niet gelijk wouden stellen aan L, kunnen we achteraf uit deze lijst per peptide de I en L locaties controleren aan de hand van de originele tekst, en de foute matches weg filteren.
\end{enumerate}

In de volgende secties worden de tweede en derde aanpak verder uitgewerkt.

\subsection{Index waarbij I $\neq$ L}\label{subsec:index-waarbij-i-neq-l}
Wanneer de indexstructuur zelf gebouwd is met I niet gelijkgesteld aan L, moeten we alle verschillende IL-combinaties tijdens het zoekproces verkennen.
Hierbij is het echter belangrijk om te weten dat er ook bepaalde extreem slechte gevallen bestaan waarbij de zoekruimte erg groot wordt, en we niet efficiënt kunnen snoeien.
De sequenties waarop we zoeken zijn namelijk niet random verdeeld over alle karakters van het alfabet.
Biologisch gezien komen bepaalde aminozuursequenties veel vaker voor.
Eén van deze patronen zijn de zogenaamde \textit{Leucine rich repeats}~\cite{leucine_rich_repeats}.
Dit zijn sequenties waarin een reeks L'en na elkaar voorkomt.
In UniProtKB bestaat er een \textbf{sequentie waar maar liefst 2397 L's na elkaar voorkomen}.
Dit is de proteïne met \textit{accession number} \texttt{A0A1Q9EZQ0}.
Wanneer we nu ook weten dat in UniProtKB ook reeksen aan I's voorkomen zorgt dit voor bepaalde extreem slechte gevallen.
Zo is hier \textbf{ook een proteïne met 641 opeenvolgende I's} (accession number: \texttt{A0A5J4P3H7}).
In het slechtste geval zou een gebruiker dus een sequentie van 641 I's of L'en kunnen proberen te matchen.
Dit zorgt ervoor dat we in de zoekboom $2^{641} \approx 9.12^{192}$ opties moeten proberen.
Het grootste deel van deze opties zal niet voorkomen, waardoor de takken van deze boom allemaal extreem kort zullen zijn.
Dit zal echter verwaarloosbaar zijn ten opzichte van het gigantisch aantal opties.
Om dit in perspectief te plaatsen: Men schat dat er in het totaal $10^{79}$ atomen in het universum zijn~\cite{atoms_in_universe} en dat het universum ongeveer $4.36^{20}$ milliseconden oud is~\cite{age_universe}.
Zelfs als het controleren van één optie minder dan een milliseconde duurt, dan zou dit dus nog onmogelijk zijn.
\textbf{Om de zoektijd en het geheugengebruik te beperken, hebben we ervoor gekozen om twee vormen van restricties op te leggen wanneer I en L gelijkgesteld worden}.
\begin{enumerate}
    \item Laat maximaal 5 seconden aan zoektijd per peptide toe.
    \item Laat per peptide in totaal maximaal 34 I's en L'en toe.
\end{enumerate}
Deze twee restricties samen moeten de servers deels helpen beschermen tegen Denial of Service attacks.
In de praktijk wordt normaal de tijdslimiet van 5 seconden eerst bereikt.
Dit vertaalt zich naar een sequentie van ongeveer 25 I's of L'en.
Elke keer we één extra teken zouden willen toelaten verdubbelt de zoektijd.
Zo duurt het al één minuut om een sequentie met 30 opeenvolgende I's of L'en te zoeken.
Wanneer we echter nog veel meer I's of L'en toelaten stuiten we vrij snel op een geheugenlimiet.
Om te voorkomen dat het programma meer geheugen probeert te vragen dan dat de server heeft, waarna het crasht, hebben we beslist maximaal 34 I's of L'en toe te laten.
In het slechtste geval gebruikt één enkele thread hierbij iets meer dan 2 GB RAM\@.
Aangezien het zoeken multithreaded is, moeten we dus een goede 20 GB aan vrij geheugen voorzien wanneer de index ingeladen is.
\textbf{Wanneer we deze limieten testen door alle testpeptidebestanden te zoeken in de volledige UniProtKB databank worden deze limieten geen enkele keer bereikt}.
\\ \\
\textbf{In de praktijk} vertaalt het gelijkstellen van I en L op deze manier zich naar een \textbf{vrij kleine overhead}.
Deze beperkte overhead valt te zien in Figuur~\ref{fig:uniprot_search}.
In deze figuur zien we bovendien dat de zoeksnelheid waarbij I niet gelijk wordt gesteld aan L erg goed is.
% TODO: schrijf hier iets bij in vergelijking met de huidige unipept index qua snelheid

\begin{figure}[ht]
    \centering
    \includegraphics[width=0.95\linewidth]{uniprot_searchtime_standard_vs_il_equality}
    \caption{Zoektijd in UniProtKB (Swiss-Prot + TrEMBL) met en zonder het gelijkstellen van I en L met een SA met sparseness factor $k = 3$.}
    \label{fig:uniprot_search}
\end{figure}

\subsection{Index waarbij I = L}
% TODO

\section{Vergelijking met andere tools}\label{subsec:vergelijking-met-andere-tools}
Aangezien we er in geslaagd zijn om UniProtKB volledig te indexeren is het nuttig om de performantie te vergelijken met bestaande tools.
De tools die we bekijken zijn de \textbf{Uniprot Peptide search tool}, de \textbf{Expasy ScanProsite tool} en \textbf{Unipept} (met de bestaande indexstructuur).
Vanwege de performantie zullen we ons beperken tot het opzoeken van één willekeurige peptide \texttt{ISPAVLFVIVILAVLFFISGLLHLLVR}.
Als referentie gebruiken we de zoektijd in de nieuwe indexstructuur gebruikmakende van een suffix array met sparseness factor 3.
Hierin duurt het zoeken van de peptide \textbf{0.01 seconden} en levert 50 matches op.
Wanneer we I en L gelijkstellen duurt dit opnieuw 0.01 seconden, maar levert dit wel 52 matches op.
Hierbij wordt natuurlijk geen extra tijd geïntroduceerd door een API call over het internet aangezien deze index (nog) niet publiek beschikbaar is.

\subsection{UniProt peptide search tool}
De UniProt peptide search tool\cite{uniprot_search_paper, uniprot_search_site} is een onderdeel van de UniProt site en laat toe om alle voorkomens van een bepaald eiwit te vinden in de volledige UniProtKB database.
Als opties kunnen we kiezen om enkel in Swiss-Prot te zoeken, of in Swiss-Prot en TrEMBL.
Voor beide datasets kunnen we kiezen of we I en L aan elkaar willen gelijkstellen.
Deze features komen exact overeen met de eigenschappen van onze nieuwe indexstructuur.
Qua performantie is deze tool echter extreem onstabiel.
Zo duurt het zoeken van \texttt{ISPAVLFVIVILAVLFFISGLLHLLVR} in de volledige UniProtKB database \textbf{enkele seconden tot zelfs 20 minuten}.
Bovendien vermoeden we ook dat er een vorm van caching toegepast wordt aangezien het zoeken voor een tweede keer zo goed als onmiddellijk het resultaat geeft.
Dit maakt het benchmarken extreem moeilijk.
Zoals verwacht zijn de gevonden matches identiek aan de matches die onze nieuwe indexstructuur vindt.
Naast de variabele performantie faalt deze tool ook regelmatig tijdens het berekenen en ophalen van de resultaten.

\subsection{Expasy ScanProsite tool}
De Expasy ScanProsite tool\cite{scanprosite} laat toe om aan de hand van een taaltje die lijkt op reguliere expressies allerlei patronen te zoeken.
Hun eigen taal laat toe om onder andere wildcards, karakterklasses, inverse klasses en herhalingen te specificeren.
De flexibiliteit van dit systeem is dus een sterk voordeel.
Een nadeel van deze tool is dat niet de volledige UniProtKB databank doorzocht wordt.
Er wordt enkel rekening gehouden met proteïnes die deel uit maken van een reference proteome\footnote{Deze zogenaamde reference proteomes zijn een collectie van eiwitten die door een bepaald organisme gemaakt kunnen worden, en die bovendien taxonomisch belangrijk bevonden worden. Deze laatste voorwaarde wil dus zeggen dat dit niet zomaar van elk organisme is, enkel van een selecte groep organismes die op basis van verschillende factoren belangrijk bevonden wordt. Een voorbeeld hiervan is de \textit{human reference proteome}. Dit zijn alle eiwitten die door een mens aangemaakt kunnen worden.}.
Dit komt er op neer dat ze bij UniProt 2024\_01 \textbf{slechts rekening houden met 85\thinspace152\thinspace388 van de 249\thinspace751\thinspace891 proteïnes}.
Wanneer we de zoekperformantie hier testen duurt het zoeken van de ene peptide zo'n \textbf{5.5 minuten} zonder gelijkstellen van I en L en 15 minuten met het gelijkstellen van I en L\@.
Hierbij zijn er slechts 30 resp. 32 matches gevonden, wegens de subset van proteïnes die gebruikt wordt.

\subsection{Huidige Unipept index}
De bestaande Unipept index laat toe om, \textbf{op voorwaarde dat je peptide tryptisch is}, alle matches te vinden met de keuze om I en L gelijk te stellen.
De peptide die we hier gebruiken is dit ook net, dus zou deze index hiervoor dezelfde matches moeten vinden.
Dit is inderdaad het geval, en gebeurt bovendien gebruikmakende van een API call in \textbf{0.1 seconden}, wat duidelijk extreem veel sneller is dan de andere bestaande opties.
Opnieuw is de zoektijd hier al zo klein dat de invloed van het gelijkstellen van I en L niet te meten valt.
Dat dit enkel werkt voor tryptische peptiden is een extreem grote beperking, volgend voorbeeld illustreert dit.
Stel dat we nu de peptide \texttt{ILAKLFIS} zoeken.
Dit is geen tryptische peptide en bijgevolg kunnen er geen matches gevonden worden.
De nieuwe indexstructuur vindt echter 14 matches zonder I en L gelijk te stellen, en 207 matches bij het gelijkstellen van I en L.


\section{Functionele analyse}
% TODO

\section{Aanbieden van de nieuwe indexstructuur}\label{sec:aanbieden-van-de-nieuwe-indexstructuur}
Alle eerder vermelde benchmarks tonen enkel de zoektijd.
Bij het opstarten moeten we echter \textbf{eerst de indexstructuur inladen}.
Dit alleen duurt 20 tot 25 minuten.
We willen dit inladen slechts eenmalig doen om dan alle requests onmiddellijk af te kunnen afhandelen.
Dit doen we aan de hand van een simpele webserver aan de hand van de Axum crate\cite{axum}.
Deze webserver laadt bij het opstarten de indexstructuur in en blijft daarna wachten op HTTP requests die een JSON bevatten met daarin de peptiden die we willen zoeken.
Op deze manier is dit probleem elegant opgelost, en kunnen we bovendien aan de hand van JSON bestanden erg makkelijk de input verwerken, en de resultaten terugsturen.



    \chapter{Conclusie \& future work}\label{ch:conclusie}
In deze masterproef hebben we verschillende opties verkend om de huidige Unipept index voor UniProtKB te vervangen.
Hierbij lag de hoofdfocus op het vinden van een nieuwe indexstructuur die aan volgende opties voldeed.
\begin{enumerate}
    \item De index moet het mogelijk maken om arbitraire peptiden te kunnen zoeken.
    \item Het geheugenverbruik van de index moet beperkt zijn, zodat het mogelijk is niet enkel kleinere proteïnedatabanken te indexeren, maar ook de volledige UniProtKB database\@.
    \item De indexstructuur moet semi-exacte matching ondersteunen, zodat I en L aan elkaar gelijkgesteld kunnen worden.
\end{enumerate}

\section{Conclusie}
Een eerste indexstructuur die we bekekenen waren \textbf{suffixbomen}.
Hierbij hebben we een eigen Rust implementatie gemaakt van het algoritme van Ukkonen.
Deze suffixbomen bieden een extreem grote vrijheid, en het zoeken gaat bovendien extreem snel.
Ze zijn echter \textbf{onbruikbaar} voor grote proteïnedatabanken te indexeren vanwege het \textbf{geheugengebruik}.
\\ \\
Een volgende optie die we verkend hebben zijn \textbf{suffix arrays}.
Na testen bleek dat deze datastructuur ons een goeie \textbf{balans gaf tussen snelheid en geheugenverbruik}.
Aan de hand hiervan zijn we er in geslaagd een nieuwe indexstructuur voor UniProtKB te ontwikkelen die de huidige Unipept index volledig kan vervangen, en alle bovenstaande doelstellingen haalt.
Aan de hand van een aangepast zoeksysteem ondersteunen we zowel exacte als semi-exacte matching, waarbij efficiënt alle informatie van de gematchte peptides teruggegeven kan worden.
Het grootste nadeel van deze manier van werken is dat de functionele en taxonomische analyses van Unipept \textit{on the fly} moeten gebeuren, wat een beperkte, negatieve impact heeft op de performantie.
\\ \\
Als derde en vierde indexstructuur hebben we kort de \textbf{FM- en R-index} getest.
Na enkele testen bleek snel dat het \textbf{geheugengebruik} tijdens het bouwen hiervan \textbf{dubbel zo hoog lag} als bij suffix arrays.
Omwille hiervan hebben we deze opties niet verder uitgewerkt, ook al hebben beide indices interessante eigenschappen voor ons probleem.
Zo laten bidirectionele FM-indices toe om algemenere inexacte matching uit te voeren, en is de resulterende index bij R-indices kleiner door het gebruik van run-length encoding.

\section{Future work}
Ondanks dat we een index gevonden hebben die de huidige Unipept index kan vervangen, en willekeurige peptiden kan vinden in plaats van enkel tryptische peptiden, zijn er meerdere plaatsen waar nog extra onderzoek naar kan gebeuren.
\\ \\
Zo zou het extreem interessant zijn om een manier te vinden om de suffix array onmiddellijk \textbf{sparse te maken tijdens het opbouwen}.
Hierbij zal het bovendien belangrijk zijn dat deze implementatie een \textbf{lage constante factor heeft op vlak van geheugengebruik}.
Indien deze constante vrij groot is, zal de winst verloren gaan ten opzichte van de huidige implementatie, aangezien we altijd een kleine sparseness factor $k$ gebruiken.
Een andere aanpak zou kunnen zijn via een online algoritme waarbij we een bestaande SSA kunnen uitbreiden met nieuwe eiwitten.
Zo kunnen we tijdens het opbouwen één eiwit per keer toevoegen aan de suffix array, maar kunnen we ook nieuwere UniProtKB releases efficiënt afhandelen door te vertrekken van de index uit de vorige release.
\\ \\
Een tweede duidelijk punt van verbetering, is tijdens het \textbf{berekenen van de LCA*}.
Indien deze voorberekend wordt tijdens het opbouwen van de index zal dit niet enkel de performantie ten goede komen, het zal ook de nood voor de drempelwaarde B (die standaard op 10\thinspace000 staat) sterk verminderen.
De meest duidelijke piste hiervoor is aan de hand van Enhanced Suffix Arrays (ESAs).
In deze masterproef hebben we deze piste niet verder verkend vanwege de eerste indicatie dat het berekenen van de extra tabellen het geheugenverbruik zou verdubbelen, en dat het \textit{on the fly} berekenen van de LCA* een acceptabele overhead met zich mee brengt.
\\ \\
Een derde mogelijke route, is het \textbf{uitbreiden van de inexacte matching}.
Op dit moment is dit slechts in een extreem beperkte vorm aanwezig.
We hebben namelijk enkel de optie om I en L aan elkaar gelijk te stellen.
Tools zoals de eerder vermelde Expasy ScanProsite tool ondersteunen verschillende manieren om inexacte matching uit te voeren.
Zo zouden we ondersteuning voor karakterklassen, een reeks van herhalingen (al dan niet met een minimum en maximum bereik) en wildcards kunnen toevoegen.
Een andere manier van inexacte matching kan zijn via het toelaten van maximaal $x$ mismatches.
Op deze manier kan men ook omgaan met kleine mutaties die ontstaan in eiwitten, of fouten die ontstaan tijdens het inlezen via een massaspectrometer.
Over deze laatste vorm van inexacte matching is door het team van Unipept een vervolg masterproef uitgeschreven om dit volgend jaar te verkennen.
Hierbij zal verder ingegaan worden op FM- en R-indices.

% =====================================================================
% End matter
% =====================================================================

% ------------ REFERENCES ------------
    \printbibliography[heading=bibintoc,title={Referenties}] % check if bibliography is in table of contents


% ------------ APPENDIX ------------
    \appendix
    \chapter{Statistieken Zoekbestanden}\label{ch:appendix-statistieken-zoekbestanden}
Deze appendix bevat extra grafieken die niet relevant waren voor de tekst, maar toch extra inzicht kunnen geven in de consistentie van de zoekbestanden.
Deze grafieken visualiseren de verdeling van aminozuren en de lengte van de peptides.

\section{Human-Prot}\label{sec:human-prot-stats}
\begin{figure}[H]
    \centering
    \subfloat[Distributie van de aminozuren in het Human-Prot zoekbestand.]{\includegraphics[width=0.485\linewidth]{humanprot_search_amino_acids}}
    \hfill
    \subfloat[Lengtedistributie van de peptiden in het Human-Prot zoekbestand.]{\includegraphics[width=0.485\linewidth]{humanprot_search_lengths}}
    \caption{Extra statistieken over het Human-Prot zoekbestand}\label{fig:humanprot_search_other_stats}
\end{figure}

\section{Swiss-Prot}\label{sec:swiss-prot-stats}
\begin{figure}[H]
    \centering
    \subfloat[Distributie van de aminozuren in het Swiss-Prot zoekbestand zonder \textit{missed cleavage}.]{\includegraphics[width=0.485\linewidth]{swissprot_searchfile_no_missed_cleavage_amino_acids}}
    \hfill
    \subfloat[Lengtedistributie van de peptiden in het Swiss-Prot zoekbestand zonder \textit{missed cleavage}.]{\includegraphics[width=0.485\linewidth]{swissprot_searchfile_no_missed_cleavage_lengths}}
    \caption{Extra statistieken over het Swiss-Prot zoekbestand zonder \textit{missed cleavage}}\label{fig:swissprot_search_no_missed_cleavage_other_stats}
\end{figure}
\begin{figure}[H]
    \centering
    \subfloat[Distributie van de aminozuren in het Swiss-Prot zoekbestand met \textit{missed cleavage}.]{\includegraphics[width=0.485\linewidth]{swissprot_searchfile_missed_cleavage_amino_acids}}
    \hfill
    \subfloat[Lengtedistributie van de peptiden in het Swiss-Prot zoekbestand met \textit{missed cleavage}.]{\includegraphics[width=0.485\linewidth]{swissprot_searchfile_missed_cleavage_lengths}}
    \caption{Extra statistieken over het Swiss-Prot zoekbestand met \textit{missed cleavage}}\label{fig:swissprot_search_missed_cleavage_other_stats}
\end{figure}

\section{SIHUMI}\label{sec:sihumi-stats}
\begin{figure}[H]
    \centering
    \subfloat[Distributie van de aminozuren in het SIHUMI S03 zoekbestand.]{\includegraphics[width=0.485\linewidth]{sihumi_03_amino_acids}}
    \hfill
    \subfloat[Lengtedistributie van de peptiden in het in het SIHUMI S03 zoekbestand.]{\includegraphics[width=0.485\linewidth]{sihumi_03_length}}
    \caption{Extra statistieken over het SIHUMI S03 zoekbestand}\label{fig:sihumi_03_other_stats}
\end{figure}
\begin{figure}[H]
    \centering
    \subfloat[Distributie van de aminozuren in het SIHUMI S05 zoekbestand.]{\includegraphics[width=0.485\linewidth]{sihumi_05_amino_acids}}
    \hfill
    \subfloat[Lengtedistributie van de peptiden in het in het SIHUMI S05 zoekbestand.]{\includegraphics[width=0.485\linewidth]{sihumi_05_length}}
    \caption{Extra statistieken over het SIHUMI S05 zoekbestand}\label{fig:sihumi_05_other_stats}
\end{figure}
\begin{figure}[H]
    \centering
    \subfloat[Distributie van de aminozuren in het SIHUMI S07 zoekbestand.]{\includegraphics[width=0.485\linewidth]{sihumi_07_amino_acids}}
    \hfill
    \subfloat[Lengtedistributie van de peptiden in het in het SIHUMI S07 zoekbestand.]{\includegraphics[width=0.485\linewidth]{sihumi_07_length}}
    \caption{Extra statistieken over het SIHUMI S07 zoekbestand}\label{fig:sihumi_07_other_stats}
\end{figure}
\begin{figure}[H]
    \centering
    \subfloat[Distributie van de aminozuren in het SIHUMI S08 zoekbestand.]{\includegraphics[width=0.485\linewidth]{sihumi_08_amino_acids}}
    \hfill
    \subfloat[Lengtedistributie van de peptiden in het in het SIHUMI S08 zoekbestand.]{\includegraphics[width=0.485\linewidth]{sihumi_08_length}}
    \caption{Extra statistieken over het SIHUMI S08 zoekbestand}\label{fig:sihumi_08_other_stats}
\end{figure}
\begin{figure}[H]
    \centering
    \subfloat[Distributie van de aminozuren in het SIHUMI S11 zoekbestand.]{\includegraphics[width=0.485\linewidth]{sihumi_11_amino_acids}}
    \hfill
    \subfloat[Lengtedistributie van de peptiden in het in het SIHUMI S11 zoekbestand.]{\includegraphics[width=0.485\linewidth]{sihumi_11_length}}
    \caption{Extra statistieken over het SIHUMI S11 zoekbestand}\label{fig:sihumi_11_other_stats}
\end{figure}
\begin{figure}[H]
    \centering
    \subfloat[Distributie van de aminozuren in het SIHUMI S14 zoekbestand.]{\includegraphics[width=0.485\linewidth]{sihumi_14_amino_acids}}
    \hfill
    \subfloat[Lengtedistributie van de peptiden in het in het SIHUMI S14 zoekbestand.]{\includegraphics[width=0.485\linewidth]{sihumi_14_length}}
    \caption{Extra statistieken over het SIHUMI S14 zoekbestand}\label{fig:sihumi_14_other_stats}
\end{figure}
    \chapter{Unipept protein counts distribution}\label{ch:appendix-unipept-protein-counts-distribution}
\begin{table}[h!]
    \centering
    \begin{tabular}{|l|r|}
        \hline
        \textbf{\# Proteïnes} & \textbf{\# Peptides}\\
        \hline
        $\geq 1$     & 1\thinspace342\thinspace470\thinspace764 \\
        $\geq 2$     & 355\thinspace979\thinspace324            \\
        $\geq 10$    & 38\thinspace697\thinspace210             \\
        $\geq 10^2$  & 2\thinspace921\thinspace879              \\
        $\geq 10^3$  & 217\thinspace922                         \\
        $\geq 10^4$  & 13\thinspace008                          \\
        $\geq 10^5$  & 118                                      \\
        $\geq 10^6$  & 0                                        \\ \hline
    \end{tabular}
    \caption{Het aantal verschillende tryptische peptiden die matchen met op zijn minst $x$ proteïnes uit UniProtKB 2023\_03. Een klein voorbeeld: stel dat we de tryptiche peptide \texttt{ACACA} zoeken. Deze heeft 24\thinspace694 matches in UniProtKB. Dit wil zeggen dat dit één van de 13\thinspace008 peptiden is met meer dan $10^4$ matches.}
    \label{tab:number_peptide_matches}
\end{table}

\begin{table}[h!]
    \centering
    \begin{tabular}{|l|r|}
        \hline
        \textbf{\# NCBI taxonomy rank} & \textbf{\# Peptides} \\ \hline
        root & 12\thinspace369 \\ \hline
        superkingdom & 43 \\ \hline
        kingdom & 16 \\ \hline
        subkingdom & 0 \\ \hline
        superphylum & 0 \\ \hline
        phylum & 8 \\ \hline
        subphylum & 7 \\ \hline
        superclass & 1 \\ \hline
        class & 18 \\ \hline
        subclass & 1 \\ \hline
        superorder & 0 \\ \hline
        order & 0 \\ \hline
        infraorder & 1 \\ \hline
        superfamily & 0 \\ \hline
        family & 2 \\ \hline
        subfamily & 0 \\ \hline
        tribe & 1 \\ \hline
        subtribe & 0 \\ \hline
        genus & 55 \\ \hline
        subgenus & 0 \\ \hline
        species\_group & 0 \\ \hline
        species\_subgroup & 0 \\ \hline
        species & 200 \\ \hline
        subspecies & 0 \\ \hline
        strain & 1 \\ \hline
        varietas & 0 \\ \hline
        forma & 0 \\ \hline
    \end{tabular}
    \caption{Verdeling van de 13\thinspace000 verschillende peptiden die met meer $\geq 10^4$ proteïnes matchen. We zien dat voor de overgrote meerderheid de LCA resulteert op de root. Slechts voor 200 peptides is het resultaat op soortniveau.}
    \label{tab:peptides_distribution}
\end{table}

\begin{table}[h!]
    \centering
    \begin{tabular}{|r|l|}
        \hline
        \textbf{\# Peptides} & \textbf{LCA} \\ \hline
        119 & Alphainfluenzavirus influenzae \\ \hline
        32 & Human immunodeficiency virus \\ \hline
        14 & Hepatitis B virus \\ \hline
        9 & Betainfluenzavirus influenzae \\ \hline
        4 & Orthoflavivirus denguei \\ \hline
        3 & Simian immunodeficiency virus \\ \hline
        1 & Alcidodes juglans \\ \hline
        1 & Bacillus subtilis \\ \hline
        1 & Bacteroides thetaiotaomicron \\ \hline
        1 & Cannabis sativa \\ \hline
        1 & Capsicum baccatum \\ \hline
        1 & Echinocucumis hispida \\ \hline
        1 & Geissoloma marginatum \\ \hline
        1 & Homo sapiens \\ \hline
        1 & Human immunodeficiency virus \\ \hline
        1 & Kalanchoe fedtschenkoi \\ \hline
        1 & Leucosceptrum canum \\ \hline
        1 & Loxia curvirostra \\ \hline
        1 & Marinilactibacillus piezotolerans \\ \hline
        1 & Melanocenchris jacquemontii \\ \hline
        1 & Merops nubicus \\ \hline
        1 & Morbillivirus hominis \\ \hline
        1 & Phalaenopsis pulcherrima \\ \hline
        1 & Phormidesmis priestleyi \\ \hline
        1 & Rhodobacter maris \\ \hline
    \end{tabular}
    \caption{Overzicht van de geassocieerde soort voor de 200 peptides uit tabel~\ref{tab:peptides_distribution} die op soortniveau eindigen.
    Van de 200 peptides eindigt de meerderheid in een LCA die geassocieerd is met een virus (zoals HIV of influenza). Dit komt doordat er veel onderzoek gedaan wordt naar de verschillende bestaande rassen. Deze zitten allemaal in de UniProt Knowledgebase.}
    \label{tab:peptides_species}
\end{table}


\end{document}