%\documentclass[12pt,a4paper,faculty=we,language=nl]{ugent-doc}
%
%% Optional: margins and spacing
%%-------------------------------
%% Uncomment and adjust to change the default values set by the template
%% Note: the defaults are suggested values by Ghent University
%\geometry{bottom=2.5cm,top=2.5cm,left=2.5cm,right=2.5cm}
%%\renewcommand{\baselinestretch}{1.15} % line spacing
%
%
%% Font
%%------
%\usepackage[T1]{fontenc}
%\usepackage[utf8]{inputenc}
%% Comment/remove the two lines below to use the default Computer Modern font
%\usepackage{libertine}
%\usepackage{libertinust1math}

% Proper word splitting
%-----------------------
%\usepackage[dutch]{babel}

% Mathematics
%-------------

\documentclass[11pt,dutch,faculty=we,layout=titlefont,underline=false,titleUppercase=true,titleUnderline=true]{ugent2016-report}
\usepackage[dutch]{babel}
\usepackage{fontspec,unicode-math}
\usepackage{amsmath}

% Figures
%---------
\usepackage{graphicx, ugent2016-assets}
\usepackage{biblatex}
\graphicspath{{./figures/}}
\addbibresource{bibliography.bib}
\usepackage{hyperref}
\hypersetup{
    colorlinks,
    linkcolor={black},
    citecolor={blue!50!black},
    urlcolor={blue!80!black}
}

\author{Bram Devlaminck}
\title{Unipept:\newline geavanceerde indexstructuren \newline voor identificatie en analyse\newline van arbitraire peptiden}
%\subtitle{Hello}
\academicyear{2023--2024}
\programme{Informatica}
\studentnumber{01902993}
\email{bram.devlaminck@ugent.be}

% Bibliography settings
%-----------------------
%\usepackage[backend=biber, style=apa, sorting=nyt, hyperref=true]{biblatex}
%\addbibresource{./bibliography.bib}
% \usepackage{csquotes} % Suggested when using babel+biblatex
% local run procedure: (1) pdfLatex on main, (2) biber on main, (3) pdfLatex on main, (4) pdfLatex on main

% Hyperreferences
%-----------------
%\usepackage[colorlinks=true, allcolors=ugentblue]{hyperref}

% Whitespace between paragraphs and no indentation
%--------------------------------------------------
%\usepackage[parfill]{parskip}

% Input for title page
%----------------------

%% The title
%\thetitle{\color{ugentblue}Unipept:\newline geavanceerde indexstructuren \newline voor identificatie en analyse\newline van arbitraire peptiden}
%\thesubtitle{\color{black}Bram Devlaminck} % Optional

%% Note: a stricter UGent style could be achieved with, e.g.:
%\usepackage{ulem}
%\renewcommand{\ULthickness}{2pt}
%\thetitle{\uline{\color{ugentblue}INCLASS KAGGLE COMPETITION:\newline LOAN-DEFAULT PREDICTION}}
%% Note: do not forget to reset the \ULthickness to 1pt after invoking \maketitle
%\renewcommand{\ULthickness}{1pt}

%% The first (top) infobox at bottom of titlepage
%\infoboxa{\bfseries\large Masterproef}
%
%
%% The second infobox at bottom of titlepage
%\infoboxb{Promotoren:
%    \begin{tabular}[t]{l}
%        Prof.\ Dr.\ Peter Dawyndt \\ % note syntax 'short space'
%        Prof.\ Dr.\ Bart Mesuere
%    \end{tabular}
%}
%
%% The third infobox at bottom of titlepage
%\infoboxc{Begeleiders:
%    \begin{tabular}[t]{l}
%        Pieter Verschaffelt \\ % note syntax 'short space'
%        Tibo Vande Moortele
%    \end{tabular}
%}
%
%% The last (bottom) infobox at bottom of titlepage
%\infoboxd{Academisch jaar: 2023--2024} % note dash, not hyphen


\begin{document}

    \maketitle


    \setmonofont[Scale=MatchLowercase,Contextuals={Alternate}]{Jetbrains Mono}
% =====================================================================
% Cover
% =====================================================================


% =====================================================================
% Front matter
% =====================================================================

% ------------ TABLE OF CONTENTS ---------
    {\hypersetup{hidelinks}\tableofcontents} % hide link color in toc
    \newpage


% =====================================================================
% Main matter
% =====================================================================


    \section{Introductie}\label{sec:introductie}


    \section{Datasets}\label{sec:datasets}
    Om te testen hoe goed een implementatie is en hoe deze vergelijkt met al bestaande implementaties is het belangrijk om enkele representatieve datasets te hebben.
    De gebruikte datasets zijn allemaal eiwit databanken die een subset zijn van UniProt (meer specifiek UniProtKB 2023\_04) (TODO: cite uniprot).
    UniprotKB is een eiwit databank die in 2 grote delen op te splitsen valt.
    \begin{enumerate}
        \item Swiss-Prot: Dit is een kleinere, manueel gecureerde, dataset die 570 157 eiwit sequenties bevat.
        \item TrEMBL: Deze dataset bevat 251 600 768 sequenties en is dus veel groter dan Swiss-Prot.
        Het grote verschil is dat deze dataset \textbf{niet} manueel gecureerd is.
    \end{enumerate}

    Om het benchmarken van implementaties praktisch te houden maak ik gebruik van 2 Subsets hiervan.
    \paragraph{Swiss-Prot} Dit is de standaard Swiss-Prot databank die deel is van UniProt (zoals eerder uitgelegd).
    Een kort overzicht van alle statistieken is terug te vinden in tabel

    \begin{table}[h!]
        \centering
        \begin{tabular}{|c|c|}
            \hline
            Metriek & Waarde \\
            \hline\hline
            Totaal aantal sequenties & 569 619 \\ % TODO: waarom is dit anders dan wat op de site staat?
            Totale lengte & 205 954 074 \\
            Minimale proteïne lengte & 2 \\
            Maximale proteïne lengte & 35 213 \\
            Gemiddelde proteïne lengte & 361.56 \\
            Mediaan proteïne lengte & 295 \\
            \hline
        \end{tabular}
        \caption{Eigenschappen van de Swiss-Prot databank}
        \label{table:1}
    \end{table}

    \paragraph{Human-Prot} Deze dataset is samengesteld aan de hand van 3 referentiedatabanken die deel uit maken van UniProt.
    Deze werd geleverd aan me door mijn begeleider Pieter Verschaffelt
    \begin{enumerate}
        \item Human Genome~\cite{proteomes_homo_sapiens}
        \item Influenza B~\cite{proteomes_infuenza_b}
        \item Human Papillomavirus~\cite{proteomes_human_papillomavirus}
    \end{enumerate}

    Deze Human_

% =====================================================================
% End matter
% =====================================================================

% ------------ REFERENCES ------------
    \printbibliography[heading=bibintoc,title={Referenties}] % check if bibliography is in table of contents


% ------------ APPENDIX ------------
    \appendix


    \section{Appendix: Your first appendix}
    Insert some figure or table here.

    \newpage


    \section{Appendix: Your second appendix}
    The second appendix was forced to start on a new page.

\end{document}