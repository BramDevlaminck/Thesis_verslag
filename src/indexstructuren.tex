\chapter{Indexstructuren}\label{ch:indexstructuren}
Indexstructuren zijn een verzamelterm voor verschillende gestructureerde verzamelingen die toelaten \textbf{efficiënt} de geïndexeerde data te \textbf{doorzoeken}.
Bij indexstructuren voor strings bestaan er twee strategieën.
\begin{enumerate}
    \item \textbf{Voorverwerk de korte zoekstring} (van lengte $n$) zoals in het algroritme van Knuth-Morris-Pratt~\cite{knuth-morris-pratt}, Boyer-Moore-Horspool~\cite{boyer-moore-horspool} en het shift-AND-algoritme~\cite{shift-and}.
    \item \textbf{Voorverwerk de lange tekst} (van lengte $m$) zoals bij suffixbomen~\cite{mcCreight_first_suffixtree}, suffix arrays~\cite{suffix_array_first_mention} en (bidirectionele) FM-indexen~\cite{fm_index, bi-directional_fm_index}.
\end{enumerate}
Voor beide aanpakken bestaan er algoritmes om lineair in de tijd ten opzichte van de stringlengte de indexstructuur op te bouwen en te zoeken.
Er is echter een belangrijk detail.
\\ \\
Bij de strategie waar de korte zoekstring indexeren is het opbouwen in $O(n)$ tijd, en het zoeken in $O(m)$ tijd (met $n$ de lengte van de zoekstring, en $m$ de lengte van de tekst).
Hierbij is het \textbf{zoeken} dus \textbf{lineair in de tijd ten opzichte van de lengte van de totale tekst}.
Dit is nadelig wanneer er veel korte strings zijn, waarvoor elke keer de index gebouwd moet worden.
Waarna we bovendien nog de zoekoperatie, waarvoor de uitvoeringstijd lineair is in de lengte van de volledige tekst, moeten uitvoeren.
\\ \\
Indien we de lange tekst indexeren kan het opbouwen in $O(m)$ tijd en het zoeken in $O(n)$ tijd (opnieuw met $n$ de lengte van de zoekstring, en $m$ de lengte van de tekst).
Hierbij is het mogelijk om één keer de indexstructuur te bouwen voor de lange tekst, waarna elke korte string \textbf{in lineaire tijd ten opzichte van zijn eigen lengte gezocht} kan worden.
Het nadeel is echter dat het opbouwen van de indexstructuur voor een grote tekst traag kan worden.
\\ \\
Ons probleem komt net heel hard overeen met de tweede aanpak, waardoor we die aanpak zullen verkennen in deze masterproef.
We hebben een grote databank met erg veel proteïnes (een lange tekst) waarin we erg veel peptides (korte strings) zoeken.
In de eerste plaats willen we \textbf{exacte matches} kunnen zoeken, maar later ook naar een vorm van \textbf{inexacte} matches.
\\ \\
Aangezien de indexstructuur slechts eenmalig voor een bepaalde proteïnedatabank opgebouwd moet worden, ligt de \textbf{primaire restrictie bij het geheugenverbruik} tijdens het opbouwen.
Het blijft echter steeds belangrijk dat de indexstructuur in een redelijke tijd opgebouwd kan worden en performant genoeg is om snel een groot aantal peptiden te zoeken.
De richttijd die we ons vooropstellen om de indexstructuur op te bouwen voor UniProtKB is één à twee dagen.
Dit is een acceptabele tijdsduur aangezien de indexstructuur slechts om de acht tal weken opnieuw opgebouwd moet worden, wat de UniProt release frequentie is.
In de volgende hoofdstukken verkennen we verschillende indexstructuren om een proteïnedatabank te indexeren.


