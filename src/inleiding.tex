\chapter{Inleiding}\label{ch:introductie}

Eiwitten of proteïnen zijn alomtegenwoordig en spelen een belangrijke rol in ons dagelijkse leven.
Ze garanderen een correcte werking van belangrijke processen binnen elk organisme.
Hieronder vallen de levensprocessen van ons eigen lichaam, maar ook die van dieren, planten, bacteriën en zelfs virussen.
Om deze processen te analyseren zijn er meerdere benaderingen mogelijk.


\section{Genomica, transcriptomica \& proteomica}\label{sec:genomica-transcriptomica-&-proteomica}
Een eerste mogelijkheid is aan de hand van het onderzoeksgebied van de \textbf{proteomica}.
Dit is de studie van alle eiwitten die binnen een enkel organisme tot expressie kunnen komen.
Hierbij probeert men te begrijpen hoe eiwitten in elkaar zitten, hoe deze met elkaar en binnen een bepaalde omgeving met elkaar interageren en wat hun belangrijkste functie is.
In de biologie bestaat er een concept dat de omzetting naar deze eiwitten beschrijft, dit noemt men het centraal dogma.
Wanneer we er een analogie bij betrekken kunnen we centraal dogma omschrijven als het proces dat een gerecht doorloopt in de keuken van een restaurant.
Hierbij zijn de eiwitten het afgewerkte gerecht.
Naast proteomica bestaan er nog twee andere gerelateerde disciplines.
\\ \\
De eerste alternatieve discipline is \textbf{genomica}, het onderzoek naar het genoom.
Het genoom van een organisme is de collectie van al het DNA binnen een organisme.
Dit stelt voor welke proteïnen mogelijks door het organisme geconstrueerd kunnen worden.
Dit is het begin van het proces van het centraal dogma, wanneer we de analogie van eerder erbij betrekken is dit het receptenboek dat alle recepten bevat.
Een belangrijk verschil met proteomica is dat DNA instructies voorstelt voor de productie van alle mogelijke proteïnen die het organisme kan maken.
Het geeft dus geen informatie over de proteïnen die op dat moment in de tijd actief zijn.
Het is een voorstelling van wat het organisme kan, niet wat het op \textit{dit} moment aan het doen is.
Belangrijk is dat ongeveer 98\% van het menselijke genoom niet-coderend is, wat wil zeggen dat dit deel van het DNA niet omgezet kan worden naar een betekenisvolle proteïne.
In de plaats kunnen deze niet-coderende delen omgezet worden naar \textit{regulatory sequences}, niet-coderende genen, of andere componenten.
\\ \\
De andere discipline is de \textbf{transcriptomica}.
Deze discipline onderzoekt het transcriptoom van een organisme, wat de verzameling is van alle RNA moleculen die in het organisme aanwezig zijn.
Het transcriptoom is een belangrijke indicatie van welke delen uit het DNA effectief proteïnen encoderen.
Dit omdat RNA, meer specifiek messenger RNA (mRNA) en transfer RNA (tRNA), een belangrijk onderdeel is van het proces om DNA om te zetten naar proteïnen.
In de restaurant analogie is dit een kopie van één recept uit het receptenboek.
Dit recept beschrijft hoe het gerecht exact gemaakt moet worden.
Een visualisatie van de analogie, en de link met het centraal dogma van de biologie, kan terug gevonden worden in Figuur~\ref{fig:recipe}.
\\ \\
Onze focus ligt vooral binnen het veld van de \textbf{metaproteomica}.
Het \textit{meta} prefix zegt dat de te analyseren stalen niet van één organisme afkomstig zijn, maar van \textbf{meerdere organismen} (typisch binnen hetzelfde ecosysteem).
Dit maakt de analyse moeilijker aangezien proteïnen van verschillende organismen gelijkaardige aminozuursequenties kunnen hebben (al dan niet door toeval).
Meer specifiek is het doel van metaproteomica om op zoek te gaan naar de taxa en functies die horen bij een verzameling van eiwitfragmenten.
Deze eiwitfragmenten noemen we peptides wanneer ze bestaan uit twee of meer aminozuren en hun lengte beperkt blijft.
In de praktijk gaat dit over sequenties van ongeveer 2 tot 50 aminozuren lang.
Een veelvoorkomende categorie van peptiden die we zullen analyseren zijn tryptische peptiden.

\begin{figure}[H]
    \centering
    \includesvg[width=0.8\textwidth]{recipe_book_analogy}
    \caption{Het centraal dogma van de biologie kan makkelijk uitgelegd worden aan de hand van de analogie van een receptenboek. Het DNA komt overeen met het receptenboek dat de recepten bevat voor elk gerecht dat gemaakt kan worden. Eén kopie van één recept komt overeen met RNA. Een afgewerkt gerecht komt dan weer overeen met een proteïne.}
    \label{fig:recipe}
\end{figure}


\section{Tryptische peptiden}\label{sec:tryptische-peptiden}
Tryptische peptiden zijn peptiden die ontstaan na het knippen van proteïnen aan de hand van \textbf{trypsine}.
Dit is een protease (eiwitafbrekend enzym) dat proteïnen opsplitst in meerdere peptiden.
Er bestaan nog andere proteases, maar trypsine is veruit de populairste door zijn eenduidig gedrag en efficiëntie.
\\ \\
Trypsine zal eiwitten knippen na elk voorkomen van lysine (K) of arginine (R) indien het eerstvolgende aminozuur geen proline (P) is.
Deze vuistregel is echter niet perfect.
Soms mist trypsine een locatie waar volgens deze regel geknipt moet worden.
Dit noemen we een \textit{missed cleavage}.
Figuur~\ref{fig:trypsine} bevat een voorbeeld van de werking.

\begin{figure}[H]
    \centering
    \includesvg[width=0.9\textwidth]{trypsine_verwerking}
    \caption{Voorbeeld van de werking van trypsine op 2 proteïnen~\cite{phdPieterUnipept}. De aminozuren in het rood zijn lysine (K) of arginine (R), waarna trypsine knipt (behalve als het eerstvolgende aminzoruur proline (P) is). De tweede proteïne bevat een voorbeeld waar niet geknipt wordt na lysine, doordat het volgende aminozuur proline is.}
    \label{fig:trypsine}
\end{figure}

Om de peptiden uit een experiment te kunnen gebruiken bij computeranalyses moeten deze omgezet worden naar een stringrepresentatie.
Dit is is echter een moeilijk en ingewikkeld proces.
Eerst wordt de massa/ladingsverhouding (m/z) van peptiden aan de hand van een massaspectrometer gemeten.
Daarna worden deze resultaten aan de hand van diverse zoekprocessen (via zogenaamde zoekmachines) omgezet naar de stringvoorstelling van de peptide.
Deze sequenties vormen de input voor tools zoals Unipept.
\\ \\
Eén belangrijke consequentie van het gebruik van een massaspectrometer is dat Isoleucine (I) en Leucine (L) niet uit elkaar gehouden kunnen worden.
Deze aminozuren bestaan uit dezelfde atomen (C\textsubscript{6}H\textsubscript{13}NO\textsubscript{2}), maar deze zijn op een andere manier verbonden.
Hierdoor hebben ze een identieke massa, wat ervoor zorgt dat ze niet te onderscheiden zijn.


\section{Unipept}\label{sec:unipept-introductie}
Unipept~\cite{unipept_orig} biedt een ecosysteem van tools aan om stalen uit het onderzoeksveld van de metaproteomica te analyseren, maar er is ook een onderdeel (UMGAP~\cite{UMGAP_paper}) gericht op het analyseren van stalen uit de metagenomica.

\begin{itemize}
    \item \textbf{Unipept Web application~\cite{unipept_orig, unipept_web, unipept_tutorial, unipept_4}} Dit is de originele Unipept tool en is publiek beschikbaar op \url{https://unipept.ugent.be}.
    Met de gebruiksvriendelijke \textit{user interface} wordt analyseren van metaproteomica data beschikbaar gesteld.
    De resultaten van deze analyse worden aan de hand van visualisaties en tabellen aan de gebruiker voorgesteld.
    Deze kunnen vervolgens makkelijk geëxporteerd worden (bijv.~voor analyse in andere tools).
    \item \textbf{Unipept CLI\footnote{\textit{commandline interface}}~\cite{unipept_cli}} Dit is een \textit{power-user} tool voor de commandolijn om analyses uit te voeren op grotere stalen.
    \item \textbf{Unipept API\footnote{\textit{application programming interface}}~\cite{unipept_api, unipept_cli}} Dit is een collectie van \textit{endpoints} die andere applicaties (inclusief de Unipept CLI), toelaat om de functionaliteit van Unipept te integreren.
    \item \textbf{Unipept Desktop~\cite{unipept_desktop, unipept_desktop_2}} Dit is de recentste toevoeging aan het Unipept ecosysteem en laat toe dat onderzoekers niet noodzakelijk met de Unipept servers moeten communiceren om analyses uit te voeren.
    Deze applicatie combineert de voordelen van de web app, CLI en API en laat toe om lokaal stalen te analyseren, gebruikmakende van een gebruiksvriendelijke UI\@.

\end{itemize}

Op dit moment is Unipept \textbf{exclusief gericht op de analyse van tryptische peptiden}.
De reden hiervoor is de manier waarop de achterliggende indexstructuur opgebouwd wordt.
Dit opbouwen gaat in grote lijnen als volgt:

\begin{enumerate}
    \item Haal alle proteïnen en bijbehorende taxonomische en functionele annotaties op uit de UniProtKB databank~\cite{UniprotKB}.
    \item Splits deze proteïnen volgens de vuistregel die trypsine nabootst.
    \item Sla alle resulterende tryptische peptiden op in een databank, samen met hun voorberekende taxonomische en functionele metadata.
\end{enumerate}

Deze aanpak heeft als voordeel dat we op een efficiënte manier tryptische peptiden kunnen opzoeken (samen met de bijbehorende annotaties).
Er is echter een belangrijke keerzijde aan deze manier van werken.
Het zoeken van niet-tryptische peptiden (hieronder vallen ook peptiden met \textit{missed cleavage}) is problematisch.
Dit komt doordat tijdens het opbouwen van de Unipept indexstructuur de vuistregel strikt gevolgd wordt, en elke peptide in de indexstructuur strikt tryptisch is.
Op basis hiervan worden de taxonomische en functionele annotaties voor elke tryptische peptide voorberekend.
\\ \\
Op dit moment is er wel een \textit{workaround} die toelaat om peptiden met \textit{missed cleavages} toch te zoeken.
Dit heeft echter wel een significante impact op de performantie.
Deze verminderde performantie bij \textit{missed cleavages}, in combinatie met het compleet ontbreken van een manier om willekeurig gesplitste peptiden te zoeken, verklaart de nood aan een nieuwe indexstructuur.
\\ \\
Voor een gedetailleerdere beschrijving van Unipept en het onderzoeksveld van metaproteomica is het aangeraden om de inleiding van het doctoraat van Dr.~Pieter Verschaffelt te lezen~\cite{phdPieterUnipept}.
Dit vormde een duidelijke en goede basis voor deze inleiding.


\section{Probleemstelling}\label{sec:probleemstelling}
In deze masterproef zoeken we een oplossing voor het snel terugvinden van \textbf{willekeurige peptiden} in een eiwitdatabank.
Bij het vinden van een match moet het daarna mogelijk zijn de informatie op te halen die hoort bij alle proteïnen waarin de peptide voorkomt.
Binnen het onderzoeksgebied van de informatica kunnen we dit probleem als volgt herformuleren:
``In een grote verzameling van middellange strings (alle eiwitten in onze databank), moeten we voor een verzameling van korte strings (peptiden) terugvinden in welke van deze middellange strings ze voorkomen.''
We willen echter niet alleen maar vinden van welke proteïnen een peptide deel is.
Ook de bijbehorende taxonomische en functionele annotaties van de gematchte proteïne moeten zo snel mogelijk te vinden zijn.
Deze gevonden annotaties moeten efficiënt verwerkt kunnen worden om zo efficiënt het eindresultaat te bekomen.
Om dit te realiseren, worden waar mogelijk annotaties geaggregeerd tijdens het opbouwen van de indexstructuur.
\\ \\
Belangrijk hierbij is dat dit alles niet alleen \textbf{snel} gebeurt, maar dat we ook proberen \textbf{het vereiste geheugen tot een minimum te beperken}.
Wat als acceptabel beschouwd wordt, hangt af van de omgeving waarin de analyses uitgevoerd worden.
Voor stalen die geanalyseerd worden op een PC m.b.v.~Unipept Desktop is dit $\pm$ 16 GB RAM\@.
Voor grotere stalen waarvan de analyse op de Unipept servers uitgevoerd wordt mikken we op 0.5-2 TB geheugengebruik.
Dit komt overeen met een realistische configuratie voor een server die gericht is op het uitvoeren geheugenintensieve taken.
\\ \\
Tot slot willen we ook \textbf{semi-exacte matching} toevoegen tijdens het zoeken.
Hierbij willen we de mogelijkheid aanbieden om Isoleucine (I) en Leucine (L) gelijk te stellen aan elkaar.
Wanneer deze gelijkgesteld zijn wil dit zeggen dat op elke plaats waar een I staat, ook een L toegelaten wordt, en omgekeerd.
Door dit te doen kunnen we de beperkingen van een massaspectrometer opvangen.
\\ \\
Om dit allemaal te bereiken is het doel van deze thesis om meerdere datastructuren uit te werken, te implementeren in Rust, en tot slot te testen.
Het gebruik van Rust laat ons toe om extreem hoge performantie te verkrijgen (vergelijkbaar met C en C++~\cite{rustPerformantie}) in combinatie met \textit{memory safety}\footnote{\textit{Memory safety} is een eigenschap die verzekert dat programma's enkel gebruik kunnen maken van geldige geheugenlocaties en geen \textit{undefined behaviour} zoals \textit{buffer overflows}, \textit{dangling pointers} en andere geheugen gerelateerde fouten kunnen vertonen.}.
Bovendien zijn sommige delen van Unipept al geschreven in Rust (zie UMGAP~\cite{UMGAP_paper, UMGAP_source}).
Dit laat toe om waar mogelijk bestaande code te hergebruiken.


\section{Benchmarkdatasets}\label{sec:datasets}
Om de snelheid, het geheugengebruik en de correctheid van de onderzochte indexstructuren en zoekalgoritmen te bepalen zullen we \textbf{twee soorten benchmarkbestanden} gebruiken.
De eerste soort zijn de \textbf{proteïnedatabanken} waarmee we de indexstuctuur opbouwen.
De grootte van deze indexstructuur is het primaire criterium aangezien deze \textbf{volledig in het RAM-geheugen} moet passen, wat een harde limiet is.
Indien de index niet in het geheugen past, zal het programma niet uitgevoerd kunnen worden (of met een erg grote performance-penalty wanneer swapruimte\footnote{Dit is wanneer een computer schijfruimte gebruikt om het tekort aan RAM-geheugen op te vangen.} gebruikt wordt).
De andere soort bestanden bevatten de peptiden die we gaan proberen terugvinden in de indexstructuur.
Daarom zullen we voor de rest van deze masterproef hiernaar verwijzen als \textbf{peptidebestanden}.
Het hoofddoel van deze peptidebestanden is om de zoekperformantie te testen.
De tijd nodig om te zoeken is een zachte limiet aangezien we mikken voor hoge performantie, maar tragere code heeft enkel als gevolg dat een gebruiker langer moet wachten.
Alle bestanden die in de volgende secties besproken worden, kunnen teruggevonden worden in onze GitHub repository\footnote{\url{https://github.com/BramDevlaminck/Thesis_benchmarkdata}}.

\subsection{Proteïnedatabanken}\label{subsec:proteine-databanken}
Om te testen hoe goed een implementatie is en hoe deze zich verhoudt ten opzichte van bestaande implementaties, is het belangrijk om representatieve datasets te gebruiken.
Deze datasets zijn allemaal eiwitdatabanken die een subset vormen van \textbf{UniProtKB} (meer specifiek UniProtKB 2023\_04)~\cite{UniprotKB}.
UniProtKB zelf bestaat uit twee onderdelen (gegeven statistieken zijn voor release 2023\_04).
\begin{enumerate}
    \item \textbf{Swiss-Prot}: Dit is een kleinere, manueel gecureerde dataset met 570\thinspace157 eiwitsequenties.
    \item \textbf{TrEMBL}: Deze dataset bevat 251\thinspace600\thinspace768 sequenties en is dus veel groter dan Swiss-Prot.
    Een bijkomend verschil is dat deze dataset \textbf{niet} manueel gecureerd is, maar algoritmisch geannoteerd werd.
\end{enumerate}
Uiteindelijk is het doel om een indexstructuur voor UniProtKB op te bouwen waarbij de probleemstelling opgelost is.
UniProtKB is echter veel te groot om mee te werken tijdens het testen.
In plaats hiervan gebruiken we tijdens het ontwikkelen twee kleinere subsets van UniProtKB\@.
Eerst wordt een overzicht gegeven van de belangrijkste eigenschappen van UniProtKB, om daarna dieper in te gaan op de twee gebruikte subsets tijdens het testen.

\paragraph{UniProtKB}
Tabel~\ref{tab:uniprotKB_eigenschappen} bevat een overzicht van de belangrijkste statistieken voor de volledige UniProtKB 2023\_04 databank.
Belangrijk om op te merken is dat de totale databank uit 86\thinspace805\thinspace673\thinspace041 aminozuren bestaat.
Omgerekend is dit 86.81 GB aangezien een karakter opgeslagen wordt in één byte.
Dit verklaart onmiddellijk de keuze om gebruik te maken van kleinere datasets tijdens het testen.

\begin{table}[h!]
    \centering
    \begin{tabular}{ l l }
        Metriek                   & Waarde                                       \\
        \hline\hline
        Totaal aantal sequenties  & 248\thinspace842\thinspace516                \\
        Totale lengte             & 86\thinspace805\thinspace673\thinspace041 aa \\
        Minimale proteïnelengte   & 1 aa                                         \\
        Maximale proteïnelengte   & 45\thinspace354 aa                           \\
        Gemiddelde proteïnelengte & 348.84 aa                                    \\
        Mediaan proteïnelengte    & 278 aa                                       \\
        \hline
    \end{tabular}
    \caption{Eigenschappen van de volledige UniProtKB 2023\_04 databank. De afkorting \textit{aa} staat voor \textit{amino acids}.}
    \label{tab:uniprotKB_eigenschappen}
\end{table}

In Figuur~\ref{fig:uniprot_aminozuur} en~\ref{fig:uniprot_length} is een gedetailleerder overzicht van de aminozuurdistributie en verdeling van de proteïnelengtes terug te vinden.

\begin{figure}[h]
    \centering
    \includegraphics[width=0.6\linewidth]{uniprotKB_aminozuur_voorkomens}
    \caption{Aantal voorkomens per aminozuur voor alle proteïnen in de UniProtKB (2023\_04) databank.}
    \label{fig:uniprot_aminozuur}
\end{figure}

\begin{figure}[h]
    \centering
    \includegraphics[width=0.95\linewidth]{uniprotKB_length_distribution_large}
    \makebox[0pt][r]{% Similar to \llap
        \raisebox{2.2em}{%
            \includegraphics[width=0.65\linewidth]{uniprotKB_length_distribution_small}% Inserted image/inset
        }\hspace*{2em}%
    }%
    \caption{Lengtedistributie van de proteïnen in de UniProtKB (2023\_04) databank. De kleinere grafiek bevat een gedetailleerder overzicht van de distributie in het interval $[0, 1000[$.}\label{fig:uniprot_length}
\end{figure}

\paragraph{Swiss-Prot} Deze databank is één van de twee standaardonderdelen van UniProtKB\@.
Een kort overzicht van alle statistieken is terug te vinden in Tabel~\ref{tab:swissprot_eigenschappen}.
Figuur~\ref{fig:swissprot_aminozuur} en Figuur~\ref{fig:swissprot_length} geven meer inzicht in de distributie van de aminozuren en lengte van de proteïnen.

\begin{table}[h]
    \centering
    \begin{tabular}{l l}
        Metriek                   & Waarde                           \\
        \hline\hline
        Totaal aantal sequenties  & 569\thinspace619                 \\
        Totale lengte             & 205\thinspace954\thinspace074 aa \\
        Minimale proteïnelengte   & 2 aa                             \\
        Maximale proteïnelengte   & 35\thinspace213 aa               \\
        Gemiddelde proteïnelengte & 361.56 aa                        \\
        Mediaan proteïnelengte    & 295 aa                           \\
        \hline
    \end{tabular}
    \caption{Eigenschappen van de Swiss-Prot databank (UniProtKB 2023\_04). De afkorting \textit{aa} staat voor \textit{amino acids}.}
    \label{tab:swissprot_eigenschappen}
\end{table}


\begin{figure}[h]
    \centering
    \includegraphics[width=0.6\linewidth]{swissprot_aminozuur_voorkomens}
    \caption{Aantal voorkomens per aminozuur voor alle proteïnen in de Swiss-Prot databank uit UniProtKB 2023\_04.}
    \label{fig:swissprot_aminozuur}
\end{figure}

\begin{figure}[h]
    \centering
    \includegraphics[width=0.95\linewidth]{swissprot_length_distribution_large}
    \makebox[0pt][r]{% Similar to \llap
        \raisebox{2.2em}{%
            \includegraphics[width=0.65\linewidth]{swissprot_length_distribution_small}% Inserted image/inset
        }\hspace*{2em}%
    }%
    \caption{Lengtedistributie van de proteïnen in de Swiss-Prot databank. De kleinere grafiek bevat een gedetailleerder overzicht van de distributie in het interval $[0, 1000[$.}\label{fig:swissprot_length}
\end{figure}

Doordat het gebruikte invoerbestand reeds verwerkt werd door een deel van de Unipept pipeline, is er een klein verschil tussen het totaal aantal sequenties in Tabel~\ref{tab:swissprot_eigenschappen} en wat eerder aangegeven werd.
Hierbij worden onder andere sequenties met een onbekend taxon id verwijderd (538 in totaal), wat het kleine verschil verklaart.

\paragraph{Human-Prot} Deze dataset is samengesteld aan de hand van drie referentiedatabanken afkomstig uit UniProtKB\@.
Dit zijn de Human Genome~\cite{proteomes_homo_sapiens}, Influenza B~\cite{proteomes_infuenza_b} en Human Papillomavirus~\cite{proteomes_human_papillomavirus} databank.
Opnieuw komen deze allemaal uit UniProtKB 2023\_04.
\\ \\
Deze Human-Prot databank is kleiner dan Swiss-Prot, waardoor het testen tijdens ontwikkeling sneller is.
Tabel~\ref{tab:humanprot_eigenschappen} somt enkele belangrijke metrieken op over deze dataset.
Figuur~\ref{fig:humanprot_aminozuur} en~\ref{fig:humanprot_length} gaan dieper in op een aantal details.
\\
\begin{table}[ht]
    \centering
    \begin{tabular}{ l l }
        Metriek                   & Waarde                          \\
        \hline\hline
        Totaal aantal sequenties  & 82\thinspace695                 \\
        Totale lengte             & 30\thinspace293\thinspace046 aa \\
        Minimale proteïnelengte   & 2 aa                            \\
        Maximale proteïnelengte   & 35\thinspace991 aa              \\
        Gemiddelde proteïnelengte & 366.32 aa                       \\
        Mediaan proteïnelengte    & 204 aa                          \\
        \hline
    \end{tabular}
    \caption{Eigenschappen van de Human-Prot databank (UniProtKB 2023\_04). De afkorting \textit{aa} staat voor \textit{amino acids}.}
    \label{tab:humanprot_eigenschappen}
\end{table}

\begin{figure}[ht]
    \centering
    \includegraphics[width=0.7\linewidth]{humanprot_aminozuur_voorkomens}
    \caption{Aantal voorkomens per aminozuur voor alle proteïnen in de Human-Prot databank.}
    \label{fig:humanprot_aminozuur}
\end{figure}

\begin{figure}[ht]
    \centering
    \includegraphics[width=0.95\linewidth]{humanprot_length_distribution_large}
    \makebox[0pt][r]{% Similar to \llap
        \raisebox{2.2em}{%
            \includegraphics[width=0.65\linewidth]{humanprot_length_distribution_small}% Inserted image/inset
        }\hspace*{2em}%
    }%
    \caption{Lengtedistributie van de proteïnen in de Human-Prot databank. De kleinere grafiek bevat een gedetailleerder overzicht van de distributie in het interval $[0, 1000[$.}\label{fig:humanprot_length}
\end{figure}

We kunnen concluderen dat zo goed als \textbf{alle letters gebruikt} worden (ook al zijn er maar 20 aminozuren).
Dit komt doordat sommige letters eigenlijk een soort wildcard voorstellen.
Zo staat ``X'' voor elk mogelijk aminozuur, ``Z'' voor ``Q'' of ``E'',\ldots~\cite{amino_acid_codes}.
\\ \\
Verder valt ook te zien dat de verdeling van de proteïnelengtes in de UniProtKB, Swiss-Prot en Human-Prot datasets vergelijkbaar zijn.
Met andere woorden: \textbf{Swiss-Prot en Human-Prot zijn een representatieve kleinere voorstelling van UniProtKB\@}.
Dit laat ons toe om gebruik te maken van de Swiss-Prot en Human-Prot eiwitdatabanken en later de resultaten te veralgemenen en op te schalen naar UniProtKB\@.

\subsection{Peptidebestanden}\label{subsec:peptide-zoek-bestanden}
De zoekperformantie van onze indexstructuur is een erg belangrijk aspect.
Om dit te meten, hebben we bij elke eiwitdatabank een lijst aan peptiden die we proberen te zoeken.
Zowel voor de Swiss-Prot als Human-Prot databank zijn enkele datasets opgesteld.

\subsubsection{Swiss-Prot}
Voor deze proteïnedatabank hebben we enkele peptidebestanden voorzien.
Twee bestanden die gesampled zijn en een reeks aan real-life stalen.
De twee artificiële bestanden zijn zo gekozen dat de ene enkel tryptische peptiden bevat, terwijl de andere ook peptiden bevat met \textit{missed cleavages}.
De eerste kan dus op dit moment al efficiënt door Unipept verwerkt worden, terwijl dit voor de tweede niet mogelijk is.

\paragraph{Artificiële stalen}
Tabel~\ref{tab:artifiele_bestanden_statistieken} bevat in kolom twee en drie een kort overzicht met statistieken voor deze gesamplede bestanden.

\begin{table}[H]
    \centering
    \begin{tabular}{l l l l}
        Metriek                    & Swiss-Prot-TRYP                & Swiss-Prot-MC                  & Human-Prot                                  \\
        \hline\hline
        Totaal aantal sequenties   & 100\thinspace000               & 100\thinspace000               & 250\thinspace000                            \\
        Totale lengte              & 1\thinspace605\thinspace909 aa & 2\thinspace544\thinspace356 aa & 2\thinspace458\thinspace834\thinspace046 aa \\
        Minimale peptidelengte     & 5 aa                           & 5 aa                           & 1 aa                                        \\
        Maximale peptidelengte     & 50 aa                          & 93 aa                          & 12 aa                                       \\
        Gemiddelde peptidelengte   & 16.06 aa                       & 25.44 aa                       & 9.84 aa                                     \\
        Mediaan peptidelengte      & 13 aa                          & 23 aa                          & 10 aa                                       \\
        Aantal vindbare peptiden   & 67\thinspace375                & 62\thinspace581                & 250\thinspace000                            \\
        Aantal tryptische peptiden & 100\thinspace000               & 4107                           & 102\thinspace659                            \\
        \hline
    \end{tabular}
    \caption{Eigenschappen van de verschillende peptidebestanden. \textit{Swiss-Prot-TRYP} bevat de statistieken voor het Swiss-Prot peptidebestand met enkel tryptische peptiden. Hierbij komen er dus geen \textit{missed cleavages} voor. \textit{Swiss-Prot-MC} bevat net wel \textit{missed cleavages}. De laatste kolom bevat de statistieken voor het peptidebestand dat hoort bij de Human-Prot eiwitdatabank. De afkorting \textit{aa} staat voor \textit{amino acids}.}
    \label{tab:artifiele_bestanden_statistieken}
\end{table}

\paragraph{Experimentele stalen}
Om de performantie beter te beoordelen, gebruiken we ook enkele stalen uit experimenten met een kleine micro-organisme gemeenschap, namelijk SIHUMIx\footnote{Simplified human intestinal microbiota}~\cite{SIHUMI_first_introduction, SIHUMI_frequently_used}.
Aangezien dit effectieve stalen zijn, bevatten deze \textit{missed cleavages} die natuurlijk ontstaan zijn.
Tabel~\ref{tab:sihumi_zoekbestanden} bevat de belangrijkste statistieken voor elk peptidebestand.
Deze peptidebestanden worden in combinatie met de Swiss-Prot proteïnedatabank gebruikt tijdens het testen.

\begin{table}[H]
    \begin{minipage}{\linewidth}
        \centering
        \resizebox{\textwidth}{!}{ % use resizebox to textwidth since this needs to be scaled down in size a bit because otherwise it does not fit on the width of the screen
            \begin{tabular}{ l l l l l l l }
                Metriek                    & SIHUMI 03           & SIHUMI 05           & SIHUMI 05           & SIHUMI 08           & SIHUMI 11           & SIHUMI 14           \\
                \hline\hline
                Totaal aantal sequenties   & 25\thinspace000     & 25\thinspace000     & 24\thinspace424     & 25\thinspace000     & 24\thinspace998     & 25\thinspace000     \\
                Totale lengte              & 420\thinspace544 aa & 420\thinspace423 aa & 373\thinspace633 aa & 316\thinspace114 aa & 366\thinspace894 aa & 430\thinspace674 aa \\
                Minimale peptidelengte     & 6 aa                & 6 aa                & 6 aa                & 6 aa                & 6 aa                & 6 aa                \\
                Maximale peptidelengte     & 50 aa               & 50 aa               & 47 aa               & 43 aa               & 50 aa               & 50 aa               \\
                Gemiddelde peptidelengte   & 16.82 aa            & 16.82 aa            & 15.30 aa            & 12.64 aa            & 14.68 aa            & 17.23 aa            \\
                Mediaan peptidelengte      & 15 aa               & 16 aa               & 14 aa               & 12 aa               & 14 aa               & 16 aa               \\
                Aantal vindbare peptiden   & 2570                & 2698                & 3652                & 4135                & 3792                & 2761                \\
                Aantal tryptische peptiden & 17\thinspace263     & 162                 & 152                 & 207                 & 153                 & 242                 \\
                \hline
            \end{tabular}}
        \caption{Eigenschappen van de SIHUMIx peptidebestanden. Elke kolom stelt een staal voor met als bestandsnaam \texttt{S<XX>.txt}. Deze stalen kunnen teruggevonden worden in onze GitHub repository \protect\footnote{\url{https://github.com/BramDevlaminck/Thesis\_benchmarkdata}} onder de \texttt{SIHUMI}-folder. De afkorting \textit{aa} staat voor \textit{amino acids}.} % protect is needed to make footnote possible in a caption. This is combined with a minipage to place the footnote directly under the table
        \label{tab:sihumi_zoekbestanden}
    \end{minipage}
\end{table}

\subsubsection{Human-Prot}
Voor deze databank hebben we één peptidebestand bestaande uit HLA-peptiden.
Dit zijn \textbf{korte, niet-tryptische peptiden} uit het immunopeptidomics onderzoeksveld.
Hierdoor kan Unipept op dit moment niet gebruikt worden om dit soort stalen te analyseren.
Elke peptide in dit peptidebestand is een sample van een proteïne uit de Human-Prot databank.
Hierdoor zijn alle peptiden die we zoeken effectief vindbaar in de dataset.
Een kort overzicht van enkele eigenschappen is terug te vinden in de laatste kolom van Tabel~\ref{tab:artifiele_bestanden_statistieken}.
\\ \\
Appendix~\ref{ch:appendix-statistieken-peptidebestanden} bevat aanvullende grafieken over bovenstaande peptidebestanden.
Deze tonen voor elk bestand de distributie van de aminozuren en de distributie van de peptidelengte.
% TODO: hier ergens vermelden als apart onderdeeltje hoeveel geheugen nodig is op dit moment voor swissprot in unipept om index op te bouwen, hoe lang dit duurt, en hoe lang het zoeken tot match duurt voor swissprot met en zonder missed cleavage


\section{Benchmark hardware}\label{sec:benchmark-hardware}
Alle benchmarks werden uitgevoerd op een virtuele machine van team Unipept tenzij anders vermeld.
Tabel~\ref{tab:Matt_hardware} bevat een overzicht van de hardware van de fysieke machine en hoeveel toegekend is aan de VM\@.
Alle informatie omtrent Unipept infrastructuur kan teruggevonden worden op de Unipept GitHub wiki~\cite{unipept_infrastructure}.
\\ \\
Een deel van de kleinere testen werd ook lokaal uitgevoerd op een laptop.
Dit zal elke keer expliciet vermeld worden indien dit het geval was.
De gebruikte laptop is een M1 Pro MacBook Pro (14 inch) uit 2021.
Tabel~\ref{tab:macbook_hardware} bevat de exacte specificaties van deze laptop.
\\ \\
Naast een VM en eigen laptop hebben we ook toegang tot enkele andere machines indien hier nood toe zou zijn.
Een eerste optie is gebruikmakende van andere servers van Unipept.
Zo is het mogelijk om applicaties die op dit moment op verschillend servers draaien tijdelijk zo te herverdelen dat er een machine met 768 GiB ram ter beschikking komt.
Een andere optie is door gebruik te maken van de HPC van UGent\footnote{\url{https://docs.hpc.ugent.be/}}.
Hierop zijn nodes beschikbaar met 940 GiB RAM\@.
Tot slot heeft de CompOmics\footnote{Computational Omics} onderzoeksgroep aan UGent ook enkele machines staan die tot 2 TiB aan geheugen hebben.
Deze onderzoeksgroep werkt aan softwaretoepassingen voor het verwerken van proteomica-data, waar het onderzoeksgebied van de metaproteomica onder valt.

\begin{table}[ht]
    \centering
    \begin{tabular}{p{0.20\linewidth}p{0.45\linewidth}p{0.25\linewidth}}
        Onderdeel         & Fysieke server                                                                      & Virtuele Machine     \\
        \hline\hline
        CPU               & 2\times Intel Xeon 4410Y (12 cores / 24 threads, 2 - 3.9 GhZ, 30 MiB cache)         & 12 threads           \\
        RAM               & 768 GiB                                                                             & 128 GiB              \\
        Opslag            & 6\times 16 TiB HDD (3.5 inch, 7.2K RPM SATA), 4\times 3.84 TiB SSD (2.5 inch, SATA) & 1 TiB SSD, 4 TiB SSD \\
        Besturingssysteem & Debian 12 (met Proxmox)                                                             & Ubuntu 22.04 LTS     \\
        \hline
    \end{tabular}
    \caption{Hardwarespecificaties van de fysieke server en virtuele machine die gebruikt worden tijdens het testen. Deze virtuele machine draait samen met enkele andere VMs op de server.}
    \label{tab:Matt_hardware}
\end{table}

\begin{table}[ht]
    \centering
    \begin{tabular}{p{0.20\linewidth}p{0.54\linewidth}}
        Onderdeel         & hardware                                               \\
        \hline\hline
        Model             & MacBook Pro (14 inch, 2021)                            \\
        CPU               & 8-core M1 Pro, 6 performance cores, 2 efficiency cores \\
        RAM               & 16 GB (LPDDR5)                                         \\
        Opslag            & 512 GB SSD                                             \\
        Besturingssysteem & MacOS (14) Sonoma                                      \\
        \hline
    \end{tabular}
    \caption{Hardwarespecificaties van de gebruikte laptop voor kleinere testen. Elke keer testen op dit toestel uitgevoerd zijn, wordt dit expliciet vermeld.}
    \label{tab:macbook_hardware}
\end{table}

\section{Mogelijke oplossingen}\label{sec:mogelijke-oplossingen}
De essentie van onze probleemstelling is het snel vinden van een grote hoeveelheid korte strings in één erg lange string.
Dit is een vorm van stringmatching.
Hiervoor bestaan twee verschillende strategieën (die ook gecombineerd kunnen worden).
\begin{enumerate}
    \item \textbf{Verwerk de korte zoekstring op voorhand} (van lengte $n$) zoals in het algoritme van Knuth-Morris-Pratt~\cite{knuth-morris-pratt}, Boyer-Moore-Horspool~\cite{boyer-moore-horspool} en het shift-AND-algoritme~\cite{shift-and}.
    \item \textbf{Verwerk de lange tekst op voorhand} (van lengte $m$) zoals bij suffixbomen~\cite{mcCreight_first_suffixtree}, suffix arrays~\cite{suffix_array_first_mention} en (bidirectionele) FM-indexen~\cite{fm_index, bi-directional_fm_index}.
\end{enumerate}
Voor beide aanpakken bestaan er algoritmen om lineair in de tijd ten opzichte van de stringlengte de indexstructuur op te bouwen en te doorzoeken.
Er is echter een belangrijk detail.
\\ \\
Bij de strategie waar we de korte zoekstring verwerken op voorhand kan het opbouwen in $O(n)$ tijd en geheugen, en het zoeken in $O(m)$ tijd (met $n$ de lengte van de zoekstring, en $m$ de lengte van de tekst).
Hierbij is het \textbf{zoeken} dus \textbf{lineair in de tijd ten opzichte van de lengte van de totale tekst}.
Dit is nadelig wanneer er veel korte strings zijn, waarvoor elke keer extra werk moet gebeuren.
Daarna moeten we bovendien nog voor elke korte string de zoekoperatie uitvoeren, waarvoor de uitvoeringstijd lineair is in de lengte van de volledige tekst.
\\ \\
Indien we de lange tekst indexeren, kan het opbouwen in $O(m)$ tijd en het zoeken in $O(n)$ tijd en geheugen (opnieuw met $n$ de lengte van de zoekstring, en $m$ de lengte van de tekst).
Hierbij is het mogelijk om één keer de indexstructuur te bouwen voor de lange tekst, waarna elke korte string \textbf{in lineaire tijd ten opzichte van zijn eigen lengte gezocht} kan worden.
Het nadeel is echter dat het opbouwen van de indexstructuur voor een grote tekst traag kan worden, en bovendien veel geheugen kan innemen.
\\ \\
Het doorzoeken van UniProtKB naar matches van peptides komt overeen met de tweede aanpak, waardoor we die aanpak zullen verkennen in deze masterproef.
We hebben een grote databank met erg veel proteïnes (een lange tekst) waarin we erg veel peptiden (korte strings) zoeken.
In de eerste plaats willen we \textbf{exacte matches} kunnen zoeken, maar later ook naar een vorm van \textbf{inexacte} matches.
\\ \\
Aangezien de indexstructuur slechts eenmalig voor een bepaalde proteïnedatabank opgebouwd moet worden, ligt de \textbf{primaire restrictie bij het geheugengebruik} tijdens het opbouwen.
Het blijft echter steeds belangrijk dat de indexstructuur in een redelijke tijd opgebouwd kan worden en performant genoeg is om snel een groot aantal peptiden te zoeken.
De richttijd die we ons vooropstellen om de indexstructuur op te bouwen voor UniProtKB is maximaal één à twee dagen.
Dit is een acceptabele tijdsduur aangezien de indexstructuur slechts om de acht weken opnieuw opgebouwd moet worden, overeenkomstig met de release frequentie van UniProtKB.
In de volgende hoofdstukken verkennen we verschillende indexstructuren om een proteïnedatabank te indexeren.