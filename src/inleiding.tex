\chapter{Introductie}\label{ch:introductie}


\section{Inleiding}\label{sec:inleiding}

\subsection{Genomica, transcriptomica \& proteomica}\label{subsec:genomica-transcriptomica-&-proteomica}
Eiwitten zijn alomtegenwoordig en spelen een belangrijke rol in ons dagelijkse leven.
Ze garanderen een correcte werking van belangrijke processen binnen elk organisme.
Hieronder vallen de levensprocessen van ons eigen lichaam, maar ook die van dieren, planten, bacteriën en zelfs virussen.
Om deze processen te analyseren zijn er meerdere benaderingen mogelijk.
Een eerste mogelijkheid is aan de hand van het onderzoeksgebied van de proteomica.
Dit is de studie van alle eiwitten die binnen een enkel organisme tot expressie kunnen komen.
Hierbij probeert men te begrijpen hoe eiwitten in elkaar zitten, hoe deze met elkaar en binnen een bepaalde omgeving met elkaar interageren en wat hun belangrijkste functie is.
Naast proteomica bestaan er nog twee andere gerelateerde disciplines.
\\ \\
De eerste alternatieve discipline is genomica, het onderzoek naar het genoom.
Het genoom van een organisme is de collectie van al het DNA binnen een organisme.
Dit stelt voor welke proteïnen mogelijks door het organisme geconstrueerd kunnen worden.
Een belangrijk verschil met proteomica is dat DNA instructies voorstelt voor de productie van alle mogelijke proteïnen die het organisme kan maken.
Het geeft dus geen informatie over de proteïnen die op dat moment in de tijd actief zijn.
Het is een voorstelling van wat het organisme kan, niet wat het op \textit{dit} moment aan het doen is.
Belangrijk is dat ongeveer 98\% van het menselijke genoom niet-coderend is, wat wil zeggen dat dit deel van het DNA niet omgezet kan worden naar een betekenisvolle proteïne.
\\ \\
De andere discipline is de transcriptomica.
Deze discipline onderzoekt het transcriptoom van een organisme, wat de verzameling is van alle RNA moleculen die in het organisme aanwezig zijn.
Het transcriptoom is een belangrijke indicatie van welke delen uit het DNA effectief proteïnen encoderen.
Dit omdat RNA, meer specifiek messenger RNA (mRNA) en transfer RNA (tRNA), een belangrijk onderdeel is van het proces om DNA om te zetten naar proteïnen.
\\ \\
Onze focus ligt vooral binnen het veld van de metaproteomica.
het \textit{meta} prefix zegt dat de te analyseren stalen niet van één organisme zijn, maar van meerdere organismen (typisch binnen hetzelfde ecosysteem).
Dit maakt de analyse moeilijker aangezien proteïnen van verschillende organismen gelijkaardige aminozuursequenties kunnen hebben (al dan niet door toeval).
Meer specifiek is het doel van metaproteomica om op zoek te gaan naar de taxa die hoort bij een verzameling van peptiden.
Een een veelvoorkomende categorie van peptiden zijn tryptische peptiden.

\subsection{Tryptische peptiden}\label{subsec:tryptische-peptiden}
Tryptische peptiden zijn peptiden die ontstaan na het knippen van proteïnen aan de hand van trypsine.
Dit is een protease (eiwitafbrekend enzym) dat proteïnen opsplitst in meerdere peptiden.
Er bestaan nog andere proteases, maar trypsine is veruit de populairste door zijn eenduidig gedrag en efficiëntie.
\\ \\
Trypsine zal eiwitten knippen na elk voorkomen van lysine (K) of arginine (R) indien het eerstvolgende aminozuur geen proline (P) is.
Deze vuistregel is echter niet perfect.
Soms zal trypsine, een locatie waar normaal geknipt moet worden, missen.
Dit noemen we een \textit{missed cleavage}.
Figuur~\ref{fig:trypsine} bevat een voorbeeld van de werking.

\begin{figure}[H]
    \centering
    \includesvg[width=0.9\textwidth]{trypsine_verwerking}
    \caption{Voorbeeld van de werking van trypsine op 2 proteïnen. De aminozuren in het rood zijn lysine (K) of arginine (R), waarna trypsine knipt (behalve als het eerstvolgende aminzoruur proline (P) is). De tweede proteïne bevat een voorbeeld waar niet geknipt wordt na lysine, doordat het opeenvolgende aminozuur proline is~\cite{phdPieterUnipept}.}
    \label{fig:trypsine}
\end{figure}

Om de peptiden uit een experiment te kunnen gebruiken bij computeranalyses moeten deze omgezet worden naar een stringrepresentatie.
Dit is is echter een moeilijk en ingewikkeld proces.
Eerst wordt de massa/ladingsverhouding (m/z) van peptiden aan de hand van een massa spectrometer gemeten.
Daarna worden deze resultaten aan de hand van diverse zoekprocessen (via zogenaamde zoekmachines) omgezet naar de stringvoorstelling van de peptide.
Deze sequenties vormen de input voor tools zoals Unipept.

\subsection{Unipept}\label{subsec:unipept-introductie}
Unipept biedt allerhande tools aan om stalen uit het onderzoeksveld van de metaproteomica te analyseren, maar er is ook een onderdeel (UMGAP~\cite{UMGAP_paper}) gericht op het analyseren van stalen uit de metagenomica.

\begin{itemize}
    \item \textbf{Unipept Web application} Dit is de originele Unipept tool en is publiek beschikbaar op \url{https://unipept.ugent.be}.
    Met de gebruiksvriendelijke \textit{user interface} wordt analyseren van metaproteomica data beschikbaar gesteld.
    De resultaten van deze analyse worden aan de hand van visualisaties en tabellen aan de gebruiker voorgesteld.
    Deze kunnen vervolgens makkelijk geëxporteerd worden (bijv. voor analyse in andere tools).
    \item \textbf{Unipept CLI} Dit is een \textit{power-user} tool om analyses uit te voeren op grotere stalen.
    \item \textbf{Unipept API} Dit is een collectie van \textit{endpoints} die andere applicaties toelaat om de functionaliteit van Unipept integreren.
    \item \textbf{Unipept Desktop} Dit is de recentste toevoeging aan het Unipept ecosysteem en laat toe dat onderzoekers niet noodzakelijk met de Unipept servers moeten communiceren om analyses uit te voeren.
    Deze applicatie combineert de voordelen van de web app, CLI en API en laat toe om lokaal stalen te analyseren, gebruik makende van een gebruiksvriendelijke UI\@.

\end{itemize}

Op dit moment is Unipept gericht op de analyse van tryptische peptiden.
De reden hiervoor is de manier waarop de achterliggende indexstructuur opgebouwd wordt.
Dit opbouwen gaat in grote lijnen als volgt:

\begin{enumerate}
    \item Haal alle proteïnen en bijbehorende taxonomische en functionele annotaties op uit de UniProtKB databank.
    \item Splits deze proteïnen volgens de vuistregel die trypsine volgt.
    \item Sla alle resulterende tryptische peptiden op in een indexstructuur.
\end{enumerate}

Deze aanpak heeft als voordeel dat we op een efficiënte manier tryptische peptiden kunnen opzoeken (samen met de bijbehorende annotaties).
Er is echter een belangrijke keerzijde aan deze manier van werken.
Het zoeken van niet-tryptische peptiden (hieronder vallen ook peptiden met \textit{missed cleavage}) is inefficiënt.
Dit komt doordat tijdens het opbouwen van de Unipept indexstructuur de vuistregel gevolgd wordt, en elke peptide in de indexstructuur strikt tryptisch is.
\\ \\
Op dit moment is er wel een \textit{workaround} die toe laat deze peptiden toch te zoeken.
Dit heeft echter wel een significante impact op de performantie.
Dit verklaart dan ook de nood aan een nieuwe indexstructuur die toe laat volledig willekeurig gesplitste peptiden te zoeken.
\newline
Voor een gedetailleerdere beschrijving over Unipept en het onderzoeksveld van metaproteomica is het aangeraden om de inleiding van het doctoraat van Dr. Pieter Verschaffelt te lezen~\cite{phdPieterUnipept}.
Dit vormde een duidelijke en goede basis voor deze inleiding.


\section{Probleemstelling}\label{sec:probleemstelling}
Het probleem waarvoor een oplossing gezocht wordt is het snel terugvinden van willekeurige peptiden in een eiwitdatabank.
Bij het vinden van een match moet het daarna mogelijk zijn de informatie op te halen die hoort bij alle proteïnen die matchen.
Binnen het onderzoeksgebied van de informatica kunnen we dit probleem als volgt herformuleren:
``In een grote verzameling van middellange strings (alle eiwitten in onze databank), moeten we voor een verzameling van korte strings (peptiden) terugvinden in welke van deze middellange strings ze voorkomen.''
\\ \\
Belangrijk hierbij is dat dit niet alleen snel gebeurt, maar we ook proberen het vereiste geheugen tot een minimum te beperken.
Dit omdat de datasets waarmee gewerkt wordt extreem groot zijn.
\\ \\
Tot slot willen we ook inexacte matching toevoegen tijdens het zoeken.
Door dit te doen kunnen we beter omgaan met kleine fouten die voorkomen tijdens het uitlezen van de experimentele stalen.
Om dit allemaal te bereiken is het doel van deze thesis om meerdere datastructuren uit te werken, te implementeren in Rust, en tot slot te testen.
Het gebruik van Rust laat ons toe om extreem hoge performantie te verkrijgen (vergelijkbaar met C en C++~\cite{rustPerformantie}) in combinatie met \textit{memory safety}\footnote{\textit{Memory safety} is een eigenschap die verzekert dat programma's enkel gebruik kunnen maken van geldige geheugenlocaties en geen \textit{undefined behaviour} zoals \textit{buffer overflows}, \textit{dangling pointers} en andere geheugen gerelateeerde fouten kunnen vertonen.}.
Bovendien zijn sommige delen van Unipept al geschreven in Rust (zie UMGAP~\cite{UMGAP_paper, UMGAP_source}).
Dit laat toe om waar mogelijk bestaande code te hergebruiken.


\section{Benchmark Datasets}\label{sec:datasets}

\subsection{Proteïnedatabanken}\label{subsec:proteine-databanken}
Om te testen hoe goed een implementatie is en hoe deze zich verhoudt ten opzichte van bestaande implementaties is het belangrijk om representatieve datasets te gebruiken.
Deze datasets zijn allemaal eiwitdatabanken die een subset vormen van UniProtKB (meer specifiek UniProtKB 2023\_04)~\cite{UniprotKB}.
UniProt bestaat uit twee onderdelen (gegeven statistieken zijn voor release 2023\_04).
\begin{enumerate}
    \item Swiss-Prot: Dit is een kleinere, manueel gecureerde, dataset met 570 157 eiwitsequenties.
    \item TrEMBL: Deze dataset bevat 251 600 768 sequenties en is dus veel groter dan Swiss-Prot.
    Het belangrijkste verschil is dat deze dataset \textbf{niet} manueel gecureerd is.
\end{enumerate}
Om het vergelijken van de performantie van verschillende implementaties praktisch te houden maak ik tijdens het testen gebruik van 2 subsets hiervan.

\paragraph{Swiss-Prot} Deze databank is één van de twee standaard onderdelen van UniProt.
Een kort overzicht van alle statistieken is terug te vinden in tabel~\ref{tab:swissprot_eigenschappen}.
Figuur~\ref{fig:swissprot_aminozuur} en figuur~\ref{fig:swissprot_length} geven meer inzicht in de distributie van de aminozuren en lengte van de proteïnen.

\begin{table}[h!]
    \centering
    \begin{tabular}{l l}
        Metriek                   & Waarde      \\
        \hline\hline
        Totaal aantal sequenties  & 569 619     \\
        Totale lengte             & 205 954 074 \\
        Minimale proteïnelengte   & 2           \\
        Maximale proteïnelengte   & 35 213      \\
        Gemiddelde proteïnelengte & 361.56      \\
        Mediaan proteïnelengte    & 295         \\
        \hline
    \end{tabular}
    \caption{Eigenschappen van de Swiss-Prot databank uit UniProt 2023\_04 die gebruikt wordt om de indexstructuur op te bouwen.}
    \label{tab:swissprot_eigenschappen}
\end{table}


\begin{figure}[H]
    \centering
    \includegraphics[width=0.7\linewidth]{swissprot_aminozuur_voorkomens}
    \caption{Aantal voorkomens per aminozuur voor alle proteïnen in de Swiss-Prot databank uit UniProt 2023\_04.}
    \label{fig:swissprot_aminozuur}
\end{figure}

\begin{figure}[H]
    \centering
    \subfloat[Overzicht van de lengtedistributie van alle proteïnen]{\includegraphics[width=0.485\linewidth]{swissprot_length_distribution_large}}
    \hfill
    \subfloat[Ingezoomd beeld van de lengtedistributie tot 1000 aminozuren lang]{\includegraphics[width=0.485\linewidth]{swissprot_length_distribution_small}}
    \caption{Lengtedistributie van de proteïnen in de Swiss-Prot databank.}\label{fig:swissprot_length}
\end{figure}

Doordat het gebruikte invoerbestand reeds verwerkt werd door een deel van de Unipept pipeline, is er een klein verschil tussen het totaal aantal sequenties in tabel~\ref{tab:swissprot_eigenschappen} en wat eerder aangegeven werd.
Hierbij worden onder andere sequenties met een onbekend taxon id verwijderd, wat het kleine verschil verklaart.

\paragraph{Human-Prot} Deze dataset is samengesteld aan de hand van drie referentiedatabanken afkomstig van UniProtKB\@.
Dit zijn de Human Genome~\cite{proteomes_homo_sapiens}, Influenza B~\cite{proteomes_infuenza_b} en Human Papillomavirus~\cite{proteomes_human_papillomavirus} databank.
Opnieuw komen deze allemaal uit UniProt 2023\_04.

Deze Human-Prot databank is kleiner dan Swiss-Prot waardoor het testen tijdens ontwikkeling sneller is.
Tabel~\ref{tab:humanprot_eigenschappen} somt enkele belangrijke metrieken op over deze dataset.
Figuur~\ref{fig:humanprot_aminozuur} en~\ref{fig:humanprot_length} gaan dieper in op een aantal details.

\begin{table}[h!]
    \centering
    \begin{tabular}{ l l }
        Metriek                   & Waarde     \\
        \hline\hline
        Totaal aantal sequenties  & 82 695     \\
        Totale lengte             & 30 293 046 \\
        Minimale proteïnelengte   & 2          \\
        Maximale proteïnelengte   & 35 991     \\
        Gemiddelde proteïnelengte & 366.32     \\
        Mediaan proteïnelengte    & 204        \\
        \hline
    \end{tabular}
    \caption{Eigenschappen van de Human-Prot databank die gebruikt om de indexstructuur op te bouwen.}
    \label{tab:humanprot_eigenschappen}
\end{table}

\begin{figure}[H]
    \centering
    \includegraphics[width=0.7\linewidth]{humanprot_aminozuur_voorkomens}
    \caption{Aantal voorkomens per aminozuur voor alle proteïnen in de Human-Prot databank.}
    \label{fig:humanprot_aminozuur}
\end{figure}

\begin{figure}[H]
    \centering
    \subfloat[Overzicht van de lengtedistributie van alle proteïnen]{\includegraphics[width=0.485\linewidth]{humanprot_length_distribution_large}}
    \hfill
    \subfloat[Ingezoomd beeld van de lengtedistributie tot 1000 aminozuren lang]{\includegraphics[width=0.485\linewidth]{humanprot_length_distribution_small}}
    \caption{Lengtedistributie van de proteïnen in de Human-Prot databank.}\label{fig:humanprot_length}
\end{figure}

We kunnen concluderen dat zo goed als alle letters gebruikt worden (ook al zijn er maar 20 aminozuren).
Dit komt doordat sommige letters eigenlijk een soort wildcard voorstellen.
Zo staat ``X'' voor elk mogelijk aminozuur, ``Z'' voor ``Q'' of ``E'',\ldots

Ook zien we dat de verdeling van de proteïnelengtes in de Swiss-Prot en Human-Prot datasets vergelijkbaar zijn.
Het zwaartepunt ligt bij Swiss-Prot wel iets later (rond 100 vs rond 200).

\subsection{Peptide zoekbestanden}\label{subsec:peptide-zoek-bestanden}
De zoekperformantie van onze indexstructuur is een erg belangrijk aspect.
Om dit te meten hebben we bij elke eiwitdatabank een lijst aan peptiden die we proberen te zoeken.
Voor beide databanken zijn enkele datasets opgesteld.

\subsubsection{Swiss-Prot}
Voor deze databank hebben we enkele zoekbestanden voorzien.
Twee bestanden die gesampled zijn en een reeks aan real-life stalen.
De twee artificiële bestanden zijn zo gekozen dat de ene enkel tryptische peptiden bevat, terwijl de andere ook peptiden bevat met \textit{missed cleavage}.
De eerste kan dus op dit moment al efficiënt door Unipept behandeld worden, terwijl dit voor de tweede niet mogelijk is.

\paragraph{Artificiële stalen}
Tabel~\ref{tab:artifiele_bestanden_statistieken} bevat in kolom twee en drie een kort overzicht met statistieken voor deze gesamplede bestanden.

\begin{table}[H]
    \hspace*{-0.35cm} % this table is just a bit too wide to, just manually move it a bit to the left to "center" it
    \begin{tabular}{l l l l}
        Metriek                    & SP zonder \textit{missed cleavages} & SP met \textit{missed cleavages} & Human-Prot    \\
        \hline\hline
        Totaal aantal sequenties   & 100 000                             & 100 000                          & 250 000       \\
        Totale lengte              & 1 605 909                           & 2 544 356                        & 2 458 834 046 \\
        Minimale proteïnelengte    & 5                                   & 5                                & 1             \\
        Maximale proteïnelengte    & 50                                  & 93                               & 12            \\
        Gemiddelde proteïnelengte  & 16.06                               & 25.44                            & 9.84          \\
        Mediaan proteïnelengte     & 13                                  & 23                               & 10            \\
        Aantal vindbare peptiden   & 67 375                              & 62 581                           & 250 000       \\
        Aantal tryptische peptiden & 100 000                             & 4107                             & 102 659       \\
        \hline
    \end{tabular}
    \caption{Eigenschappen van de verschillende zoekbestanden. De kolommen die beginnen met SP bevatten de statistieken van de Swiss-Prot zoekbestanden. De laatste kolom bevat de statistieken voor het zoekbestand dat hoort bij de Human-Prot eiwitdatabank.}
    \label{tab:artifiele_bestanden_statistieken}
\end{table}

\paragraph{Experimentele stalen}
Om de performantie beter te beoordelen, gebruiken we ook enkele stalen uit experimenten met een kleine micro-organisme gemeenschap, namelijk SIHUMIx\footnote{Simplified human intestinal microbiota}~\cite{SIHUMI_first_introduction, SIHUMI_frequently_used}.
Aangezien dit effectieve stalen zijn, bevatten deze \textit{missed cleavages} die natuurlijk ontstaan zijn.
Tabel~\ref{tab:sihumi_zoekbestanden} bevat de belangrijkste statistieken voor elk zoekbestand.
Elke kolom stelt een staal voor met als bestandsnaam \texttt{S<XX>.txt}.
Deze zoekbestanden worden in combinatie met de Swiss-Prot databank gebruikt tijdens het testen.

\begin{table}[H]
    \centering
    \begin{tabular}{ l l l l l l l }
        Metriek                    & S03     & S05     & S07     & S08     & S11     & S14     \\
        \hline\hline
        Totaal aantal sequenties   & 25 000  & 25 000  & 24 424  & 25 000  & 25 000  & 25 000  \\
        Totale lengte              & 420 544 & 420 423 & 373 633 & 316 114 & 366 922 & 430 674 \\
        Minimale proteïnelengte    & 6       & 6       & 6       & 6       & 6       & 6       \\
        Maximale proteïnelengte    & 50      & 50      & 47      & 43      & 50      & 50      \\
        Gemiddelde proteïnelengte  & 16.82   & 16.82   & 15.30   & 12.64   & 14.68   & 17.23   \\
        Mediaan proteïnelengte     & 15      & 16      & 14      & 12      & 14      & 16      \\
        Aantal vindbare peptiden   & 2570    & 2698    & 3652    & 4135    & 3792    & 2761    \\
        Aantal tryptische peptiden & 17 263  & 162     & 152     & 207     & 155     & 242     \\
        \hline
    \end{tabular}
    \caption{Eigenschappen van de SIHUMIx zoekbestanden. Elke kolom stelt een staal voor met als bestandsnaam \texttt{S<XX>.txt}.}
    \label{tab:sihumi_zoekbestanden}
\end{table}

\subsubsection{Human-Prot}
Voor deze databank hebben we één zoekbestand bestaande uit HLA-peptiden.
Dit zijn korte, niet-tryptische peptiden uit het immunopeptidomics onderzoeksveld.
Hierdoor kan Unipept op dit moment niet gebruikt worden om dit soort stalen te analyseren.
Elke peptide in dit zoekbestand is een sample van een proteïne uit de Human-Prot databank.
Hierdoor zijn alle peptiden die we zoeken effectief vindbaar in de dataset.
Een kort overzicht van enkele eigenschappen is terug te vinden in de laatste kolom van tabel~\ref{tab:artifiele_bestanden_statistieken}.
\\ \\
Appendix~\ref{ch:appendix-statistieken-zoekbestanden} bevat aanvullende grafieken over bovenstaande zoekbestanden.
Deze tonen voor elk zoekbestand de distributie van de aminozuren en de distributie van de peptidelengte.


\section{Benchmark hardware}\label{sec:benchmark-hardware}
Alle benchmarks zijn uitgevoerd op een virtuele machine van Unipept.
Tabel~\ref{tab:Matt_hardware} bevat een overzicht van de hardware van de fysieke machine en hoeveel toegekend is aan de VM\@.
Al deze informatie, en ook over de andere machines gebruik door Unipept, kan terug gevonden worden op de Unipept GitHub wiki~\cite{unipept_infrastructure}.

\begin{table}[h!]
    \centering
    \begin{tabular}{p{0.20\linewidth}p{0.45\linewidth}p{0.25\linewidth}}
        Onderdeel         & Fysieke server                                                            & Virtuele Machine    \\
        \hline\hline
        CPU               & 2x Intel Xeon 4410Y (12 cores / 24 threads, 2 - 3.9 GhZ, 30 MiB cache)    & 12 threads          \\
        RAM               & 768 GiB                                                                   & 128 GiB             \\
        Opslag            & 6x 16 TiB HDD (3.5 inch, 7.2K RPM SATA), 4x 3.84 TiB SSD (2.5 inch, SATA) & 1 TiB SSD, 4TiB SSD \\
        Besturingssysteem & Debian 12 (met Proxmox)                                                   & Ubuntu 22.04 LTS    \\
        \hline
    \end{tabular}
    \caption{Hardwarespecificaties van de fysieke server en virtuele machine die gebruikt worden tijdens het testen. Deze virtuele machine draait samen met nog enkele andere VMs op de server.}
    \label{tab:Matt_hardware}
\end{table}
